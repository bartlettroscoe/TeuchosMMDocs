\documentclass[pdf,ps2pdf,11pt]{SANDreport}
\usepackage{pslatex}
%Local stuff
\usepackage{graphicx}
\usepackage{latexsym}
%\usepackage{color}
\usepackage[all]{draftcopy}
%
% Command to print 1/2 in math mode real nice
%
\newcommand{\myonehalf}{{}^1 \!\!  /  \! {}_2}

%
% Command to print over/under left aligned in math mode
%
\newcommand{\myoverunderleft}[2]{ \begin{array}{l} #1 \\ \scriptstyle #2 \end{array} }

%
% Command to number equations 1.a, 1.b etc.
%
\newcounter{saveeqn}
\newcommand{\alpheqn}{\setcounter{saveeqn}{\value{equation}}
\stepcounter{saveeqn}\setcounter{equation}{0}
\renewcommand{\theequation}
	{\mbox{\arabic{saveeqn}.\alph{equation}}}}
\newcommand{\reseteqn}{\setcounter{equation}{\value{saveeqn}}
\renewcommand{\theequation}{\arabic{equation}}}

%
% Shorthand macros for setting up a matrix or vector
%
\newcommand{\bmat}[1]{\left[ \begin{array}{#1}}
\newcommand{\emat}{\end{array} \right]}

%
% Command for a good looking \Re in math enviornment
%
\newcommand{\RE}{\mbox{\textbf{I}}\hspace{-0.6ex}\mbox{\textbf{R}}}
\newcommand{\CPLX}{\mbox{\textbf{C}}\hspace{-1.1ex}\mbox{l}\hspace{0.4ex}}

%
% Commands for Jacobians
%
\newcommand{\Jac}[2]{\displaystyle{\frac{\partial #1}{\partial #2}}}
\newcommand{\jac}[2]{\partial #1 / \partial #2}

%
% Commands for Hessians
%
\newcommand{\Hess}[2]{\displaystyle{\frac{\partial^2 #1}{\partial #2^2}}}
\newcommand{\hess}[2]{\partial^2 #1 / \partial #2^2}
\newcommand{\HessTwo}[3]{\displaystyle{\frac{\partial^2 #1}{\partial #2 \partial #3}}}
\newcommand{\hessTwo}[3]{\partial^2 #1 / (\partial #2 \partial #3)}



%\newcommand{\Hess2}[3]{\displaystyle{\frac{\partial^2 #1}{\partial #2 \partial #3}}}
%\newcommand{\myHess2}{\frac{a}{b}}
%\newcommand{\hess2}[3]{\partial^2 #1 / (\partial #2 \partial #3)}

%
% Shorthand macros for setting up a single tab indent
%
\newcommand{\bifthen}{\begin{tabbing} xxxx\=xxxx\=xxxx\=xxxx\=xxxx\=xxxx\= \kill}
\newcommand{\eifthen}{\end{tabbing}}

%
% Shorthand for inserting four spaces
%
\newcommand{\tb}{\hspace{4ex}}

%
% Commands for beginning and ending single spacing
%
\newcommand{\bsinglespace}{\renewcommand{\baselinestretch}{1.2}\small\normalsize}
\newcommand{\esinglespace}{}


\setcounter{tocdepth}{3}
\newtheorem{listing}{Listing}
\usepackage{hyperref}

\newtheorem{commandment}{Commandment}
\newtheorem{anticommandment}{Anti-Commandment}

\raggedright

% If you want to relax some of the SAND98-0730 requirements, use the "relax"
% option. It adds spaces and boldface in the table of contents, and does not
% force the page layout sizes.
% e.g. \documentclass[relax,12pt]{SANDreport}
%
% You can also use the "strict" option, which applies even more of the
% SAND98-0730 guidelines. It gets rid of section numbers which are often
% useful; e.g. \documentclass[strict]{SANDreport}

% ---------------------------------------------------------------------------- %
%
% Set the title, author, and date
%

\title{\center
Teuchos Memory Management Classes for C++ \\[2ex] A Comprehensive
Strategy for Safe and Efficient Memory Management in C++ for High
Performance Computing }

\author{
Roscoe Bartlett
}

\date{}

%Sandia National
%Laboratories\footnote{ Sandia is a multiprogram laboratory operated by Sandia
%Corporation, a Lockheed-Martin Company, for the United States Department of
%Energy under Contract DE-AC04-94AL85000.}, Albuquerque NM 87185 USA


% ---------------------------------------------------------------------------- %
% Set some things we need for SAND reports. These are mandatory
%
\SANDnum{SAND2010-xxx}
\SANDprintDate{January 2010}
\SANDauthor{
Roscoe Bartlett
}

% ---------------------------------------------------------------------------- %
% Build your markings. See example files and SAND Report Guide
%

\SANDreleaseType{Unlimited Release}
%\SANDreleaseType{Not approved for general release}
%\SANDmarkTopBottomCoverBackTitle{}
%\SANDmarkBottomCover{}
%\SANDmarkTopBottomCoverTitle{}
%\SANDmarkTop{}
%\SANDmarkBottom{}
%\SANDmarkTopBottom{}
%\SANDmarkCover{}
%\SANDmarkCoverTitle{}


% ---------------------------------------------------------------------------- %
%
% Start the document
%

\begin{document}

\maketitle


% ------------------------------------------------------------------------ %
% An Abstract is required for SAND reports
%

%\clearpage

%
\begin{abstract}
%


The ubiquitous use raw pointers in higher-level code is the primary
cause of all memory usage problems and memory leaks in C++
programs. This paper describes what might be considered a radical
approach to the problem which is to eliminate the use of all raw
pointers and all raw calls to {}\texttt{new} and {}\texttt{delete} in
higher-level C++ code.  Instead, a set of cooperating concrete
template classes developed in the Trilinos package Teuchos are used to
replace every use of raw C++ pointers in every use case where it
appears in high-level code.  Included in the set of memory management
classes are the typical reference-counted smart pointers similar to
{}\texttt{boost::shared\_ptr} (and therefore C++09
{}\texttt{std::shared\_ptr}) and {}\texttt{boost::shared\_array}.
However, what is missing in boost and the new standard library are
non-reference counted classes for remaining use cases where raw C++
pointers would need to be used.  These classes have a debug build mode
where nearly all programmer arrows are caught and gracefully reported
at runtime.  The default optimized build mode strips all runtime
checks and allows the code to perform as efficiently as raw C++
pointers with reasonable usage.  Also included is a novel approach for
dealing with the circular references problem that imparts little extra
overhead and is almost completely invisible to most of the code
(unlike the boost and therefore C++09 approach).  Rather than being a
radical approach this is simply the logical result of a trend in the
C++ development and standards community that started with
{}\texttt{std::auto\_ptr} and is continued (but not finished) with
{}\texttt{std::shared\_ptr}.

%
\end{abstract}
%

% ------------------------------------------------------------------------ %
% An Acknowledgement section is optional but important, if someone made
% contributions or helped beyond the normal part of a work assignment.
% Use \section* since we don't want it in the table of context
%
%\clearpage
%\section*{Acknowledgment}
%
%
%The format of this report is based on information found
%in~\cite{Sand98-0730}.

% ------------------------------------------------------------------------ %
% The table of contents and list of figures and tables
% Comment out \listoffigures and \listoftables if there are no
% figures or tables. Make sure this starts on an odd numbered page
%
\cleardoublepage   % TOC needs to start on an odd page
\tableofcontents
\listoffigures
\listoftables

% ---------------------------------------------------------------------- %
% An optional preface or Foreword
%\clearpage
%\section{Preface}
%Although muggles usually have only limited experience with
%magic, and many even dispute its existence, it is worthwhile
%to be open minded and explore the possibilities.

% ---------------------------------------------------------------------- %
% An optional glossary. We don't want it to be numbered
%\clearpage
%\subsection*{Nomenclature}
%\addcontentsline{toc}{section}{Nomenclature}
%\begin{itemize}
%\item[alohomora]
%spell to open locked doors and containers
%\end{itemize}

% ---------------------------------------------------------------------- %
% This is where the body of the report begins; usually with an Introduction
%

\SANDmain % Start the main part of the report


%
\section{Introduction}
%

A critical problem in computational science and engineering (CS\&E)
software as well as in other types of software developed in C++ is in
the effective and safe management of memory and data.  CS\&E software
often has the goal of high performance where arbitrary data copy leads
to undo overhead and can actually complicate the software in many
cases.  It is common for CS\&E software to share and pass around large
blocks of memory in order to do work efficiently.  At the most basic
level, large arrays of integral and floating point data are managed
along with more complex general objects and arrays of objects.  In
C++, the only universally accepted way to deal with memory for single
objects and arrays of objects is to use raw C++ pointers.  However,
raw C++ pointer facilities for the manipulation and sharing of basic
memory are inherently unsafe and error prone.  The problem is further
exacerbated when larger programs composed out of different separately
developed and maintained components are integrated together.
Assumptions about the origin, ownership, and process for reclaiming
memory and other resources remain the most basic problems with
lower-level C++ programming techniques and are unfortunately still
ubiquitous in the C++ community and even in the current C++
best-practices literature {}\cite{C++CodingStandards05,
EffectiveC++ThirdEdition}.  Our inability to effectively address the
basic problem of the usage of memory in large-scale modular C++ codes
affects every aspect of software quality and productivity and
undermines the most basic software verification foundation for our
codes.

C++ is an incredibility large and complex language that very few
people really know how to use in a confident and successful way.
Arguably the most serious problems in C++ are related to dynamic
memory management which must be used with any moderately complex
object-oriented program.  The built-in C++ support for dealing with
dynamic memory allocation with {}\texttt{new} and {}\texttt{delete} is
fundamentally incompatible with the built-in exception handling
mechanism using {}\texttt{try}, {}\texttt{throw}, and
{}\texttt{catch}.  One cannot effectively build large-scale integrated
software using just these low-level language features at the
application programming level.  Software developed this way segfaults
and leaks memory unpredictably and is nearly impossible to integrate
with other code.  The only successful way to use C++ to create complex
robust software is to develop and rigorously adopt a set of
programming idioms for safely dealing with memory.  By developing the
right support software and associated idioms, we make C++ programs
safer, better defined, faster to develop, and more efficient when run.

The reason that C++ is in this state of affairs is due to how C++ came
into being and how it evolved over many years {}\cite{stroustrup94,
stroustrup07}.  C++ was first developed in the early 1980s as an
extension to the popular C programming language and was first called
``C with Classes''.  At the time, high efficiency, very low runtime
overhead, and strong compatibility with C were critical requirements
to the success of the new language.  Without this, the original
creator of C++, Bjanre Stroustrup, concluded that C++ would be ``still
born'' {}\cite{stroustrup97}.  The first C++ compilers where little
more than preprocessors putting out C code on the backend which was
then compiled into executable binary code.

As the years went on, however, object-oriented programming was
refined, computers become faster with more memory, and it was realized
that more runtime support was required to enable some more advanced
usages of C++.  As new features were added to C++ to support new
programming idioms, a strong need for backward compatibility
constrained the design of the language.  The most unfortunate example
of this, which was already mentioned, is the fundamental
incompatibility of built-in dynamic memory management (i.e.\ using
{}\texttt{new}/{}\texttt{delete}) and built-in exception handling
(i.e.\ using {}\texttt{try}/{}\texttt{throw}/{}\texttt{catch}) that
was added more than a decade later {}\cite{stroustrup94}.

Therefore, because of the way that C++ ``evolved'' along with a strong
need for backward compatibility, we have a language that is a disaster
when used in raw form on complex large-scale programs.  Many
programming teams have exploited this natural capability of C++ to
create travesties of software which in turn have dumbfounded many a
C++ programmer (and entire teams) and have resulted in giving C++ a
bad name in the general software engineering community (see Section 6
``DDD Matters Today'' in {}\cite{DomainDrivenDesignQuickly} and ``The
Case of the Construction Blob'' in {}\cite[Chapter
9]{WorkingEffectivelyWithLegacyCode05} for a few examples).

However, C++ has some unique features that differentiate it from every
other language in wide use:

\begin{itemize}

{}\item Strong typing

{}\item Support for creating very efficient concrete data types with
efficiency on par with built-in data types that don't not require
dynamic memory management

{}\item Support for operator overloading

{}\item Support for object-oriented programming

{}\item Support for generic programming (i.e.\ templates)

{}\item Powerful turning-complete compile-time programming mechanism
(i.e.\ template meta-programming)

\end{itemize}

No other programming language with wide availability has this set of
features.  For instance, C++ can be used to create class libraries for
capabilities like automatic differentiation {}\cite{ref:ad} for
computing derivatives of functions (see the Trilinos
Sacado\footnote{\texttt{http://trilinos.sandia.gov/packages/sacado/}}
package {}\cite{phippsEtAl2006}) that achieves a level of generality
and efficiency that has no rival in a software library in any other
programming language.

It is this unique feature set that is C++'s saving grace.  In C++, you
can actually develop a set of new data types that in essence can be
used to develop new programming environments in C++.  This essentially
allows you to define a new programming language within C++ with a
level of efficiency and flexibility that does not exist in any other
programming language.  This is exactly what this paper advocates
with respect to basic memory management in C++; developing a new
higher-level programming language in C++ for abstracting and
encapsulating all raw memory usage as well as dynamic memory
management that is very compatible with the built-in C++ exception
handling mechanism.

This paper describes a set of low-level C++ classes and supporting
software in the Teuchos package in
Trilinos\footnote{http://trilinos.sandia.gov} that is used to
encapsulate all raw pointers and enable strong debug runtime checking
while allowing for very high performance in optimized builds.

The Teuchos memory management classes and the idioms that they help to
define (which are described in this paper) do not remove the need for
programmers to learn and understand the intricate details of the C++
memory model and type system.  On the contrary, learning to
effectively use these memory management classes requires more effort
over just learning raw C++.  However, the payoff is that the programs
that result from the use of these classes and idioms will be more
likely to be correct on first writing, will be easier to debug when
there are defects, will be easier and safer to maintain, and will be
more self documenting (which helps all of the above).  In fact, the
self-documenting expressiveness of the resulting programs written
using these classes and idioms is unmatched in any other programming
language currently in popular use, including Java and Python.  This
statement will be backed up throughout this paper and then reiterated
in the final conclusions section.

The remainder of this paper assumes that the reader has some basic
knowledge of C++ and is somewhat familiar with smart reference-counted
pointer classes like {}\texttt{boost::shared\_ptr} (which is the basis
for the new C++09 {}\texttt{std::shared\_ptr} class).  The Teuchos
equivalent for these smart pointer classes is {}\texttt{Teuchos::RCP}
which we abbreviate as just {}\texttt{RCP} in sample code.  If the
reader is not familiar with the basics of smart reference-counted
pointer classes, then they should refer to
{}\cite{RefCountPtrBeginnersGuide} and {}\cite{C++CodingStandards05}.
If the reader is not familiar with fundamental C++ concepts like
implicit type conversions, templates, object lifetime models, raw
reference and pointers, and other basic topics, then some more basic
background will be needed.  However, specific references to basic C++
material in books like {}\cite{EffectiveC++ThirdEdition, stroustrup97,
C++CodingStandards05} are make throughout this document.  So if you
are a novice C++ programmer and are willing to look up the mentioned
references, then this paper can be a good guide to help you learn this
basic C++ material as well.

A final warning: the material in this paper is pretty detailed and
will take a significant investment in time and experience with writing
code with the memory management classes using the idioms describe here
before a developer will be proficient.  It takes years just to master
raw C++ so it should be no surprise that learning a new set of idioms
to fix a large number of the problems with raw C++ will also take a
significant amount of time and effort.  What is needed is a culture
change in the C++ programming community where the memory management
classes and the idioms described here are taught at a very early
stage; much like the STL is now being taught in introductory C++
courses.  What we need is a revolution in C++ education but we have to
start somewhere and that is what this paper is all about, getting
started and on the road to a better generation of C++ programmers and
C++ software.

The reminder of this paper is organized as follows.  First, the
fundamental problems with raw C++ pointers is described in Section
???.  ToDo: Finish describing the other sections.


%
{}\section{Fundamental problems with raw C++ pointers}
%

In this section, I summarize some of the fundamental problems with
basic C++ features related to raw pointers.  What I am going to argue
is that while some people will claim that C++ pointers are strongly
typed, I will show that they are actually very weakly typed in many
respects and how this weak typing is the cause of many programming
errors that result in incorrect programs and segfaults.

In the following examples, I will use the simple classes shown in
Listing {}\ref{listing:Simple_A_B}:

\begin{listing}:\\
\label{listing:Simple_A_B}
{\small\begin{verbatim}
  class A {
  public:
    ...
    void incrementA() { ++(*char_ptr_); }
    ...
  private:
    char *char_ptr_;
  };

  class B : public A {
  public:
    ...
    void incrementB() { ++(*int_ptr_); }
    ...
  private:
    int size_;
    int *int_ptr_;
  };
\end{verbatim}}
\end{listing}

The concrete class hierarchy in Listing {}\ref{listing:Simple_A_B} was
chosen to demonstrate some insidious and perhaps less well known flaws
in the C++ type system when dealing with raw C++ pointers.


%
{}\subsection{Problems using raw C++ pointers for handling single objects}
%

There are a number of problems with using raw C++ pointers to manage
single objects.  For example, given a class of type {}\texttt{B} in
Listing {}\ref{listing:Simple_A_B} consider a pointer declared as:

{\small\begin{verbatim}
  B some_b(...);
  B *b_ptr = &some_b;
\end{verbatim}}

Some of the legitimate things that you can do with this pointer are:

{\small\begin{verbatim}
  // Call member functions
  b_ptr->incrementA();
  b_ptr->incrementB();
  // Extract reference
  B &b_ref = *b_ptr;
  // Copy pointer
  B *b_ptr2 = b_ptr;
  // Implicit conversion to const
  const B *b_ptr3 = b_ptr;
  // Implicit conversion to base type
  A *a_ptr4 = b_ptr;
\end{verbatim}}

However, nothing good can {}\underline{ever} come of any of the
following operations when a pointer is only pointing to a single
object:

{\small\begin{verbatim}
  b_ptr++
  b_ptr--
  ++b_ptr
  --b_ptr
  b_ptr+i
  b_ptr-i
  b_ptr[i]
\end{verbatim}}

No C++ compiler I have ever worked with will even issue a warning when
array operations are invoked on a raw C++ pointer for which it is
clear is only pointing to a single object.

The problem here of course is that there is no way to tell the C++
compiler that a raw pointer is only pointing to a single object.  With
respect to differentiating single objects and arrays of objects, C++
pointers are untyped and the compiler provides no help whatsoever in
statically asserting correct usage.  This is strike one for the
notation that C++ pointers being strongly typed!


%
{}\subsection{Problems using raw C++ pointers for handling arrays of
objects}
\label{sec:problem-with-raw-array-pointers}
%

When considering the semantics of raw C++ pointers one realizes that
raw pointers are really designed primarily for dealing with contiguous
arrays of objects (save for one exception that I mention below).  I
say this because almost every operation that C++ defined for raw
pointers makes sense and is fairly well defined when raw C++ pointers
are pointing with contiguous arrays of objects.  I will not review
every valid C++ operation for raw pointers to contiguous arrays of
objects (see {}\cite{stroustrup97} for a complete listing).  Instead,
I want to show a few examples where the C++ type system using raw
pointers falls flat on its face when dealing with arrays of memory.

One particularly troubling example where the C++ type system fails
when dealing with raw C++ pointers to contiguous arrays of memory is
shown in Listing {}\ref{listing:BadArrayPointerConversion}.

\begin{listing}:\\
\label{listing:BadArrayPointerConversion}
{\small\begin{verbatim}

  void foo( const int n )
  {
    B *b_array = new[n];
    ...
    A *a_array = b_array; // Compiles just fine :-(
    for (int i = 0; i < n; ++i) {
      a_array[i]->incrementA(); // KABOMMMMM!

    }
  }

\end{verbatim}}
\end{listing}

There are a lot of beginning and even some more experienced C++
programmers that would think that the C++ code in Listing
{}\ref{listing:BadArrayPointerConversion} is just fine.  I have seen
this mistake made more than once and the resulting program may seem to
run okay in some cases but in the above case will almost certainly
segfault right away.  The above code fragment is wrong, wrong, wrong
as described in {}\cite[Gotcha \#33]{C++Gotchas03} and {}\cite[Item
\#100]{C++CodingStandards05}.  I will not go into great detail about
what happens here, but converting from an array of type {}\texttt{B}
it a base of type {}\texttt{A} is almost always asking for disaster
because the alignment of the base type {}\texttt{A} will be wrong
according to the full type {}\texttt{B} (again see {}\cite[Gotcha
\#33]{C++Gotchas03} all the gory details).  As a result, for the
second iteration {}\texttt{i=1}, the embedded pointer in
{}\texttt{a\_array[1].char\_ptr\_} is pointing to garbage because on
most 32 bit machines with most compilers, the address in
{}\texttt{a\_array[1].char\_ptr\_} is actually the binary
representation of the integer {}\texttt{b\_array[0].size\_}.
Therefore, calling {}\texttt{a\_array[1]->increment()} on most 32 bit
machines is equivalent to performing:

{\small\begin{verbatim}
  ++(*reinterpret_cast<char*>(b_array[0].size_)); // KABOMMMMM!
\end{verbatim}}

If this sort of thing comes as a surprise to you, you should probably
fear using raw memory in C++ more than you currently do and you should
seriously consider the safer approach to encapsulating raw memory
usage that is being advocated in this paper.

So how did C++ come to allow such a completely wrong and dangerous
operation like shown in Listing
{}\ref{listing:BadArrayPointerConversion}?  It is because of the
untyped dual nature of raw C++ pointers in trying to handle both
single objects and contiguous arrays of objects with one set of what
should be non-overlapping semantics.  The ability to cast raw C++
pointers from derived types to base types only ever generally makes
sense when the pointer is pointing to a single object and will not be
interpreted as a pointer to a contiguous array of objects.  Note that
C does not have this problem since there is no such thing as type
derivation and the designers of C never even envisioned that raw C
pointers would be used for such a thing.  However, when the original
designer of C++ adopted the C type system along with raw pointers and
tried to apply it to an object-oriented language, he inadvertently
opened up a number of serious language gotchas that we are still
living with to this day.  This is strike two for the notation that C++
pointers are strongly typed!

ToDo: Find another issue with type safety of raw C++ pointers so you
can write ``Strike three, you are out!''.


%
{}\subsection{Problems with the incompatibility of
{}\texttt{new}/{}\texttt{delete} and {}\texttt{try}/{}\texttt{catch}}
%

The use of raw pointers and raw calls to {}\texttt{new} and
{}\texttt{delete} is also fundamentally incompatible with the built-in
C++ exception handling mechanism using {}\texttt{try} and
{}\texttt{catch}.  For example, the following code will leak memory
if the function {}\texttt{someFunc()} throws a C++ exception:

{\small\begin{verbatim}
  void foo()
  {
    A *a = new A(...);
    someFunc(); // Could throw an exception
    delete a; // Will never be called if someFunc() throws!
  }
\end{verbatim}}

According to current C++ best practices relating to memory management
and exception handling as described in {}\cite[Item
{}\#29]{EffectiveC++ThirdEdition} and {}\cite[Item
{}\#71]{C++CodingStandards05}, code like shown above that leaks memory
is totally unacceptable in production quality C++ code.  This
fundamental incompatibility of the built-in C++ dynamic memory
management facilities using {}\texttt{new}/\texttt{delete} and the
built-in exception handling mechanism using
{}\texttt{try}/\texttt{catch} was clear even to the committee that
created the official C++ standard in 1998.  However, again, because of
the need for backward compatibility they were powerless to fix the
problem at the language level.  Instead, the C++ standards committee
included the first standard C++ smart pointer class;
{}\texttt{auto\_ptr}.  I will not go into the details about
{}\texttt{auto\_ptr} other than to say that is solved only the most
basic problem with raw C++ pointers and that is that it ensures that
memory will be reclaimed when exceptions are thrown.  For example, the
following rewritten function will not leak any memory when
{}\texttt{someFunc()} throws:

{\small\begin{verbatim}
  void foo()
  {
    std::auto_ptr<A> a(new A(...));
    someFunc(); // Could throw and exception
    // NOTE: delete will get called on the A object no matter how this
    // function exists (i.e. normal exit or with a throw) since it is
    // called by the destructor of the stack object 'a' of type
    // std::auto_ptr<A>. 
  }
\end{verbatim}}

The introduction of {}\texttt{std::auto\_ptr} is the first example
that I can think of where a user-defined type was added to the
standard C++ library in order to define an idiom meant to fix a
fundamental C++ language flaw.  Note that I said ``flaw'' and not
``deficiency''.  It is generally excepted in most modern programming
languages that the language proper will not support every programming
model or idiom that is of general interest and instead class libraries
are provided to fill the gaps.  The problem with C++ pointers is
different in that it is the language definition itself which is flawed
and is not just simply missing some desirable feature.  You could
argue that what C++ is really missing is garbage collection (GC) but
even that is not the case because to add GC would be fundamentally
incompatible with the current user-controlled memory management
facility using {}\texttt{new} and {}\texttt{delete}.  There is a lot
of C++ code out there that requires that destructors for objects be
called exactly when expected such as when {}\texttt{delete} is called.
Any form of language-supported GC will break some backward
compatibility of C++ and therefore we may never see a C++ standard
with full GC.

The boost library and the up-coming C++09 standard add more types that
I will mention below that continue in this trend of providing new
user-defined types and idioms to address fundamental C++ language
flaws and deficiencies.  However, as I describe in meat of this paper,
both the boost and the C++09 standard library fall short of providing
a complete and comprehensive solution to the problems with raw C++
pointers and raw access to memory.

Note that the upcoming C++09 standard as it is currently defined (at
least the time of this writing) will do nothing to fix the majority of
these nonsensical raw C++ pointer gotchas because to do so would
destroy backward compatibility of many millions of lines of existing
C++ code.  Because of the need for backward compatibility, we cannot
rely on any future C++ standard to fix the basic problems with raw C++
pointers.  Instead, the meat of this paper advocates using new C++
user-defined types to create a new safer type-system in C++ and
avoiding the direct use of raw C++ pointers in all but the lowest
level of code.

One has to wonder how the addition of such a feature as exception
handling which is so incompatible with a primary language feature like
basic memory management was ever accepted into the C++ language
standard.  It is clear what the advantages of exception handling are
in C++ (and any other language) but since the naive use of such a
feature will result in memory leaks one just have to wonder how this
was justified.  However, all is not lost since we don't have to write
naive memory management code and fixing this problem is one of the
main features of the memory management classes proposed here.


%
{}\section{Problems with popular approaches for addressing memory
management in C++}
%

Because of some of the obvious problems with using raw C++ pointers to
access raw memory and using raw calls to {}\texttt{new} and
{}\texttt{delete} to perform dynamic memory management, various
authors have advocated a number of different approaches for addressing
these problems.  I will review a few of these approaches and I
will argue why these are much too sub-optimal.


%
{}\subsection{Problems with using {}\texttt{std::vector} for handling
all arrays}
%

A very common approach that I have seen used to try to get around
using raw C++ pointers for managing contiguous arrays of data is to
use the container class {}\texttt{std::vector} in {}\underline{every}
use case where a raw C++ array or pointer to an array would be used.
Before I describe use cases where {}\texttt{std::vector} is being
poorly used, first let's review what {}\texttt{std::vector} is and
what it is good for.  The standard library class
{}\texttt{std::vector} is a general-purpose concrete contiguous data
container class for storing and retrieving value objects\footnote{See
Section {}\ref{sec:value-and-reference-types} for a definition of
``Value Types''.}.  What makes using {}\texttt{std::vector} attractive
as compared to a simple class that you would write yourself is:

\begin{itemize}

{}\item\texttt{std::vector} is a Standard Template Library (STL)
complaint data container which makes it easy to use with STL-like
generic algorithms.

{}\item\texttt{std::vector} contains functions for efficiently
expanding and shrinking the size of the array that can have
platform/compiler specific optimizations with much better performance
than what you would roll on your own.

{}\item\texttt{std::vector} is standardized so you can use it as a
means for interoperability with other software in appropriate
situations.

\end{itemize}

These are pretty much the advantages of using {}\texttt{std::vector}
over other alternatives.  When used as a general purpose data
container where you will be changing the size of the array on the fly,
{}\texttt{std::vector} is convenient, general, and efficient (just
what components from a standard library should be).  However, in other
use cases, {}\texttt{std::vector} is far from convenient, general, or
efficient.  As one example, consider using {}\texttt{std::vector} to
replace raw C++ pointers for array arguments in {}\underline{all} C++
functions as some authors have suggested (e.g.\ see [???]).  For
example, consider the function that adds elements from one array into
another array expressed with raw C++ pointers shown in Listing
{}\ref{listing:addArrayIntoArray-raw}.

\begin{listing}:\\
\label{listing:addArrayIntoArray-raw}
{\small\begin{verbatim}
  template<class T>
  void addArrayIntoArray( const int n, const T a[], T b[] )
  {
    for (int i = 0; i < n; ++i)
      b[i] += a[i];
  }
\end{verbatim}}
\end{listing}

The advantages of the function in Listing
{}\ref{listing:addArrayIntoArray-raw} are that a) it is clean, b) the
arrays of data can be sub-views of large arrays, and c) it will yield
very fast code.  Of course the problem with the above function
{}\texttt{addArrayIntoArray(...)} is that is uses raw C++ pointers.  How
does the function {}\texttt{addArrayIntoArray(...)} know that
{}\texttt{a} and {}\texttt{b} are valid pointers and really point to
valid arrays of data with at least {}\texttt{n} elements.  It is
impossible for the function {}\texttt{addArrayIntoArray(...)} to
assert anything about the data and completely relies on the caller of
the function to validate the data.  Even in a debug build of the code,
there is no way for the implementation of the function
{}\texttt{addArrayIntoArray(...)} to validate that the preconditions
concerning arguments have been met.  This is not good and does not
allow for even the most basic approaches for defensive programming.

Therefore, some C++ programmers look at this and then they change
functions like {}\texttt{addArrayIntoArray(...)} in Listing
{}\ref{listing:addArrayIntoArray-raw} to use {}\texttt{std::vector}
which is shown in Listing
{}\ref{listing:addArrayIntoArray-std-vector}.

\begin{listing}:\\
\label{listing:addArrayIntoArray-std-vector}
{\small\begin{verbatim}
  template<class T>
  void addArrayIntoArray( const std::vector<T> &a, std::vector<T> &b )
  {
    DEBUG_MODE_ASSERT_EQUALITY( a.size(), b.size() );
    for (int i = 0; i < a.size(); ++i)
      b[i] += a[i];
  }
\end{verbatim}}
\end{listing}

The advantages of the function in Listing
{}\ref{listing:addArrayIntoArray-std-vector} are that a) the size of
each array is kept with the pointer to the array itself inside of each
{}\texttt{std::vector} object, b) The sizes of the arrays can be
asserted by the implementation of the function
{}\texttt{addArrayIntoArray(...)}, c) it is easy for callers who
already use single {}\texttt{std::vector} objects.

While this use of {}\texttt{std::vector} replaces raw C++ pointers as
basic array function arguments, it has several serious problems in
both usability and performance in some important use cases.  The
primary disadvantages of using {}\texttt{std::vector} as general array
arguments to functions is a) there is no flexibility in how the arrays
are allocated, and b) one cannot pass sub-views of larger arrays of
data.

To illustrate the problems with using {}\texttt{std::vector} for all
array arguments to functions, consider a situation where the
application wants to allocate big arrays of data and then operate on
pieces of the array based on different logic.  One motivation for
allocating big arrays of data is to avoid memory fragmentation and
improve data locality.  Now consider in Listing
{}\ref{listing:someBlockAlgo-std-vector} what the client code would
have to look like when using the form of
{}\texttt{addArrayIntoArray(...)} in Listing
{}\ref{listing:addArrayIntoArray-std-vector} which takes in
{}\texttt{std::vector} objects.

\begin{listing}:\\
\label{listing:someBlockAlgo-std-vector}
{\small\begin{verbatim}
  void someBlockAlgo( const int numBlocks, const std::vector<double> &big_a,
    std::vector<double> &big_b )
  {
    DEBUG_MODE_ASSERT_EQUALITY( big_a.size(), big_b.size() );
    const int totalLen = big_a.size();
    const int blockSize = totalLen/numBlocks; // Assume no remainder!
    
    const int blockOffset = 0;
    for (int block_k = 0; block_k < numBlocks;
      ++block_k, blockOffset += blockSize)
    {
      if (big_a[blockOffset] > 0.0) {
        // Create temporary std::vectors to do function call
        std::vector a(big_a.begin()+blockOffset,
          big_a.begin()+blockOffset+blockSize);
        std::vector b(big_a.begin()+blockOffset,
          big_b.begin()+blockOffset+blockSize);
        // Do the operation
        addArrayIntoArray(a, b);
        // Copy back into the output array
        std::copy(b.begin(), b.end(), big_b.begin() + blockOffset);
      }
    }
  }
\end{verbatim}}
\end{listing}

As you can see, the client code that uses the {}\texttt{std::vector}
version of {}\texttt{addArrayIntoArray(...)} is niether clean, nor
efficient as temporary copies of all of the data have to be created
just to make the function call and then data has be be copied back
into the full array.

Now consider the client code in Listing
{}\ref{listing:someBlockAlgo-std-vector-raw-ptr} which uses the raw C++
pointer version of {}\texttt{addArrayIntoArray(...)} in Listing
{}\ref{listing:addArrayIntoArray-raw}.

\begin{listing}:\\
\label{listing:someBlockAlgo-std-vector-raw-ptr}
{\small\begin{verbatim}
  void someBlockAlgo( const int numBlocks, const std::vector<double> &big_a,
    std::vector<double> &big_b )
  {
    DEBUG_MODE_ASSERT_EQUALITY( big_a.size(), big_b.size() );
    const int totalLen = big_a.size();
    const int blockSize = totalLen/numBlocks; // Assume no remainder!
    
    const int blockOffset = 0;
    for (int block_k = 0; block_k < numBlocks;
      ++block_k, blockOffset += blockSize)
    {
      if (big_a[blockOffset] > 0.0) {
        addArrayIntoArray( blockSize, &big_a[blockOffset], &big_b[blockOffset]);
      }
    }
  }
\end{verbatim}}
\end{listing}

As you can clearly see, using the raw C++ pointer version of
{}\texttt{addArrayIntoArray(...)} makes the client code much cleaning
and much more efficient.  However, of course, if the client makes any
mistakes with its arrays of memory, then the resulting program will
(in the base case) segfault, or will silently produce the wrong
result, or (in the worst case) actually produce the right result on
the current platform but will fail on other platforms.

The Teuchos array utility classes make algorithms involving subviews
like shown above very clean, very efficient, and very safe (see the
same versions of this example code using these new Teuchos classes in
Section ???).

In summary, {}\texttt{std::vector} is {}\underline{not} an efficient
or convenient general-purpose replacement for raw C++ pointers as
function arguments in many important use cases.


%
{}\subsection{Problems with relying on standard memory checking
utilities}
%

Some programmers simply use raw C++ pointers and think that standard
memory checking tools like
Valgrind\footnote{{}\texttt{http://valgrind.org}} and
Purify\footnote{\texttt{http://www.ibm.com/software/awdtools/purify}}
will catch all of their mistakes.  When I first started coding in C++
back in 1996, I was very aware of the problems with using raw pointers
in C++ after experiencing the segfaults and memory leaks that all C++
programmers experience.  At the time, I had experimented some with
writing my own utility classes that encapsulated raw C++ pointers and
I considered taking that further.  However, at that time, I
conjectured that going through the effort of encapsulating all raw C++
pointers might be a waste of time because it would not be long until
someone came up with an 100\% bullet-proof memory checking tool for
C++ that would make my feable programmer-controlled attempts to wrap
raw pointers obsolute.  After more than 10+ years of C++ programming
experience where I have written perhaps a million lines of C++ code on
a number of different platforms/compilers, I have come to regret that
decision.

Through painful experience and then through some more careful thought,
I have come to realize that memory checking tools like Valgrind and
Purify will never be able to provide an even sufficient (forget about
100\%) means to validate the memory usage of our C++ programs.  With
respect to existing tool implementations, I have experienced cases
where both Valgrind and Purify have reported not even a single warning
before the program segfaulted with essentially no feedback at all.  I
will not go into detail about what techniques memory tools like
Valgrind and Purify use to verify memory usage other than to say that
they can do a lot by just taking control of {}\texttt{malloc(...)} and
{}\texttt{free(...)} and in inserting checks into the execution of the
program by controlling the manipulation of the program stack.

One such case where the Valgrind and Purify were completely unhelpful
occured with an off-by-one error with an {}\texttt{std::vector} using
Linux/gcc.  In the end, the way that I found the off-by-one error was
by just staring at the code over and over again until I happened to
see the problem.  However, what I discovered through two days of
debugging was that {}\texttt{std::vector} used its own allocator which
allocated big chunks of memory through {}\texttt{malloc(...)}.  It
then proceeded to do its own memory allocation scheme, which was very
fast but was invisible to the watchfull eyes of Valgrind and Purify.
Any reads to this block of memory looked fine to Valgrind and Purify
because it was all contained within the block returned from
{}\texttt{malloc(...)}.  What the off-by-one error did was to write
over a library managed part of the memory block and that silent
corruption would doom a later attempt by {}\texttt{std::vector} to
allocate memory.

There are other categories of use cases where external memory checking
tools like Valgrind and Purify will never be able to verify correct
memory usage.  One example is semantic off-by-one errors commited in
larger blocks of data.  To demonstrate this type of error, consider
the example code in the function {}\texttt{someBlockAlgo(...)} in
Listing {}\ref{listing:someBlockAlgo-std-vector-raw-ptr} which uses
the raw C++ pointer version of the function
{}\texttt{addArrayIntoArray(...)} in Listing
{}\ref{listing:addArrayIntoArray-raw}.  Now consider what happens when
a developer introduces an off-by-one error as shown in
{}\texttt{addArrayIntoArray(...)} in Listing
{}\ref{listing:addArrayIntoArray_rawError}.

\begin{listing}:\\
\label{listing:addArrayIntoArray_rawError}
{\small\begin{verbatim}
  template<class T>
  void addArrayIntoArray( const int n, const T a[], T b[] )
  {
    for (int i = 0; i <= n; ++i)
      b[i] += a[i];
  }
\end{verbatim}}
\end{listing}

In case you missed it, the off-by-one error shown in Listing
{}\ref{listing:addArrayIntoArray_rawError} is the replacement of the
loop termination statement {}\texttt{i < n} with {}\texttt{i <= n}
which is a very common C++ programming error.

Now let's consider the implications that the off-by-one error shown in
Listing {}\ref{listing:addArrayIntoArray_rawError} will have of the
data in {}\texttt{big\_b} as driven by the code in Listing
{}\ref{listing:someBlockAlgo-std-vector-raw-ptr}.  If the last block
{}\texttt{block\_k=numBlocks-1} of data is processed, then there is a
reasonable chance a memory checking tool like Valgrind would catch the
off-by-one error being committed at the very end of the array
{}\texttt{big\_b}.  However, as described above, Valgrind may not
catch even this error.  Also, note that turning on bounds checking
with {}\texttt{std::vector} (i.e.\ by enabling
{}\texttt{\_GLIB\_CXX\_DEBUG} with gcc) will not catch this error
either because of the way the raw pointers are extracted in and and
passed in the function call:

{\small\begin{verbatim}
  addArrayIntoArray( blockSize, &big_a[blockOffset], &big_b[blockOffset]);
\end{verbatim}}

Now consider a defect caused by this off-by-one error for which no
automated memory checking tool that will ever be devised will ever be
able to catch.  This type of defect will occur, for example, when for
the last block {}\texttt{block\_k=numBlocks-1} we have
{}\texttt{big\_a[(numBlocks-2)*blockSize] > 0.0} and
{}\texttt{big\_a[(numBlocks-1)*blockSize] <= 0.0}.  In this case, only
the next-to-last block of data will be processed by the defective
{}\texttt{addArrayIntoArray(...)} function.  This will not result in a
classic off-by-one error that a memory checking tool would catch
because it would not touch memory outside of what is stored in
{}\texttt{big\_b}.  However, this off-by-one error commited in Listing
{}\ref{listing:addArrayIntoArray_rawError} would result in the array
entry {}\texttt{big\_b[(numBlocks-2)*blockSize+blockSize]} being
erroneously modified.  This is a defect that might only slightly
damage the final result of the program for the typical use case and
might therefore go unnoticed for years.  However, when the program was
really being used for something important years later for a
non-typical use case, this small off-by-one error could result in
reporting incorrect results which could embarrass a company, result in
a lawsuit, or (in an exteme case) even cause someone's death.

The point that I am trying to make in the above example is that
automated memory checking tools like Valgrind and Purify will never be
able to check the {}\textit{semantic} correctness of our usage of<
memory.  The off-by-one defect shown above is 100\% correct from a
strict memory usage point of view (i.e.\ only allocated memory can be
written to and only allocated and initialized memory can be read from)
but is 100\% wrong from a semantic point of view (i.e.\ the function
{}\texttt{addArrayIntoArray(...)} can only operate on the elements of
data from {}\texttt{0} to {}\texttt{n-1}).  The utility classes in
Teuchos described below help to verify that memory is used in a
semantically correct way.


%
{}\section{Philosophy of memory management: Safety, speed, generality
and the 100\% guarantee}
%

When looking at different strategies for memory management in C++ and
in other languages, I believe it helps to think a little on the
philosophical level which can actually help ground us.

When looking at the different memory management appraoches implemented
in varous programming languages, the core issues come down to tradoffs
in saftey and correctness versus speed and generality.  For example, a
language like C sacrifices safety and correctness for speed and
generality.  Because C is so ``close'' to the hardware, you can
implement very specialized memory management appraoched tailored to
very specific types of domains.  However, the price one pays for this
raw speed and flexibility in C is the fact that there is very little
compiler-supported checking needed to assert correct memory usage.

Now take Python on the other extreme.  If you write code only in
Python, you will almost never experience and memory leak or segfault
of any kind due to code that you directly write.  Here we have a
language which is nearly 100\% safe (assuming the language
implementation is 100\% correct) but offers less flexibility in how
memory is managed and results in very slow native code as compared to
C in many cases (e.g.\ for computationally intensive loops).

So how important is a 100\% guarantee that memory will always be used
correctly like is provided in a language like Python?  How important
is a 100\% guarantee in any area?  Well, if you can get a 100\%
guarantee without having to pay a significant price for it then you
would be a fool not to accept it.  For example, if you have a choice
between two venders selling the same product for the same price but
one vender will give you a 100\% money-back guarantee, with all things
being equal, it would probably be foolish not to go with the vender
with the 100\% guarantee.

However, in most areas, greater safety (not to mention a 100\%
guarantee) comes with greater costs.  Instead of demanding a 100\%
guarantee, we typically accept some level of extra risk as long as we
have taken basic precautions to protect ourselves.  To demonstrate
this, let's consider another analogy which I like to refer to as the
{}\textit{Transportation Analogy}.  When considering modes of
transporation, we accept that we are not 100\% safe when driving our
cars on the road but we do it anyway.  The reason that we get into our
cars every day is that we take reasonable precautions like purchasing
a car with a good safely design, wearing seat-belts, obeying the
traffic laws, driving a reasonable speed, and practicing defensive
driving.  What I am going to argue is that the approach to memory
management that I am advocating in this paper is the equivalent of
driving a car, wearing your seat belt, and taking other reasonable
safety precautions.

Now let's talk about the safety versus speed/efficiently extremes in
the Transportation Analogy and in the areas of memory mangement.  At
one extreme, writing all high-level code in C++ (or C) using raw
pointers for everything is like riding a high-performance motorcycle
on a crowded interstate going 150 mph, without wearing a helment or
any other safety gear, while doing a wheelie.  At this extreme, one
wrong move and you are dead.

At the other exteme, writing all code in a language like Python is
like driving around in a reinforeced tank that does a maximum of 10
mph where you sit inside wearing a car racing helment with the Hans
device, full racing safety gear, and have a massive air bag system to
encase your entire body in foam three feet thick in case of a
collision.  On this side of extreme safely, we could hit a Mac truck
head on and be just fine.  The only way to really kill ourselves is if
we were to dirve off of a shear cliff.

If we all required a near 100\% safely guarantee, we would all be
driving around in reinforced tanks like the one described above but we
don't.  We don't because we are not willing to pay the price of the
near 100\% guarantee provided by the tank.  We can't afford it
financially and it would take forever to get back and forth to
work. Instead, we are content with our less than 100\% safe cars
because they are affordable and fast and do not pose unreasonable
risks.

Now, we can incrementally go from either extreme to the more balanced
position in both the Transportation Analogy and with memory management
in C++ and Python.

From the extreme of safety with less speed and flexibility represented
by the reinforced tank (and Python) you can incrementally move toward
the middle ground of the car. You can start by removing the racing
helment and Hans device, followed by decreasing the weight and
incresing the speed of the tank, and so on.  Continuing on this trend
of sacrificing safety in favor of greater speed and agility leads us
to our typical car.  Likewise, moving from an extreme of safety to a
more reasonable balance between safety and speed/flexibility in Python
involves taking pieces of computationally intensive Python code,
rewriting them in C/C++, and then calling them from Python.  This is
an approach that many Python enthusiasts are advocating [???] but make
no mistake that in going down this road that you are sacrificing
safety in Python in the name of speed and flexibility.  You are giving
up Python's nearly 100\% guarantee when you do this.

From the extreme of speed and flexibilty with little reguard for
safety represnted by the motorcycle (and C/C++ raw pointers) you can
also incrementally move toward the car.  You can start by putting on a
helment, followed by slowing down some, and so on.  Continuing on this
trend of adding safety will eventually see you morphing the motorcycle
into the typcial car.  Likewise, moving from an extreme of less safety
toward a more reasonable balance between speed/flexibiltiy and safety
in C++ involves adding more and more utility classes to hide more and
more uses of raw C++ pointers in our high-level C++ code.  This is the
trend that the C++ community has been following for more than the last
decade.  We see it first in the introduction of
{}\texttt{std::auto\_ptr} and {}\texttt{std::vector}.  This was then
followed by the development of {}\texttt{boost::shared\_ptr} (and
threfore {}\texttt{std::shared\_ptr} in C++09) and
{}\texttt{boost::array\_ptr}.  What I am suggesting in this paper is
the logical conclusion of this journey which is the development of a
complete set of utility classes in order to remove all raw C++
pointers from our high-level C++ code; i.e.\ complete the transition
from the motorcycle (C/C++ raw pointers) to the car (Teuchos C++
memory management classes).

With the approach being advocated in this paper for the Teuchos Memory
Management classes, using the debug mode is like driving around in the
tank where you are protected from almost any danger.  However, using
the optimized mode is like driving around with the high-performance
motorcycle.  Try that with a car!


%
{}\section{Important prerequisites}
%

Before finally discussing the Teuchos memory management classes, I
need to discuss a set of prerequisite concepts that are needed in
order to understand the wholistic memory management approach described
here.


%
{}\subsection{Value types versus reference types}
\label{sec:value-and-reference-types}
%

Because of the flexibility of C++, many C++ programmers can and do
implement a wide varity of types yielding objects with different types
of usage symentics.  A quick summary of ``accepted'' class types in
C++ is given in Item 33 ``Be clear what kink of clas you're writing''
in {}\cite{C++CodingStandards05}.  I see little point here in trying
to classify all of the crazy ways that I have seen people code objects
in C++ that stray from these ``accepted'' class types.  Instead, I
want to suggest that we should classify 99\% of our classes as either
{}\textit{Value Types} or {}\textit{Reference Types}.  Value Types and
Reference Types are said to use {}\textit{Value Semantics} and
{}\textit{Reference Semantics}, respectively, and that is sometimes
how these datatypes are described in various C++ literature.

{}\textit{Value Types} in general:

\begin{itemize}

{}\item have public destructors, default constructors, copy
constructors, and assignemnt operators (implementing deap copy,
semantics)

{}\item have an identity that is determined by their value, not their
address,

{}\item are usually allocated on the stack or as direct data members
in other class objects,

{}\item are usually {}\underline{not} allocated on the heap, and

{}\item do not have any virtual functions and are not to be used as
base classes (see Item 35 in {}\cite{C++CodingStandards05}).

\end{itemize}

If {}\texttt{S} denotes a typical Value Type, the class defintion of
{}\texttt{S} includes:

{\small\begin{verbatim}
  class S {
  pubic:
    ~S();
    S();
    S(const S&);
    S& operator=(const S&);
    ...
  };
\end{verbatim}}

All of the built-in intrisic C++ data-types like {}\texttt{char},
{}\texttt{int}, and {}\texttt{double} are Value Types.  Likewise,
derived class types like {}\texttt{std::complex} and
{}\texttt{std::vector} are also Value Types.  Value Types have also be
called by other names in the C++ literature. Stroustroup refers to
Value Types as ``true local variables'' in {}\cite{stroustrup94}.  The
term Abstract Data Type (ADT) in older C++ literature such as
[???Advanced C++???] usually maps to the concept of a Value type, but
usaully carries greater significance in implying that operator
overloading is used to make an ADT look more like a built-in C++ type
(such as is the case for {}\texttt{std::complex}).

Alternatively, {}\textit{Reference Types} in general:

\begin{itemize}

{}\item do not have a public copy constructor or assignement operator,

{}\item are manipulated through a (smart) pointer or reference,

{}\item have an identity that is determined by their address and not
their value,

{}\item are allocated on the heap,

{}\item typically are not permitted to be or cannot be allocated on
the stack,

{}\item are copied through an abstract clone function (if copying is
allowed at all),

{}\item have one or more virtual functions, and

{}\item are usually designed to be used as base classes.

\end{itemize}

Referene Types (i.e.\ Reference Semantics) are typically used for base
classes in C++.  Examples of base classes in the C++ standard library
include {}\texttt{std::ios\_base} and
{}\texttt{std::basic\_streambuf}.  Reference Types in the form of
abstract base classes form the foundation for object-oriented
programming in C++.

If {}\texttt{A} denotes a typical Reference Type, the class defintion
of {}\texttt{A} generally includes:

{\small\begin{verbatim}
  class A {
  pubic:
    virtual ~A();
    virtual A* clone() const = 0;  // NOTE: Should use RCP (see later)
    virtual void someFunc() = 0;
    ...
  protected: // or private
    A(const A&);
    A& operator=(const A&);
    ...
  };
\end{verbatim}}

Note that you can almost always choose to manipulate a Value Type
using Reference Semantics.  For example, it is very common to choose
to dynamically allocate large value objects like
{}\texttt{std::vector} and then pass around (smart) pointers and
references to the object to avoid unnecessary and expensive copying
and to facilitate sharing state.

While the ideas of Value Types and Reference Types and Value Semantics
and Reference Semantics are long estabilished in the C++ literature
(even if the terminology is not very uniform), many C++ programmers
either seem to not know about these idoms or choose not to follow them
for some reason.  By forcing the majority of our classes into either
using {}\textit{Value Semantics} or {}\textit{Reference Semantics} we
eliminate meaningless variability in our C++ programs and we free
ourselves to think about more important things.

{}\textbf{Side Note:} The somewhat rigid classification of C++ types
into Value Types and Reference Types is similar in motivation and in
many other respects to Eric Evans' differentiation of all domain types
into {}\textit{Value Objects} and {}\textit{Entities} in Domain Driven
Design (DDD) {}\cite{DomainDrivenDesign}.  While there are
similarities between DDD's Value Objects and Entities and C++'s Value
Types and Reference Types, respectively, there is not a one-to-one
mapping.  In DDD, the distinction between Value Object and an Entity
has more to do with the nature of the object in relation to the domain
model and is not related to how memory is manged.  Evans assumes that
you are using a language like Java where all objects use reference
semantics.


%
{}\subsection{Persisting associations versus non-persisting associations}
\label{sec:persisting-nonpersisting-associations}
%

Another important prerequisite for understanding the Teuchos memory
management classes is the distinction between {}\textit{non-persisting
associations} and {}\textit{persisting associations}.  Working
definitions for these are:

\begin{itemize}

{}\item\textit{Non-Persisting associations} are associations between
two or more objects that exist only within a single function call for
formal function arguments, or a single statement for function return
objects, where no memory of any of the objects is retained as a side
effect after the function returns or the statement ends.

{}\item\textit{Persisting associations} are associations that exist
between two or more objects that extend past a single function call
for formal function arguments, or a single statement for function
return objects.

\end{itemize}

To help define these two different types of associations, consider the
class and function defintions in Listing
{}\ref{listing:NonPersistingPersistingAssociationsRawPointers}.

\begin{listing}:\\
\label{listing:NonPersistingPersistingAssociationsRawPointers}
{\small\begin{verbatim}
  class A {
  public:
    void fooA() const;
  };

  class B {
  public:
    void fooB1(const A &a) { a.fooA(); }
    void fooB2() const { ... }
  };

  class C {
  public:
    C() : b_(0) {}
    void fooC1(B &b, const A &a)
      { b_ = &b; b_->fooB1(A); }
    void fooC2() const
      { b_->fooB2(); }
  private:
    B* b_;    
  };

  void someFunc(C &c, B &b, const A &a)
  {
    c.fooC1(b, a);
    c.fooC2();
  }
\end{verbatim}}
\end{listing}

The function {}\texttt{B::fooB1(...)} in Listing
{}\ref{listing:NonPersistingPersistingAssociationsRawPointers}
involves a non-perissting association with respect to the {}\texttt{A}
and {}\texttt{B} objects since no memory of the object {}\texttt{a}
remains after the function {}\texttt{B::fooB1(...)} exists.
Non-persisting associations represent typical input/output-only
arguments to a function.

The function {}\texttt{C::fooC1(...)} in Listing
{}\ref{listing:NonPersistingPersistingAssociationsRawPointers} creates
a perissting association between a {}\texttt{C} object and a
{}\texttt{B} object since the memory of the {}\texttt{B} object is
retained in the {}\texttt{C} object that persists after the function
{}\texttt{C::fooC1(...)} exits.  This memory of the {}\texttt{B}
object stored in the {}\texttt{C::b\_} pointer data member is then
used to implement the function {}\texttt{C::fooC2()}.  Note that the
function {}\texttt{C::fooC1(...)} also involves a non-persisting
association with the {}\texttt{A} object {}\texttt{a} since it is only
used to call {}\texttt{B::fooB1(...)} and no memory of {}\texttt{a}
lingers after {}\texttt{C::fooC1(...)} exists.

As a final example, consider the nonmember function
{}\texttt{someFunc(...)} in Listing
{}\ref{listing:NonPersistingPersistingAssociationsRawPointers}.  While
{}\texttt{someFunc(...)} is a free function, it actually involves the
creation of a presisting asscoation between the {}\texttt{C} and
{}\texttt{B} objects as a side effect.

In the idioms advocated in this paper, smart reference counted
pointers are used for all persisting associations and never for
non-persisting associations.  Using the basic Teuchos {}\texttt{RCP}
class, the raw pointer code in Listing
{}\ref{listing:NonPersistingPersistingAssociationsRawPointers} would
be refactored into the code in Listing
{}\ref{listing:NonPersistingPersistingAssociationsRCP}.

\begin{listing}:\\
\label{listing:NonPersistingPersistingAssociationsRCP}
{\small\begin{verbatim}
  class A {
  public:
    void fooA() const;
  };

  class B {
  public:
    void fooB1(const A &a) { a.fooA(); }
    void fooB2() const { ... }
  };

  class C {
  public:
    C() : b_(0) {}
    void fooC1(const RCP<B> &b, const A &a)
      { b_ = &b; b_->fooB1(A); }
    void fooC2() const
      { b_->fooB2(); }
  private:
    RCP<B> b_;
  };

  void someFunc(C &c, const RCP<B> &b, const A &a)
  {
    c.fooC1(b, a);
    c.fooC2();
  }
\end{verbatim}}
\end{listing}

Note that the classes {}\texttt{A} and {}\texttt{B} remain unchanged
because they do not involve any persisting associations.

Another situation where the concepts of persisting and non-persisting
associations comes up relates to how objects are returned by C++
functions as return values.  A presisting relationship is made through
a function return object if that object is remembered past a single
statement.  For example, cnsider the following code fragment:

\begin{listing}:\\
{\small\begin{verbatim}
  std::vector<int> v(n);
  ...
  int &ele = v[0];  // Creates a persisting return object relationship
  ...
  ele = 5;          // Changes v[0] much later!
\end{verbatim}}
\end{listing}

The above code fragment shows a presenting relationship between the
client code that is initializing the local reference {}\texttt{ele}
and the {}\texttt{std::vector} container object {}\texttt{v}.  This is
very fragile and dangerous code because if {}\texttt{v} is resized,
grown or have some other type of change, then the reference pointed to
by {}\texttt{ele} can be invalid.  For example, the following code
fragement will likely result in a runtime memory usage error and (if
we are lucky) segfault:

\begin{listing}:\\
{\small\begin{verbatim}
  std::vector<int> v(n);
  int &ele = v[0];
  v.resize(10*n);
  ele = 5;  // ele is likely to be invalid here!
\end{verbatim}}
\end{listing}

If we are unlucky, the statement {}\texttt{ele = 5} will work just
fine on one platform with one implementation of the
{}\texttt{std::vector} but will break on another platform when run
with a different dataset.  Note that tools like Valgrind and Purify
may not flag the problem due to how many implementations of
{}\texttt{std::vector} deal with memory.

Basically the problem here is that the
{}\texttt{std::vector::operator=(size\_type)} function returns a raw
C++ reference that should never be remembered past a single statement.
The safe way to change an element is:

\begin{listing}:\\
{\small\begin{verbatim}
  std::vector<int> v(n);
  v[0] = 5; // Non-persisting relationship!
  v.resize(10*n);
\end{verbatim}}
\end{listing}

Here, we will say that the
{}\texttt{std::vector::operator=(size\_type)} should only be used for
non-persisting relationships as shown above.

For a more detailed treatment of persisting and non-persisting
associations, and on the basics of {}\texttt{RCP}, see the discussion
in {}\cite{RefCountPtrBeginnersGuide}.


%
{}\section{Teuchos classes for safer memory management and usage}
%

The primary purpose for the Teuchos memory management classes is to
allow the replacement of all raw C++ pointers in all application-level
code.  These classes are efficient and general and, in a debug build
of the code, will catch and gracefully report 99\% or more of the
programming errors typically made with the ubiquitous high-level use
of raw C++ pointers.


%
{}\subsection{Outline of the basic approach}
{}\label{sec:outline_of_basic_approach}
%

The basic approach being advocated here and implemented in the Teuchos
memory management classes is to:

\begin{itemize}

{}\item Encapsulate all raw C++ pointers in high-level code in
specially designed memory management classes, caputre raw C++ pointers
as soon as possible, and avoid exposing raw calls to {}\texttt{new} in
application-level code.

{}\item Provide a complete set of cooperating types that work together
to safely and conveniently implement all hand-offs of raw C++
pointers using supported and carefully scrutinized conversion code.
Also, never define implicit conversions from these safe types to raw
C++ pointers or the entire type safe system falls apart.

{}\item Differentiate memory management classes for handling single
objects and for handling contiguous arrays of objects.

{}\item Differentiate memory management classes according to
persisting and non-perissting associations.

  \begin{itemize}

  {}\item Use reference counting for memory management classes
  designed to handle persisting associations.

  {}\item Do not impose the overhead of reference counting for memory
  management classes designed to handle non-persisting associations.

  \end{itemize}

{}\item Provide encpauslations for all uses of raw C++ pointers for
arrays of memory including dynamically sized arrays, statically sized
arrays, and stack-based arrays.

{}\item Provide a default {}\textit{optimized mode} where maximum
performance and minimal overhead are the goals where raw C++ pointer
performance is achieved for all reasonable use cases.

{}\item Provide an optional {}\textit{debug mode} whose goal is to
provide maximum runtime checking with low overhead by default to catch
all sorts of common errors like:.

  \begin{itemize}
  {}\item Dereferencing null pointers
  {}\item Dereferencing dangling pointers (references)
  {}\item Array access errors like off-by-one and other errors
  {}\item Incorrect iterator usage
  {}\item Circular dependencies
  \end{itemize}

\end{itemize}


\begin{table}
\begin{center}
\begin{tabular}{|l|c|c|}
\hline
& Non-persisting
& Persisting \\
& Associations
& Associations \\
\hline
single objects
& {}\texttt{Ptr<T>}
& {}\texttt{RCP<T>} \\
\hline
contiguous arrays
& {}\texttt{ArrayView<T>}
& {}\texttt{ArrayRCP<T>} \\
\hline
\end{tabular}
\end{center}
\caption{\label{table:BasicSingleArrayTypes}
Basic Teuchos memory management utility classes for encapsulating raw
pointers for single objects and contiguous arrays of objects.}
\end{table}


\begin{table}
\begin{center}
\begin{tabular}{|l|c|c|c|c|}
%
\hline
%
Operation
& \texttt{Ptr<T>}
& \texttt{RCP<T>}
& \texttt{ArrayView<T>}
& \texttt{ArrayRCP<T>} \\
%
\hline
%
\multicolumn{5}{l}{\textit{Raw pointer-like functionality}} \\
%
\hline
%
Implicit conv derived to base
& x
& x
& 
&  \\
%
\hline
%
Implicit conv non-const to const
& x
& x
& x
& x \\
%
\hline
%
Dereference \texttt{operator*()}
& x
& x
& 
& x \\
%
\hline
%
Member access \texttt{operator->()}
& x
& x
& 
& x \\
%
\hline
%
\texttt{operator[](s)}
& 
& 
& x
& x \\
%
\hline
%
operators \texttt{++}, \texttt{--}, \texttt{+=(s)}, \texttt{-=(s)}
& 
& 
& 
& x \\
%
\hline
%
\multicolumn{5}{l}{\textit{Other functionality}} \\
%
\hline
%
Reference counting machinery
& 
& x
& 
& x \\
%
\hline
%
Iterators: begin(), end()
& 
& 
& x
& x \\
%
\hline
%
\texttt{ArrayView} subviews
& -
& -
& x
& x \\
%
\hline
%
\end{tabular}
\end{center}
\caption{\label{table:TypesSummary}
Summary of capabilities of the basic Teuchos memory management
classes.}
\end{table}

The basic templated Teuchos memory management classes for encapsulating
raw C++ pointers for single objects and arrays shown in Table
{}\ref{table:BasicSingleArrayTypes} are {}\texttt{Ptr},
{}\texttt{RCP}, {}\texttt{ArrayView}, and {}\texttt{ArrayRCP}.  A
summary of the capabilities of these classes is shown in Table
{}\ref{table:TypesSummary}.  What you can see from this table is that
raw pointer-like functionality is partitioned across these various
sets of clasess in logical and safe ways.  For example, array-related
operations are not defined on the single-object classes
{}\texttt{Ptr}, {}\texttt{RCP} but implicit conversion from derived
types to base types is allowed.  Alternatively, the array classes
{}\texttt{ArrayView}, and {}\texttt{ArrayRCP} do {}\underline{not}
support the dangerous and ill-conceived ability to implicitly convert
arrays of derived types to arrays of base types that is decried in
Section {}\ref{sec:problem-with-raw-array-pointers}.  Note that the
class {}\texttt{ArrayView} does not support all of the raw pointer
iterator-like operations that {}\texttt{ArrayRCP} does like the
dereference operator {}\texttt{operator*()} or the pointer offset
functions that change the pointer while {}\texttt{ArrayRCP} does.  The
reason that {}\texttt{ArrayRCP} does support these iterator-like
operations is so that it can be used as a general purpose iterator
implementation while {}\texttt{ArrayView} does not.  Note that all of
the array classes defined in Teuchos all support a basic iterator
interface with the typedefs {}\texttt{iterator} and the functions
{}\texttt{begin()} and {}\texttt{end()}.  In optimized mode, these
iterators are simply raw C++ pointers yielding maximum performance.
However, in debug mode, the iterators are implemented as
{}\texttt{ArrayRCP} objects resulting in fully checked iterators.

\begin{table}
\begin{center}
\begin{tabular}{|l|l|}
\hline
Array class
& Specific use case \\
\hline
{}\texttt{Array<T>}
&  Contiguous dynamically sizable, expandable, and contractible arrays \\
{}\texttt{Tuple<T,N>}
&  Contiguous statically sized (with size {}\texttt{N}) arrays \\
\hline
\end{tabular}
\end{center}
\caption{\label{table:ExtendedArrayTypes}
Teuchos array container classes.}
\end{table}

In addition to the four basic memory management classes shown in Table
{}\ref{table:BasicSingleArrayTypes} which provide the most fundamental
replacements for all raw C++ pointers in all high-level code, Teuchos
also contains a few other array container classes for a few more
specific use cases shown in Table {}\ref{table:ExtendedArrayTypes}.
The three array container classes {}\texttt{Array}, and
{}\texttt{Tuple} pretty much cover the majority of use cases in C++
where raw C++ pointer arrays are used for containers.  The class
{}\texttt{Array} is a directly replacement for {}\texttt{std::vector}
and actually wraps it internally.

These classes all work together to provide a high level of debug-mode
runtime checking to catch the majority of common programming errors
and report these errors gracefully with informative error messages.  A
debug-mode build of the code is facilitated by defining the
preprocessor macro {}\texttt{TEUCHOS\_DEBUG}.  When
{}\texttt{TEUCHOS\_DEBUG} is not defined, the Teuchos memory
management classes are configured to impart minimal overhead and yeild
fast performance.  When {}\texttt{TEUCHOS\_DEBUG} is defined, these
classes are configured to perform maximal debug runtime checking.
These classes are also carefully designed so that if a program is
implemented correctly using these classes and executes correctly in
the optimized mode, the program compiled with the debug checking
turned on will execute in exactly the same way.  However, if any
undefined, dangerous, or just plain wrong behavoir is being used, then
these memory management classes will throw exceptions and the
exception objects will have very good error messages embedded in them.

What is important to understand is that all of these memory management
classes must be developed together with knowledge of each other's
internal implementations in order to provide for solid debug-mode
runtime checking.  For example, in general, you cannot mix in other
memory management classes like {}\texttt{boost::shared\_ptr} (i.e.\
{}\texttt{std::shared\_ptr} in C++09) and {}\texttt{std::vector} and
provide for the level of runtime checking that is supported by the
concert of Teuchos memory management classes described here.  More
details about why this is so are given in Section
{}\ref{sec:debug-mode-runtime-checking} in the context of debug-mode
runtime checking for and reporting of dangling references.

As with the development of any set of C++ classes, a set of
accompanying idioms must also be developed for maximizing their
effective use.  The idioms described in this paper involving the
Teuchos memory management classes result in code with maximum
compile-time checking, maximum debug-mode runtime checking, and which
is highly self-documenting.


%
{}\subsection{The proper role of raw C++ pointers}
\label{sec:role-of-raw-pointers}
%

The thesis of this paper is that we should eliminate the use of all
raw C++ pointers in all of our high-level C++ code and instead use a
system of specially designed types for the right situations.  Does
that mean that we should never use raw C++ pointers in any C++ code?
The answer of course is {}\underline{no}, since we will always have to
use raw C++ pointers in special situations.

Here are the valid situations where it is appropriate (or required)
to use raw C++ pointers in our C++ programs:

\begin{itemize}

{}\item{}\textit{Use raw C++ pointers for extremely well-encapsulated,
low-level, high-performance algorithms}

In order to achieve high performance in computationally intensive
code, we will always have to use raw C++ pointers.  This includes
using raw C++ pointers to build complex specialized data structures
and similar purposes.  In this context, we can think of raw C++
pointers as a fairly compact and efficient way to communicate with the
compiler about how we want to manage memory at the hardware level.
The ability to do this type if fine-grained manipulation of memory
has always been one of the strengths of C and C++ in systems-level
programming.  Therefore, we can think of using raw pointers in C++ as
a kind of portable assembly language that we always have at our
disposal on every platform and with every compiler.  However, note
that our use of raw C++ pointers in the name of performance must be
very well encapsulated and we must thoroughly review (by our best
programmers) and exhaustively test this code.  Note that we can always
use a series of typedefs to substitute our safe memory management
classes like {}\texttt{ArrayRCP} in place of raw C++ pointers when we
need to debug our code.  In fact, such an approach would be
recommended practice.

{}\item{}\textit{Use raw C++ pointers to communicate with legacy C++
code and with other languages through C bindings}

We will always have to use raw C++ pointers to reuse and communicate
with legacy C++ code and to call functions in other languages through
the now-universal approach of using C bindings.  However, we must
endeavor to minimize the amount of our C++ code that has naked raw C++
pointers and we should only expose a raw C++ pointer at the last
possible moment (such as in the call to the external functions
themselves).  Again, we must carefully encapsulate our access to this
non-complient code that requires us to expose raw C++ pointers.

\end{itemize}

One point is worth nothing here which is that in our new modern C++
software we must never use raw C++ pointers in the basic interfaces
between our various modules as that is where a majority of mistakes in
the use of memory will be made.  This goes somewhat contrary to the
advice in Item 63 ``Use sufficiently portable types in a module's
interface'' in {}\cite{C++CodingStandards05}.  If our new modern safe
C++ software must be called by non-compliant software that uses raw
C++ pointers, then we can provide specialized C-like interfaces for
those clients that use raw C++ pointers for communication.  Of course,
once we do this, we will have to rely on our clients to pass in memory
correctly and keep it valid as long as our modules need it.


%
{}\subsection{Memory management classes replacing raw pointers for
single objects}
%

Here we describe the templated classes {}\texttt{Ptr} and
{}\texttt{RCP} which are used to encapsulate raw C++ pointers to
single objects in more detail.  Again, {}\texttt{Ptr} is used for
non-persisting associations and {}\texttt{RCP} is for persisting
assoications.  Below, and in all of the code listings, we assume that
they are enclosed in the {}\texttt{Teuchos} namespace or there are
appropriate {}\texttt{using Teuchos::XXX} declarations for the various
names in place.

%
{}\subsubsection{\texttt{Teuchos::Ptr<T>}}
%

The templated class {}\texttt{Ptr} is the simplest of all the Teuchos
memory management classes.  In optimized mode it is just the thinest
of wrappers around a raw C++ pointer.  Listing {}\ref{listing:Ptr}
shows what the implementation of {}\texttt{Ptr} looks like in
optimized mode:

\begin{listing}: Teuchos::Ptr class\\
\label{listing:Ptr}
{\small\begin{verbatim}
  template<class T>
  class Ptr {
  public:
    Ptr( ENull null_in = null ) : ptr_(0) {}
    explicit Ptr( T *ptr ) : ptr_(ptr) {}
    Ptr(const Ptr<T>& ptr) : ptr_(ptr.ptr_) {}
    template<class T2> Ptr(const Ptr<T2>& ptr) : ptr_(ptr.ptr_) {}
    Ptr<T>& operator=(const Ptr<T>& ptr) { ptr_=ptr.ptr_; return *this; }
    T* operator->() const { return ptr_; }
    T& operator*() const { return *ptr_; }
    T* get() const { return ptr_; }
    T* getRawPtr() const { return ptr_; }
    const Ptr<T>& assert_not_null() const;
  private:
    T *ptr_;
  };
\end{verbatim}}
\end{listing}

In optimized mode, the only overhead imparted by {}\texttt{Ptr} is the
default initialization to null.  All other functions are just inline
accessors to the underlying raw C++ pointer member {}\texttt{ptr\_}.
Therefore, the performance when using this type is the same as when
using a raw C++ pointer.

However, in debug mode (enabled when {}\texttt{TEUCHOS\_DEBUG} is
defined), then the {}\texttt{Ptr} becomes more complex and performs a
number of runtime checks like for null dereferences and dangling
references (see Section {}\ref{sec:dangling-references}).

One note about the default null constructor shown in Listing
{}\ref{listing:Ptr} which is:

{\small\begin{verbatim}
  template<class T>
  Ptr<T>::Ptr( ENull null_in = null ) : ptr_(0) {}
\end{verbatim}}

{}\noindent{}is that the type {}\texttt{ENull} is the simple enum in
the {}\texttt{Teuchos} namespace:

{\small\begin{verbatim}
  enum ENull { null };
\end{verbatim}}

This simple enum allows for the safe implicit conversion from the enum
value {}\texttt{null} to any {}\texttt{Ptr<T>} object.  For example,
you can write code like:

{\small\begin{verbatim}
  Ptr<A> a_ptr = null;
\end{verbatim}}

This implicit conversion from {}\texttt{null} is shared by the other
Teuchos memory management classes {}\texttt{RCP<T>},
{}\texttt{ArrayView<T>}, and {}\texttt{ArrayRCP<T>}.  This allows you
to call functions that accept one of these objects and just pass in
{}\texttt{null} when appropriate and the implicit conversion will be
done automatically if possible.

The main purpose for the existance of the {}\texttt{Ptr} class is to
replace raw C++ pointers in function calls for typical input,
input/output, and output arguments where no peristing relationship is
present.  For example, consider the function that modifies a type
{}\texttt{A} object shown in Listing {}\ref{listing:modifyA-rawPtr}.

\begin{listing}:\\
\label{listing:modifyA-rawPtr}
{\small\begin{verbatim}
  void modifyA( A *a )
  {
    a->increment();
  }
\end{verbatim}}
\end{listing}

Using {}\texttt{Ptr}, the function {}\texttt{modifyA(...)} in Listing
{}\ref{listing:modifyA-rawPtr} would be changed to the form shown in
Listing {}\ref{listing:modifyA-Ptr}.

\begin{listing}:\\
\label{listing:modifyA-Ptr}
{\small\begin{verbatim}
  void modifyA( const Ptr<A> &a )
  {
    a->increment();
  }
\end{verbatim}}
\end{listing}

In this context, the primary advantage of the form shown in Listing
{}\ref{listing:modifyA-Ptr} as apposed to Listing
{}\ref{listing:modifyA-rawPtr} is that in debug mode, a check for a
null pointer would be performed automatically.  If a null dereference
occured, then an exception would be thrown with a very good error
message. I have seen platforms where a null dereference did not
automatically result in a gracefull assert, stoping the program.  I
have seen cases where somehow memory was corrupted and the program
continued!  My general phylosophy is to make as few assumptions as
possible about the behavior of the compiler and platform because I
have found that ``typical and obvious'' behavior is not universal.  I
have learned the hard way that you will pay a price for such
assumptions in lost time debugging obscure things like a null pointer
dereference that should have stopped the program but did not.  Don't
take chances with your code, take control!

When all of the high-level code has been converted over to use these
memory management classes and there are no more raw C++ pointers, then
client code should never have to construct a {}\texttt{Ptr} using a
raw C++ pointer.  However, as code is being transistioned over and
when such code is called by non-complient code, construction from a
raw pointer is needed.  The recommended way to convert from a raw C++
pointers to {}\texttt{Ptr} is to use the following templated
non-member function:

\begin{listing}: Teuchos::ptr(...)\\
\label{listing:ptr}
{\small\begin{verbatim}
  template<class T>  Ptr<T> ptr(T *p);
\end{verbatim}}
\end{listing}

Using this non-member constructor function, client code would then be
written as shown in Listing {}\ref{listing:using-ptr}.

\begin{listing}:\\
\label{listing:using-ptr}
{\small\begin{verbatim}
  void foo( A* a )
  {
    using Teuchos::ptr;
    modifyA(ptr(a));
  }
\end{verbatim}}
\end{listing}

A more typical use case for the construction of a {}\texttt{Ptr}
object is from a C++ object/reference.  This type of construction
should always be performed using one of the non-member constructor
functions shown in Listing
{}\ref{listing:ptr-from-ref-nonmember-constructors}.

\begin{listing}: Safe nonmember constructors for Teuchos::Ptr\\
\label{listing:ptr-from-ref-nonmember-constructors}
{\small\begin{verbatim}
  template<typename T> Ptr<T>        ptrFromRef( T& arg );
  template<typename T> Ptr<T>        inOutArg( T& arg );
  template<typename T> Ptr<T>        outArg( T& arg );
  template<typename T> Ptr<T>        optInArg( T& arg );
  template<typename T> Ptr<const T>  constOptInArg( T& arg );
\end{verbatim}}
\end{listing}

The different forms of non-member constructor functions shown in
Listing {}\ref{listing:ptr-from-ref-nonmember-constructors} are to
allow for self-documenting code for calls to functions that accept
{}\texttt{Ptr}-wrapped objects.

A complete and comprehensive set of idioms for using {}\texttt{Ptr}
along with the other Teuchos memory management types is given in
Section {}\ref{sec:idioms}.


%
{}\subsubsection{\texttt{Teuchos::RCP<T>}}
%

The class {}\texttt{RCP}, the real workhorse of the Teuchos memory
mangement classes, is used to manage single objects in persisting
associations.  {}\texttt{RCP} is very similar to other high-quality
reference-counted smart pointer classes like
{}\texttt{boost::shared\_ptr} and of course the upcoming standard
C++09 class {}\texttt{std::shared\_ptr}.  However, {}\texttt{RCP} has
some key features that differentiate it from these other better known
smart pointer classes.  In particular, {}\texttt{RCP} has built in
support for the detection of circular references (Section ???), has
built in support for resolving circular references with built-in weak
pointers (Section ???), and other strong debug runtime checking such
as detecting multiple non-related {}\texttt{RCP} objects owning the
same reference-counted objects and other types of checks.

Because the class {}\texttt{RCP} is described in
{}\cite{RefCountPtrBeginnersGuide} and is so similar in use to
{}\texttt{boost::shared\_ptr}, this class will not be described in too
much detail here.  However, a fairly complete definition of the class
{}\texttt{RCP} is shown in Listing {}\ref{listing:RCP} (the full
listing can be found in the Doxygen documentation).

\begin{listing}: Class and helper function listing for RCP \\
\label{listing:RCP}
{\small\begin{verbatim}
  template<class T>
  class RCP {
  public:
    // General functions
    RCP(ENull null_arg = null);
    explicit RCP(T* p, bool has_ownership = false);
    template<class Dealloc_T> RCP(T* p, Dealloc_T dealloc, bool has_ownership);
    RCP(const RCP<T>& r_ptr);
    template<class T2> RCP(const RCP<T2>& r_ptr);
    ~RCP();
    RCP<T>& operator=(const RCP<T>& r_ptr);
    bool is_null() const;
    T* operator->() const;
    T& operator*() const;
    T* getRawPtr() const;
    Ptr<T> ptr() const;
  
    // Other shared_ptr compariblity functions
    ...
  
    // Reference counting 
    ERCPStrength strength() const;
    bool is_valid_ptr() const;
    int strong_count() const;
    int weak_count() const;
    int total_count() const;
    void set_has_ownership();
    bool has_ownership() const;
    Ptr<T> release();
    RCP<T> create_weak() const;
    RCP<T> create_strong() const;
    template<class T2> bool shares_resource(const RCP<T2>& r_ptr) const;
    const RCP<T>& assert_not_null() const;
    const RCP<T>& assert_valid_ptr() const;
  
  private:
    T *ptr_;
    RCPNode node_;
    ...
  };

  // General non-member constructor functions
  template<class T>  RCP<T> rcp(T* p, bool owns_mem = true);
  template<class T> RCP<T> rcpFromRef(T& r);

  // Deallocation policy functions (see Section ???)
  ...
 
  // Conversion functions (see Section ???)
 
  template<class T, class Embedded>
  RCP<T> rcpWithEmbeddedObjPreDestroy(T* p, const Embedded &embedded,
    bool owns_mem = true);
  
  template<class T, class Embedded>
  RCP<T> rcpWithEmbeddedObjPostDestroy(T* p, const Embedded &embedded,
    bool owns_mem = true);
  
  template<class T, class Embedded>
  RCP<T> rcpWithEmbeddedObj(T* p, const Embedded &embedded, bool owns_mem=true);
  
  template<class T> bool is_null(const RCP<T> &p);
  
  template<class T> bool operator==(const RCP<T> &p, ENull);
  
  template<class T> bool operator!=(const RCP<T> &p, ENull);
  
  template<class T1, class T2>
  bool operator==(const RCP<T1> &p1, const RCP<T2> &p2);
  
  template<class T1, class T2>
  bool operator!=(const RCP<T1> &p1, const RCP<T2> &p2);
  
  template<class T1, class T2>
  void set_extra_data(const T1 &extra_data, const std::string& name,
    const Ptr<RCP<T2> > &p, EPrePostDestruction destroy_when = POST_DESTROY,
    bool force_unique = true);
  
  template<class T1, class T2> 
  const T1& get_extra_data(const RCP<T2>& p, const std::string& name);
  
  template<class T1, class T2>
  T1& get_nonconst_extra_data(RCP<T2>& p, const std::string& name);
  
  template<class T1, class T2>
  Ptr<const T1> get_optional_extra_data(const RCP<T2>& p, const std::string& name);
  
  template<class T1, class T2>
  Ptr<T1> get_optional_nonconst_extra_data(RCP<T2>& p, const std::string& name);
  
  template<class TOrig, class Embedded, class T> 
  const Embedded& getEmbeddedObj(const RCP<T>& p);
  
  template<class TOrig, class Embedded, class T>
  Embedded& getNonconstEmbeddedObj(const RCP<T>& p);
  
  template<class TOrig, class Embedded, class T>
  Ptr<const Embedded> getOptionalEmbeddedObj( const RCP<T>& p );
  
  template<class TOrig, class Embedded, class T>
  Ptr<Embedded> getOptionalNonconstEmbeddedObj( const RCP<T>& p );

\end{verbatim}}
\end{listing}

Again, basic usage of the {}\texttt{RCP} class is described in
{}\cite{RefCountPtrBeginnersGuide}.  The basic idiom of smart pointers and
reference counting is fairly well known and is well documented in the
literature and there is a good overivew in {}\cite{RefCountPtrBeginnersGuide}
so basic information will not be repleaded here.  However, some of the more
advanced functionality for {}\texttt{RCP} is described in later chapters that
is not described in {}\cite{RefCountPtrBeginnersGuide} or an any of the
existing C++ literature.


%
{}\subsubsection{Raw C++ references}
%

Why do we have a subsection on raw C++ references under the section
for Teuchos Memory Management classes?  The reason is that raw C++
references to single objects are used in the idioms described in this
paper for non-persisting associations for single objects and this was
a reasonable place to discuss them.

While this paper argues that raw C++ pointers have no place in
application-level code because they are fundamentally unsafe does that
also mean that raw C++ references are also unsafe in application-level
code?  After all, under the covers raw C++ references really are just
raw C++ pointers in disguise.  While this is true, in practice raw C++
references are significantly safer than raw C++ pointers, espeically
if the idioms outlined in this paper are carefully followed.  In
addition, the use of raw C++ references is exploited (as explained in
Section ???) in defining idioms that increase the self-documenting
nature of C++ code and play a role in defining non-persisting
associations related to function formal arguments and return objects.

Basically, a raw C++ reference is relatively safe as long as a) it is
always initialized to a valid object, and b) it is only used for
non-persisting relationships (especially as const input arguments in
C++ functions).  If a raw C++ reference is initalized directly from an
object or from dereferening a smart pointer, then you are guaranteed
that the object will be valid when the reference is first created (at
least in a debug build where dereferencing null smart pointers
throws).  While raw C++ references are faily safe when used with the
idioms described in this paper, there are no 100\% guarantees.  There
is typically no guarantee that the object pointed to by a raw reference
will stay valid, even in cases where you would assume that it should
be if other idioms outlined in this paper are broken or in rare other
cases even if no specific idiom rule is broken.


%
{}\subsection{Memory management classes replacing raw pointers for
arrays of objects}
\label{sec:array-classes}
%

The Teuchos memory management module actually defines four different
C++ classes for dealing with contiguous arrays of objects:
{}\texttt{ArrayView}, {}\texttt{ArrayRCP}, {}\texttt{Array}, and
{}\texttt{Tuple}.  As stated in Section
{}\ref{sec:outline_of_basic_approach} each of these classes is needed
in order to address different important use cases for dealing with
continguous arrays of objects.  The conventions outlined in the paper
never have you exposing a raw C++ pointer to an array or directly
using built-in (statically sized) C++ arrays.

All of the Teuchos array classes provide a common subset of the
interface of {}\texttt{std::vector} which includes the typedefs and
member functions shown in Table {}\ref{tbl:common-array-type-members}.

\begin{table}
{\small\begin{center}
\begin{tabular}{|l|}
\hline
{}\textbf{std::vector compatible typedefs} \\
\hline
{}\texttt{value\_type} \\
{}\texttt{size\_type} \\
{}\texttt{difference\_type} \\
{}\texttt{pointer} \\
{}\texttt{const\_pointer} \\
{}\texttt{reference} \\
{}\texttt{const\_reference} \\
{}\texttt{iterator} \\
{}\texttt{const\_iterator} \\
{}\texttt{element\_type} \\
\hline
{}\textbf{std::vector compatible functions} \\
\hline
{}\texttt{size\_type size()} \\
{}\texttt{[const\_]reference operator{}(size\_type) [const]} \\
{}\texttt{[const\_]reference front() const} \\
{}\texttt{[const\_]reference back() const} \\
{}\texttt{[const\_]iterator begin() [const]} \\
{}\texttt{[const\_]iterator end() [const]} \\
\hline
{}\textbf{ArrayView returning functions} \\
\hline
{}\texttt{ArrayView<[const] T> view(size\_type offset, size\_type size) [const]} \\
{}\texttt{ArrayView<[const] T> operator[]()(size\_type offset, size\_type size) [const]} \\
{}\texttt{ArrayView<[const] T> operator()() [const]} \\
{}\texttt{operator ArrayView<const T>() const} \\
\hline
{}\textbf{Other common functions} \\
\hline
{}\texttt{[const\_]pointer getRawPtr() [const]} \\
{}\texttt{bool is\_null() const} \\
{}\texttt{std::string toString() const} \\
\hline
\end{tabular}
\caption{\label{tbl:common-array-type-members} Common members for
{}\texttt{ArrayView}, {}\texttt{ArrayRCP}, {}\texttt{Array}, and
{}\texttt{Tuple} .}
\end{center}}
\end{table}

A few things to note about the array interface components shown in
Table {}\ref{tbl:common-array-type-members}:

\begin{itemize}

{}\item{}All of the Teuchos array classes are drop-in replacements for
any code that uses {}\texttt{std::vector} that does not grow or shink
the container by supporting the necessary typedefs, query functions,
element access, and iterator access.  This helps in migrating current
code that uses {}\texttt{std::vector} but should be using
{}\texttt{Array}, {}\texttt{ArrayView}, {}\texttt{ArrayRCP} or
{}\texttt{Tuple}.

{}\item{}All of the array classes support returning \texttt{ArrayView}
subviews of contiguous ranges of elements.

{}\item{}All of the array classes support a handy
{}\texttt{getRawPtr()} function that allows a client to get the base
pointer address, or null, to the array.  The standard
{}\texttt{std::vector} class supports no such function which is very
painful for users.

\end{itemize}

The exact functions shown in Table
{}\ref{tbl:common-array-type-members} for {}\texttt{ArrayView} and
{}\texttt{ArrayRCP} and a little different than for {}\texttt{Array}
due to the different nature of these view classes as apposed to the
container class {}\texttt{Array}.  As described in Section ???, the
classes {}\texttt{ArrayView} and {}\texttt{ArrayRCP} can encapsulate
both non-const and const types {}\texttt{T} as their template argument
while {}\texttt{Array} can only accept a non-constr type {}\texttt{T}.
Therefore, the {}\texttt{std::vector} compatible functions in
{}\texttt{ArrayView} and {}\texttt{ArrayRCP} are all {}\texttt{const}
functions since they don't change what data these objects point to,
but only change the data themselves.

One other aspect to note about the Teuchos array classes is that they
deviate from the standard C++ library convention of using an unsigned
integer for {}\texttt{size\_type}.  Instead, they use a signed integer
for {}\texttt{size\_type} typedefed to the signed type
{}\texttt{Teuchos\_Ordinal} which is 32 bit on a 32 bit machine and 64
bit on a 64 bit machine\footnote{{}\texttt{Teucohs\_Ordinal} is
typedefed to the standard C library type {}\texttt{ptrdiff\_t} which
is always signed and 32 bit or 64 bit which are always 32 bit or 64
bit machines, respectively.}.  The reasoning for breaking from the
{}\texttt{std::vector} standard for {}\texttt{size\_type} is described
in Appendix {}\ref{sec:unsigned_size_type}.


%
{}\subsubsection{\texttt{Teuchos::ArrayView<T>}}
%

The class {}\texttt{ArrayView}, the simplest of the Teuchos array
memory management classes, is designed to replace raw pointers in
non-persisting associations primarily for formal function array
arguemnts.  In an optimized build, it simply holds a raw base array
pointer and an integer size.  In an optimized build,
{}\texttt{ArrayView} looks like in Listing {}\ref{listing:ArrayView}.

{}\begin{listing}: {}\texttt{Teuchos::ArrayView} Declaration (See
Table {}\ref{tbl:common-array-type-members} for common array members.)
\label{listing:ArrayView}
{\small\begin{verbatim}

  template<class T>
  class ArrayView {
  public:

    // Constructors/Assignment/Destructors
    ArrayView( ENull null_arg = null );
    ArrayView( T* p, size_type size );
    ArrayView(const ArrayView<T>& array);
    ArrayView(std::vector<typename ConstTypeTraits<T>::NonConstType>& vec);
    ArrayView(const std::vector<typename ConstTypeTraits<T>::NonConstType>& vec);
    ArrayView<T>& operator=(const ArrayView<T>& array);
    ~ArrayView();

    // Deep copy  
    void assign(const ArrayView<const T>& array) const;

    // Common array class members (see above) and other functions
    ...

  private:
    T *ptr_;     // Optimized implementation
    int size_;

  };

  // Non-member helpers

  template<class T>
  ArrayView<T> arrayView( T* p, typename ArrayView<T>::size_type size );

  template<class T>
  ArrayView<T> arrayViewFromVector( std::vector<T>& vec );

  template<class T>
  ArrayView<const T> arrayViewFromVector( const std::vector<T>& vec );

  template<class T>
  std::vector<T> createVector( const ArrayView<T> &av );

  template<class T>
  std::vector<T> createVector( const ArrayView<const T> &av );

  template<class T>
  bool is_null( const ArrayView<T> &av );

  template<class T>
  bool nonnull( const ArrayView<T> &av );

  template<class T>
  std::ostream& operator<<( std::ostream& out, const ArrayView<T>& av );

  template<class T2, class T1>
  ArrayView<T2> av_reinterpret_cast(const ArrayView<T1>& p1);

\end{verbatim}}
\end{listing}

ToDo: In the code, match up the above set of delcarations with the
class ArrayView itself.  Also, test all the typedefs to make sure they
work correctly.

A few things to note about {}\texttt{ArrayView} shown in
{}\ref{listing:ArrayView} in addition to the comments in Section
{}\ref{sec:array-classes} include:

\begin{itemize}

{}\item{}{}\texttt{ArrayView} is extremely lightweight in an optimized
build, carrying only a pointer and in integer size.  This allows you
to replace the typical pointer and separate size argument with a
single aggregate light weight object.  Therefore, it yield very
efficient code.

{}\item{}\texttt{ArrayView} in optimized mode has all trivial inlined
functions that work with the raw pointer so it is as efficient has raw
pointer code.

{}\item{}{}\texttt{ArrayView} is a drop in replacement for any code
that uses {}\texttt{std::vector} that does not grow or shink the
container by supporting the necessary typedefs, query functions, and
iterator access.  This helps in migrating current code that uses
{}\texttt{std::vector} but should be using {}\texttt{ArrayView}.

{}\item{}{}\texttt{ArrayView} implicitly converts from an
{}\texttt{std::vector} so functions called by existing client code
that uses {}\texttt{std::vector} can be safely and transparently
refactored to use {}\texttt{ArrayView} instead of
{}\texttt{std::vector}.

{}\item{}\texttt{ArrayView} directly supports the creation of subviews
of contiguous ranges of elements.

{}\item{}\texttt{ArrayView} Supports all safe raw array pointer
implicit convesions such as from non-const to a const but does not
support implicit conversion from derived to base types which is almost
always a programming error (see Section
{}\ref{sec:problem-with-raw-array-pointers}).  Also supported is
{}\texttt{reinterpret\_cast} which is not safe but is a valid type of
conversion of array types ins low-level code.

{}\item{}\texttt{AraryView}s of {}\texttt{Array} and
{}\texttt{ArrayRCP} objects are fully supported as subviews and
implicit conversions (see Section
{}\ref{sec:teuchos-type-conversions}).

{}\item{}\texttt{ArrayView} supports a handy {}\texttt{getRawPtr()}
function that allows a client to get the base pointer address, or
null, to the array.  The standard {}\texttt{std::vector} class
supports no such function which is very painful for users.

\end{itemize}

What makes {}\texttt{ArrayView} non-trivial and special, however, is
that in a debug build, the implementation takes on a variety of
runtime checking to catch all sorts of errors such as dangling
iterators and other sub-views (Section
{}\ref{sec:dangling-references}), range checking (Section
{}\ref{sec:null-dereferences-range-checking}), and other types of
checking.

It should be noted that you should almost never create an
{}\texttt{ArrayView} object directly from a raw pointer but instead
create them as views of {}\texttt{Array}, {}\texttt{ArrayRCP},
{}\texttt{Tuple } and other {}\texttt{ArrayView} object.  If you are
routinely creating {}\texttt{ArrayView} from raw pointers, then your
code is not safe and you need to study the core idioms described in
Section {}\ref{sec:idioms}.

The class {}\texttt{ArrayView} has no equivalent in boost or the
current C++ or prposed C++0x standard.  This is a critical class
needed to allow for flexibility.


%
{}\subsubsection{\texttt{Teuchos::ArrayRCP<T>}}
%

The class {}\texttt{ArrayRCP} is the counterpart to
{}\texttt{ArrayView} for general flexible array views except it is
used for persisting relationships where reference-counting machinary
is required.  An {}\texttt{ArrayRCP} object can provide a contiguous
view into any array of data allocated in any way possible and can
allow the user to define what is done to release memory any way they
would like.

The class declaration for {}\texttt{Teuchos::ArrayRCP} is shown in
Listing {}\ref{listing:ArrayRCP}.

\begin{listing}: {}\texttt{Teuchos::ArrayRCP} Declaration (optimized build)\\
\label{listing:ArrayRCP}
{\small\begin{verbatim}
  template<class T>
  class ArrayRCP {
  public:
  
    // Constructors/initializers
    ArrayRCP(ENull null_arg=null);
    ArrayRCP(T* p, size_type lowerOffset, size_type upperOffset,
      bool has_ownership);
    template<class Dealloc_T>
      ArrayRCP( T* p, size_type lowerOffset, size_type upperOffset,
        Dealloc_T dealloc, bool has_ownership);
    explicit ArrayRCP(size_type lowerOffset, const T& val = T());
    ArrayRCP(const ArrayRCP<T>& r_ptr);
    ~ArrayRCP();
    ArrayRCP<T>& operator=(const ArrayRCP<T>& r_ptr);

    // Object/Pointer Access Functions 
    T* operator->() const;
    T& operator*() const;
    ArrayRCP<T>& operator++();
    ArrayRCP<T> operator++(int);
    ArrayRCP<T>& operator--();
    ArrayRCP<T> operator--(int);
    ArrayRCP<T>& operator+=(size_type offset);
    ArrayRCP<T>& operator-=(size_type offset);
    ArrayRCP<T> operator+(size_type offset) const;
    ArrayRCP<T> operator-(size_type offset) const;
  
    // ArrayRCP Views 
    ArrayRCP<const T> getConst() const;
    ArrayRCP<T> persistingView(size_type lowerOffset, size_type size) const;
  
    // Size and extent query functions 
    size_type lowerOffset() const;
    size_type upperOffset() const;
    size_type size() const;
  
    // Implicit conversions
    operator ArrayView<T>() const;
    operator ArrayRCP<const T>() const;
  
    // std::vector like and other misc functions
    void assign(size_type n, const T &val);
    template<class Iter>
      void assign(Iter first, Iter last);
    void deepCopy(const ArrayView<const T>& av);
    void resize(const size_type n, const T &val = T());
    void clear();

    // Common array class members (see above)
    ...
  
    // Reference counting (same as for RCP)
    ...
  
  private:
    T *ptr_; // NULL if this pointer is null
    RCPNodeHandle node_; // NULL if this pointer is null
    size_type lowerOffset_; // 0 if this pointer is null
    size_type upperOffset_; // -1 if this pointer is null
  };
  
  // Nonmember functions
  
  template<class T>
  ArrayRCP<T> arcp(T* p, typename ArrayRCP<T>::size_type lowerOffset,
    typename ArrayRCP<T>::size_type size, bool owns_mem = true);
  
  template<class T, class Dealloc_T>
  ArrayRCP<T> arcp(T* p, typename ArrayRCP<T>::size_type lowerOffset,
    typename ArrayRCP<T>::size_type size, Dealloc_T dealloc, bool owns_mem);
  
  template<class T>
  ArrayRCP<T> arcp( typename ArrayRCP<T>::size_type size );
  
  template<class T>
  ArrayRCP<T> arcpClone( const ArrayView<const T> &v );
  
  template<class T, class Embedded>
  ArrayRCP<T>
  arcpWithEmbeddedObjPreDestroy(T* p, typename ArrayRCP<T>::size_type lowerOffset,
    typename ArrayRCP<T>::size_type size, const Embedded &embedded,
     bool owns_mem=true);
  
  template<class T, class Embedded>
  ArrayRCP<T>
  arcpWithEmbeddedObjPostDestroy(T* p, typename ArrayRCP<T>::size_type lowerOffset,
    typename ArrayRCP<T>::size_type size, const Embedded &embedded,
    bool owns_mem=true);
  
  template<class T, class Embedded>
  ArrayRCP<T>
  arcpWithEmbeddedObj(T* p, typename ArrayRCP<T>::size_type lowerOffset,
    typename ArrayRCP<T>::size_type size, const Embedded &embedded,
    bool owns_mem=true);
  
  template<class T>
  ArrayRCP<T> arcp(const RCP<std::vector<T> > &v);
  
  template<class T>
  ArrayRCP<const T> arcp(const RCP<const std::vector<T> > &v);
  
  template<class T>
  ArrayRCP<T> arcpFromArrayView(const ArrayView<T> &av);
  
  template<class T>
  RCP<std::vector<T> > get_std_vector(const ArrayRCP<T> &ptr);
  
  template<class T>
  RCP<const std::vector<T> > get_std_vector(const ArrayRCP<const T> &ptr);
  
  template<class T>
  bool is_null(const ArrayRCP<T> &p);
  
  template<class T>
  bool nonnull(const ArrayRCP<T> &p);
  
  template<class T>
  bool operator==(const ArrayRCP<T> &p, ENull);
  
  template<class T>
  bool operator!=(const ArrayRCP<T> &p, ENull);
  
  template<class T1, class T2>
  bool operator==(const ArrayRCP<T1> &p1, const ArrayRCP<T2> &p2);
  
  template<class T1, class T2>
  bool operator!=(const ArrayRCP<T1> &p1, const ArrayRCP<T2> &p2);
  
  template<class T1, class T2>
  bool operator<(const ArrayRCP<T1> &p1, const ArrayRCP<T2> &p2);
  
  template<class T1, class T2>
  bool operator<=(const ArrayRCP<T1> &p1, const ArrayRCP<T2> &p2);
  
  template<class T1, class T2>
  bool operator>(const ArrayRCP<T1> &p1, const ArrayRCP<T2> &p2);
  
  template<class T1, class T2>
  bool operator>=(const ArrayRCP<T1> &p1, const ArrayRCP<T2> &p2);
  
  template<class T>
  typename ArrayRCP<T>::difference_type
  operator-(const ArrayRCP<T> &p1, const ArrayRCP<T> &p2);
  
  template<class T2, class T1>
  ArrayRCP<T2> arcp_const_cast(const ArrayRCP<T1>& p1);
  
  template<class T2, class T1>
  ArrayRCP<T2> arcp_reinterpret_cast(const ArrayRCP<T1>& p1);
  
  template<class T2, class T1>
  ArrayRCP<T2> arcp_implicit_cast(const ArrayRCP<T1>& p1);
  template<class T1, class T2>
  void set_extra_data(const T1 &extra_data, const std::string& name,
    const Ptr<ArrayRCP<T2> > &p, EPrePostDestruction destroy_when = POST_DESTROY,
    bool force_unique= rue );
  
  template<class T1, class T2>
  T1& get_extra_data( ArrayRCP<T2>& p, const std::string& name );
  
  template<class T1, class T2>
  const T1& get_extra_data( const ArrayRCP<T2>& p, const std::string& name );
  
  template<class T1, class T2>
  T1* get_optional_extra_data( ArrayRCP<T2>& p, const std::string& name );
  
  template<class T1, class T2>
  const T1* get_optional_extra_data(const ArrayRCP<T2>& p,
    const std::string& name);
  
  template<class Dealloc_T, class T>
  Dealloc_T& get_nonconst_dealloc( const ArrayRCP<T>& p );
  
  template<class Dealloc_T, class T>
  const Dealloc_T& get_dealloc( const ArrayRCP<T>& p );
  
  template<class Dealloc_T, class T>
  const Dealloc_T* get_optional_dealloc( const ArrayRCP<T>& p );
  
  template<class Dealloc_T, class T>
  Dealloc_T* get_optional_nonconst_dealloc( const ArrayRCP<T>& p );
  
  template<class TOrig, class Embedded, class T>
  const Embedded& getEmbeddedObj( const ArrayRCP<T>& p );
  
  template<class TOrig, class Embedded, class T>
  Embedded& getNonconstEmbeddedObj( const ArrayRCP<T>& p );
  
  template<class T>
  std::ostream& operator<<( std::ostream& out, const ArrayRCP<T>& p );
\end{verbatim}}
\end{listing}

Some of the main features of the {}\texttt{ArrayRCP} class are:

\begin{itemize}

{}\item ???

\end{itemize}

Some of the other features of the {}\texttt{RCP} class are discussed
in Sections ???, ???, and ???.


%
{}\subsubsection{\texttt{Teuchos::Array<T>}}
\label{sec:Array}
%

The class {}\texttt{Array} is a complete replacement for
{}\texttt{std::vector} that is integrated with the
{}\texttt{ArrayView} class for debug-mode runtime checking.  In an
optimized build, {}\texttt{Array} is nothing but an inline wrapper
around a fully encapsulated {}\texttt{std::vector} object.  This means
that in a optimized build, {}\texttt{Array} takes advantage of all of
the platform-specific optimizations contained in the native
{}\texttt{std::vector} implementation and imparts no extra space/time
overhead.  However, in a debug build, a full set of
platform-independent runtime checking is performed that is as strong
or stronger than any checked STL implementation (see {}\cite[Item
\#83]{C++CodingStandards05}) and includes dangling reference detection
and reporting of {}\texttt{ArrayView} views (see Section ???).  In
addition, {}\texttt{Array} supports better runtime debug output with
better exception error messages.

The class declaration for the {}\texttt{Array} class is shown in
Listing ???.

ToDo: Fill this in!


%
{}\subsubsection{\texttt{Teuchos::Tuple<T,N>}}
%

ToDo: Fill this in!


%
{}\subsubsection{Array Views}
%

ToDo: Fill this in!


%
{}\subsection{Const verses non-const pointers and objects}
\label{sec:teuchos-const-nonconst-pointer-objects}
%

The core pointer enacapsulation classes {}\texttt{Ptr},
{}\texttt{RCP}, {}\texttt{ArrayView} and {}\texttt{ArrayRCP} allow for
the underlying object (or array of objects) to be const or non-const
and for the outer pointer type to be const or non-const, just like
with a regular C++ pointer.  To draw the analogy with raw pointers,
consider the equivalent declarations of a raw pointer and the pointer
encapsulation class (using {}\texttt{RCP} as a standin to represent
all four classes) in Table
{}\ref{sec:teuchos-const-nonconst-pointer-objects}.

\begin{table}
%
%\fbox{
\begin{center}
%
%\begin{minipage}{{}\textwidth}
%
\begin{tabular}{|l|l|l|}
%
\multicolumn{3}{c}{\textbf{Equivalencies for const production for raw and smart pointers}} \\
\hline
Description
& Raw pointer 
& Smart pointer \\
\hline
\hline
Basic declaration (non-const obj)
& {}\texttt{typedef A* ptr\_A}
& {}\texttt{RCP<A>} \\
\hline
Basic declaration (const obj)
& {}\texttt{typedef const A* ptr\_const\_A}
& {}\texttt{RCP<const A>} \\
\hline
\hline
non-const pointer, non-const object
& {}\texttt{ptr\_A}
& {}\texttt{RCP<A>} \\
\hline
const pointer, non-const object
& {}\texttt{const ptr\_A}
& {}\texttt{const RCP<A>} \\
\hline
non-const pointer, const object
& {}\texttt{ptr\_const\_A}
& {}\texttt{RCP<const A>} \\
\hline
const pointer, const object
& {}\texttt{const ptr\_const\_A}
& {}\texttt{const RCP<const A>} \\
\hline
\end{tabular}

%
%\end{minipage}
%
\end{center}
%} % end fbox
\caption{\label{sec:teuchos-const-nonconst-pointer-objects}
Equivalences between raw pointer and smart pointer types for const
protection.}
%
\end{table}

The majority of problems that beginners have with the Teuchos memory
management classes is relataed to the inability to make the basic
equivalencies between raw pointers and smart pointers shown in Table
{}\ref{sec:teuchos-const-nonconst-pointer-objects} (see Section for
specific examples).  It is crtical that the programmer recognize this
equivalence with raw pointers because it impacts many things
especially implicit type conversions to satisfy function calls (again
see Section ???).


%
{}\subsection{Conversions}
\label{sec:conversions}
%

Type conversions exist both for a single smart pointer type for the
embedded type argument (e.g.\ {}\texttt{RCP<T>} implicitly converts to
{}\texttt{RCP<const T>}) and also between different smart pointer
types (e.g.\ {}\texttt{Array} implicitly converts to
{}\texttt{ArrayView}).  There are implicit conversions and explicit
conversions.  These two types of converions are described in the
following two sections.

%
{}\subsubsection{Implicit and explicit raw-pointer like type conversions}
\label{sec:raw-pointer-like-type-conversions}
%

The core Teuchos memory management types {}\texttt{Ptr},
{}\texttt{RCP}, {}\texttt{ArrayView} and {}\texttt{ArrayRCP} allow all
of the reasonable implicit and explicit type conversions that are
allowed by raw C++ pointers.  C++ defines implicit convers for raw
pointers from non-const to const and from derived to base type.  Table
??? shows what implicit and explicit conversions are supported for the
four core memory management types.

\begin{table}
\begin{center}
\begin{tabular}{|l|c|c|c|c|}
\multicolumn{5}{l}{\textbf{Implicit and explicit conversions supported}} \\
\hline
Operation
& \texttt{Ptr<T>}
& \texttt{RCP<T>}
& \texttt{ArrayView<T>}
& \texttt{ArrayRCP<T>} \\
\hline
\hline
Implicit conv derived to base
& x
& x
& 
&  \\
\hline
Implicit conv non-const to const
& x
& x
& x
& x \\
\hline
{}\texttt{const\_cast}
& x
& x
& x
& x \\
\hline
{}\texttt{static\_cast}
& x
& x
& 
&  \\
\hline
{}\texttt{dynamic\_cast}
& x
& x 
& 
&  \\
\hline
{}\texttt{reinterpret\_cast}
& 
& 
& x
& x \\
\hline
\end{tabular}
\end{center}
\caption{\label{table:implicit_explicit_conversions}
Implicit and explicit conversions supported by Teuchos memory
management types.}
\end{table}

As can be seen in Table {}\ref{table:implicit_explicit_conversions},
the smart pointer types for single objects do not support the same
implicit and explicit conversions as for the smart pointer types for
arrays.  The justification for this was explained in Section
{}\ref{sec:problem-with-raw-array-pointers}.

As explained in Section {}\ref{sec:problem-with-raw-array-pointers},
it almost always incorrect and dangerous to allow implicit conversions
from derived to base type for pointers that pointers that point into
continguous arrays of objects.  Therefore, the types
{}\texttt{ArrayView} and {}\texttt{ArrayRCP} do not support implicit
conversions from derived to base types.  Due to similar logic, it
almost never makes any sense to perform a static cast or a dynamic on
a pointer to an array of contiguous objects so the types
{}\texttt{ArrayView} and {}\texttt{ArrayRCP} do not support static and
dynamic casts.

In well formed programs, there is a justification to perform
reinterpret casts for contingous arrays of POD (plain old data).  For
example, as mentioned in Section ???, it is prefectly reasonable to
allocate a large array of {}\texttt{char} (untyped) data and then
create subviews and reinterpret cast to separate arrays of
{}\texttt{double} and {}\texttt{int} data, for instance.  However, for
single objects, I can not think of a single valid reason to perform a
reinterpret cast and therefore, reinterpret cast is not supported for
the types {}\texttt{Ptr} and {}\texttt{RCP}.

All implicit conversions are supported by templated copy constructors
for all four classes.  The explicit conversion operators for these
four types are shown in the Listings {}\ref{Ptr-conversions},
{}\ref{RCP-conversions}, {}\ref{ArrayView-conversions},
{}\ref{ArrayRCP-conversions}.

\begin{listing}: Conversion functions for Ptr\\
\label{listing:Ptr-conversions}
{\small\begin{verbatim}
  template<class T2, class T1> Ptr<T2> ptr_implicit_cast(const Ptr<T1>& p1);
  template<class T2, class T1> Ptr<T2> ptr_static_cast(const Ptr<T1>& p1);
  template<class T2, class T1> Ptr<T2> ptr_const_cast(const Ptr<T1>& p1);
  template<class T2, class T1> Ptr<T2> ptr_dynamic_cast(const Ptr<T1>& p1,
    bool throw_on_fail = false);
\end{verbatim}}
\end{listing}

\begin{listing}: Conversion functions for RCP\\
\label{listing:RCP-conversions}
{\small\begin{verbatim}
  template<class T2, class T1> RCP<T2> rcp_implicit_cast(const RCP<T1>& p1);
  template<class T2, class T1> RCP<T2> rcp_static_cast(const RCP<T1>& p1);
  template<class T2, class T1> RCP<T2> rcp_const_cast(const RCP<T1>& p1);
  template<class T2, class T1> RCP<T2> rcp_dynamic_cast(const RCP<T1>& p1,
    bool throw_on_fail = false);
\end{verbatim}}
\end{listing}

\begin{listing}: Conversion functions for ArrayView\\
\label{listing:ArrayView-conversions}
{\small\begin{verbatim}
  template<class T2, class T1> ArrayView<T2> av_const_cast(const ArrayView<T1>& p1);
  template<class T2, class T1> ArrayView<T2> av_reinterpret_cast(const ArrayView<T1>& p1);
\end{verbatim}}
\end{listing}

\begin{listing}: Conversion functions for ArrayRCP\\
\label{listing:ArrayRCP-conversions}
{\small\begin{verbatim}
  template<class T2, class T1> ArrayRCP<T2> arcp_const_cast(const ArrayRCP<T1>& p1);
  template<class T2, class T1> ArrayRCP<T2> arcp_reinterpret_cast(const ArrayRCP<T1>& p1);
\end{verbatim}}
\end{listing}

Note that the dynamic cast conversion functions
{}\texttt{ptr\_dynamic\_cast()} and {}\texttt{ptr\_dynamic\_cast()}
both take an option extra argument {}\texttt{throw\_on\_fail} that if
set to {}\texttt{true} will result in an excpetion being thrown on a
dynamic cast failure along with a very helpful error message.


%
{}\subsubsection{Conversions between different memory management types}
\label{sec:teuchos-type-conversions}
%

It is critical that all conversions between the various Teuchos memory
management classes be performed using conversion code provided by the
memory management classes themselves or by associated helper
functions.  Client code should never convert between memory management
types by exposing a raw C++ pointer.  As soon as a raw C++ pointer is
exposed, nearly all of the debug-mode runtime checking will be
disabled.


{\bsinglespace
\begin{figure}
\begin{center}
%\fbox{
\includegraphics*[angle=270,scale=0.65]{TeuchosPtrConversions}
%}
\end{center}
\caption{
\label{fig:TeuchosPtrConversions}
Conversions between Teuchos memory management single object pointer
types.}
\end{figure}
\esinglespace}


{\bsinglespace
\begin{figure}
\begin{center}
%\fbox{
\includegraphics*[angle=270,scale=0.65]{TeuchosArrayConversions}
%}
\end{center}
\caption{
\label{fig:TeuchosArrayConversions}
Conversions between Teuchos memory management array types.  }
\end{figure}
\esinglespace}


Figures {}\ref{fig:TeuchosPtrConversions} and
{}\ref{fig:TeuchosArrayConversions} show many of the types of
conversions that are supported between the different memory mangement
types.  For single objects, the conversions between different
{}\texttt{RCP} and {}\texttt{Ptr} objects of various kinds shown in
figures {}\ref{fig:TeuchosPtrConversions} include both implicit and
explicit conversions.  Conversions between different array types as
shown in Figure {}\ref{fig:TeuchosArrayConversions} include both
implicit and explicit conversions and view and copy conversions
yielding four different types of conversions.

Conversions between different Teuchos memory management types,
espeically in templated application code, is one of the most confusing
aspects of using these classes.  As shown in Figures
{}\ref{fig:TeuchosPtrConversions} and
{}\ref{fig:TeuchosArrayConversions}, many different implicit
conversions are defined.  An implicit conversion will only be
performed by the C++ compiler when it is needed to allow a piece of
code to compile, when a only a single implicit conversion is defined,
and when not calling a templated function where the template arguments
are not fully specified.  Also, the C++ compiler will not be able to
do an implicit conversion when an ambiguous function call exists.
Explaining the behavior of these implicit conversions in C++ gets down
to the nitty gritty details of the C++ type system that many C++
programers take for grated or don't understand all that well.

Almost all of the problems that people have with implicit conversions
occur when trying to call functions where implicit conversions are
required to satisfy the signature of the function.  Some of these
problems occur when developers fail to understand the C++ type system.
Other problems are due to a fundamental handicap that smart pointer
types have with respect to raw C++ pointers.

Implicit conversions of the Teuchos memory mangament classes (or any
other C++ classes in any other library) fail for one of the following
reasons:

\begin{enumerate}

{}\item{}Implicit conversions to functions failing because memory
management types are not passed by const reference (or by value) and
are mistakenly (or on purpose) passed by non-const reference. (This is
a programming error.)

{}\item{}Implicit conversions failing because templated functions can
not perform implicit conversions in order to satisfy a call. (This is
a language usability annoyance associated with templates and has
nothing to do with the Teuchos memory management classes themselves.)

{}\item{}Implicit conversions failing due to ambiguous calls to
overloaded functions that would work just fine when raw C++ pointers
are involved. (This is a fundamental shortcoming of smart pointers or
any other class as compared to raw C++ pointers.)

\end{enumerate}

ToDo: Give code examples for all of the above reasons that implicit
conversions fail.

%
{}\subsection{Core idioms for the use of memory management classes}
\label{sec:idioms}
%

Well designed C++ class libraries are created together with a set of
idioms for their use and this is especially true for the Teuchos
Memory Management classes.  This paper describes idoms related to the
creation of single dynamically allocated objects, for defining and
using local variables and data members, for passing objects and arrays
of objects to and from functions, and for returning objects and arrays
of objects as return values from functions.  It is critical that these
idioms be used consistently in order to yield the safest, highest
quality, clearest code.


%
{}\subsubsection{The non-member constructor function idiom}
\label{sec:nonmember-constructor-idiom}
%

The mainstream C++ literature espousing the use of smart
reference-counted pointers like {}\texttt{boost::shared\_ptr} seems to
lack a solution for an effective, safe, and clean way to create new
dynamically allocated objects.  To demonstrate the issues involved,
consider the C++ class {}\texttt{Blaget} shown in Listing
{}\ref{listing:BlagetClass}:

\begin{listing}: A class taking multiple dynamically allocatable objects \\
\label{listing:BlagetClass}
{\small\begin{verbatim}
  class Blaget {
  public:
    Blaget(const RCP<Widget> &widgetA, const RCP<Widget> const widgetB);
      widgetA_(widget), widgetB_(widget) {}
    ...
  private:
    RCP<Widget> widgetA_;
    RCP<Widget> widgetB_;
  };
\end{verbatim}}
\end{listing}

Now consider how you might go about constructing a {}\texttt{Blaget}
object on the stack given newly dynamically allocated
{}\texttt{Widget} objects.  A compact, clean, and seemingly safe way
to do so is shown in Listing {}\ref{listing:BlagetConstruct1}.

\begin{listing}: A leaky way to construct \\
\label{listing:BlagetConstruct1}
{\small\begin{verbatim}
  Blaget blaget( rcp(new Widget()), rcp(new Widget()) );
\end{verbatim}}
\end{listing}

The problem with the code in Listing {}\ref{listing:BlagetConstruct1}
is that it might result in a memory leak if an exception is thrown by
one of the constructors for {}\texttt{Widget} (see {}\cite[Item
13]{C++CodingStandards05}).  The reason that a memory leak might occur
is that a C++ compiler is allowed to evaluate both {}\texttt{new
Widget()} calls before calling the {}\texttt{rcp()} functions.
If the second constructor {}\texttt{Widget()} throws an exception
after the first {}\texttt{Widget()} constructor has been invoked
but before the {}\texttt{RCP} object wrapping the first
{}\texttt{Widget} object is constructed, then the memory created by
the first {}\texttt{new Widget()} will never be reclaimed.

The current C++ literature (see {}\cite[Item
13]{C++CodingStandards05}) recommends rewriting constructor code like
shown in Listing {}\ref{listing:BlagetConstruct1} using temporary local
varibles as shown in Listing {}\ref{listing:BlagetConstruct2}.

\begin{listing}: A sound but verbose way to construct \\
\label{listing:BlagetConstruct2}
{\small\begin{verbatim}
  RCP<Widget> widgetA(new Widget());
  RCP<Widget> widgetB(new Widget());
  Blaget blaget(widgetA, widgetB);
\end{verbatim}}
\end{listing}

While the code in Listing {}\ref{listing:BlagetConstruct2} will avoid a
memory leak being created in case an exception is thrown, competent
Java and Python programs will rightfully be disgusted that they have
to create temporary variables just to call another constructor.  From
a software engineering perspective, it is undesirable to create
useless local {}\texttt{RCP} objects like {}\texttt{widgetA} and
{}\texttt{widgetB} because they might be inadvertently copied and used
for other purposes, resulting in undesirable side-effects.

The way to solve the problems described above is to provide non-member
constructor functions for all of your dynamically allocatable
reference-type classes and then always call them to create your
{}\texttt{RCP}-wrapped objects in client code.  In fact, to avoid
mistakes when using reference-type classes, you should disallow the
creation of the objects except through a provided non-member
constructor.  A {}\textit{non-member constructor} complient
{}\texttt{Widget} class declaration is shown in Listing
{}\ref{listing:WidgetNonmemberConstructor}.

\begin{listing}: The Non-member Constructor Idom\\
\label{listing:WidgetNonmemberConstructor}
{\small\begin{verbatim}
  class Widget {
    friend RCP<Widget> widget();
  public: // Public interface
    void display(std::ostream&);
  private: // or protected
    Widget(); // Not for user's to call!
  };

  // Non-member constructor function
  inline
  RCP<Widget> widget()
  {
    return rcp(new Widget());
  }
\end{verbatim}}
\end{listing}

Using the non-member constructor function {}\texttt{widget()}, the
unsafe constructor call in Listing {}\ref{listing:BlagetConstruct1} can
be written as shown in Listing {}\ref{listing:BlagetConstruct3}.

\begin{listing}: Clean and bullet-proof way using non-member
constructor function idiom \\
\label{listing:BlagetConstruct3}
{\small\begin{verbatim}
  Blaget blaget(widget(), widget());
\end{verbatim}}
\end{listing}

The code in Listing {}\ref{listing:BlagetConstruct3} will never result
in a memory leak if an exception is thrown because each argument is
returned as a fully formed {}\texttt{RCP} object which will clean up
memory if any exception is thrown.

Note that the use of the {}\textit{non-member constructor idom} not
only means that raw calls to {}\texttt{delete} are eliminated from all
high-level C++ code, but it also means that raw calls to
{}\texttt{new} should be largely eliminated as well!

The non-member constructor idiom as shown in Listing
{}\ref{listing:WidgetNonmemberConstructor} where a reference-type
object can only be dynamically allocated and returned wrapped in an
{}\texttt{RCP} object is recommended for all reference-type objects.
The reason for this is that, as described in Section ???, when an
object is dynamically allocated in managed in an {}\texttt{RCP}
object, a number of important debug-mode runtime checks can be
performed which cannot be when the object is first allocated on the
stack or managed as a static object.


%
{}\subsubsection{General idioms for handling arrays of objects}
%

Before describing specific idioms for class data members, formal
function arguments, and function return types it is worth discussing
how arrays of objects are treated in a common way in all of these
idioms and why.  A common set of idioms that is used throughout is how
arrays of value-type objects and reference-types objects are handled.
When dealing with an array of value-type objects, typically a
contiguous array of objects will be allocated.  For example, to create
an array of value-type objects you would declare:

{\small\begin{verbatim}
  Array<S> valTypeArray;
\end{verbatim}}

In this case, the array holding value-type objects and the value-type
objects themselves are one in the same.  The same goes for persisting
and non-persisting views of array of value-type objects represented as
{}\texttt{ArrayRCP<[const] S>} and {}\texttt{ArrayView<[const] S>},
respectively.  It is common for numerical programs to create very
large arrays of value-type objects such as for integers and floating
point numbers.  Therefore, it is usually important to share these
arrays and pass them around instead of creating copies.  Because if
this, it is typical to see {}\texttt{ArrayRCP<[const] S>} being used
to share large value-type arrays of objects.

On the other hand, we can't generally allocate a contiguous array of
reference-type objects.  Instead, we have to allocate and use a
contiguous array of (smart) pointer objects that then point to
individually allocated reference-type objects.  For example, to store
an array of dynamically allocated reference-type objects, you would
delcare:

{\small\begin{verbatim}
  Array<RCP<A> > refTypeArray;
\end{verbatim}}

Anyone familiar with object-oriented programming in C++ should already
knows this, but they might be used to allocating and working with
arrays of raw pointers like {}\texttt{std::vector<T*>}.  This is a
really bad idea of course which is mentioned in Item 79 ``Store only
values and smart pointers in containers'' in
{}\cite{C++CodingStandards05}.  In this case, we can think of the
array of {}\texttt{RCP} value-type objects and the reference-type
objects of type {}\texttt{A} themselves to be different objects.  For
example, you can change what {}\texttt{A} object is pointed to in the
{}\texttt{RCP<A>} object stored in the contiguous array to without
changing the {}\texttt{A} object itself such as with:

{\small\begin{verbatim}
  void foo(Array<RCP<A> > &refTypeArray, const RCP<A> &someA)
  {
    refTypeArray[0] = someA;
  }
\end{verbatim}}

Likewise, we can change an {}\texttt{A} object itself without
disturbing the {}\texttt{Array<RCP<A> >} object itself such as with:

{\small\begin{verbatim}
  void foo(const Array<RCP<A> > &refTypeArray)
  {
    refTypeArray[0]->someChange();
  }
\end{verbatim}}

As opposed to value-type arrays, we typically do not create large
arrays of reference-type objects.  As a result, we usually don't care
to share the array storage of {}\texttt{Ptr} or {}\texttt{RCP} objects
itself, only the reference-type objects they point to.  Because of
this, you typically will not see {}\texttt{ArrayRCP<[const]
RCP<[const] A> >} objects being passed around and stored.  Instead,
you would typically just pass {}\texttt{ArrayView<[const] RCP<[const]
A> >} objects and then use this array to create a new
{}\texttt{Array<[const] RCP<[const] A> >} object to copy the smart
pointers.  In general, we use arrays of {}\texttt{RCP} objects for
representing persisting associations and arrays of {}\texttt{Ptr}
objects for representing non-persisting associations when dealing with
reference-type objects.


%
{}\subsubsection{Idioms for class object data members and local
variables}
%

In general, class object data members and local varaibles represent a
presisting relationship and therefore should have unique ownership or
use reference counting.  That means that the types {}\texttt{Ptr} and
{}\texttt{ArrayView} should almost never be used for class object data
members or local variables (especially not for data members).
However, local varibles of type {}\texttt{Ptr} and
{}\texttt{ArrayView} will be created in a function when that are
created off of other {}\texttt{Ptr} and {}\texttt{ArrayView} objects
(passed through the formal argument list).

\begin{table}
%
%\fbox{
\begin{center}
%
%\begin{minipage}{{}\textwidth}
%
\begin{tabular}{|l|l|}
%
\multicolumn{2}{c}{\textbf{Class Data Members for Value-Type Objects}} \\
%
\hline
Data member purpose
& Data member delcaration \\
\hline
\hline
non-shared, single, const object
& {}\texttt{const S s\_;} \\
\hline
non-shared, single, non-const object
& {}\texttt{S s\_;} \\
\hline
non-shared array of non-const objects
& {}\texttt{Array<S> as\_;} \\
\hline
shared array of non-const objects
& {}\texttt{RCP<Array<S> > as\_;} \\
\hline
non-shared statically sized array of non-const objects
& {}\texttt{Tuple<S,N> as\_;} \\
\hline
shared statically sized array of non-const objects
& {}\texttt{RCP<Tuple<S,N> > as\_;} \\
\hline
shared fixed-sized array of const objects
& {}\texttt{ArrayRCP<const S> as\_;} \\
\hline
shared fixed-sized array of non-const objects
& {}\texttt{ArrayRCP<S> as\_;} \\
\hline
\end{tabular}

%
%\end{minipage}
%
\end{center}
%} % end fbox
\caption{\label{fig:data_member_value_type}
Idioms for class data member delcarations for value-type objects.}
%
\end{table}

\begin{table}
%
%\fbox{
\begin{center}
%
%\begin{minipage}{{}\textwidth}
%
\begin{tabular}{|l|l|}
%
\multicolumn{2}{c}{\textbf{Class Data Members for Reference-Type Objects}} \\
%
\hline
Data member purpose
& Data member delcaration \\
\hline
\hline
non-shared or shared, single, const object
& {}\texttt{RCP<const A> a\_;} \\
\hline
non-shared or shared, single, non-const object
& {}\texttt{RCP<A> a\_;} \\
\hline
non-shared array of shared const objects
& {}\texttt{Array<RCP<const A> > aa\_;} \\
\hline
non-shared array of shared non-const objects
& {}\texttt{Array<RCP<A> > aa\_;} \\
\hline
shared fixed-sized array of shared const objects
& {}\texttt{ArrayRCP<RCP<const A> > aa\_;} \\
\hline
``...'' (const ptr)
& {}\texttt{ArrayRCP<const RCP<const A> > aa\_;} \\
\hline
shared fixed-sized array of shared non-const objects
& {}\texttt{ArrayRCP<RCP<const A> > aa\_;} \\
\hline
``...'' (const ptr)
& {}\texttt{ArrayRCP<const RCP<const A> > aa\_;} \\
\hline
\end{tabular}

%
%\end{minipage}
%
\end{center}
%} % end fbox
\caption{\label{fig:data_member_reference_type}
Idioms for class data member delcarations for reference-types
objects.}
%
\end{table}


Tables {}\ref{fig:data_member_value_type} and
{}\ref{fig:data_member_reference_type} gives idioms for class object
data members.  Usages for local varaibles are similar.  Table
{}\ref{fig:data_member_value_type} shows a few use cases involving
value-type objects.  Table {}\ref{fig:data_member_reference_type}
shows use cases involving reference-type objects.  Every possible use
case is not shown in thess tables, only the most common ones.  There
is almost no end to the number of different types of data structures
that can be created by embedding these memory management types in each
other.  When creating these composite data structures one just needs
to understand the implications for the selections of the class type
and the use of const.

It is important to note that we do not show an {}\texttt{RCP<S>} data
member for a value-type object in Table
{}\ref{fig:data_member_value_type}.  That is because once you declare
an {}\texttt{RCP} object pointing to a value-type object, at that
point you are treating the value-type object with reference symatics
so we would consider it to be a reference-type object (which takes you
to Table {}\ref{fig:data_member_reference_type}).

Note that there are a few other important differences between the way
that value-type objects and reference-type objects are handled.  The
main difference, obviously, is that you can hold a value-type object
by value but not for a reference-type object.  We see this in how
single objects are stored and how arrays of objects are delcared.


%
{}\subsubsection{Idioms for the specification of formal arguments for
C++ functions}
%



\begin{table}[p]
%
%\fbox{
\begin{center}
%
%\begin{minipage}{{}\textwidth}
%
\begin{tabular}{|l|l|}
%
\multicolumn{2}{c}{\textbf{Passing Value Objects, Non-Persisting Associations}} \\
%
\hline
Argument Purpose
& Formal Argument Declaration \\
\hline
\hline
single, non-changeable object (required)
& {}\texttt{S s} $\;\;\;\;$\textbf{or}$\;\;\;\;$ {}\texttt{const S s} $\;\;\;\;$\textbf{or}$\;\;\;\;$ {}\texttt{const S \&s} \\
%& \begin{minipage}{12ex}
%    {\tiny .}\\
%    {}\texttt{S s} \\
%    or \\
%    {}\texttt{const S s} \\
%    or \\
%    {}\texttt{const S \&s}
%    {\tiny .}
%  \end{minipage} \\
\hline
single, non-changeable object (optional)
& {}\texttt{const Ptr<const S> \&s} \\
\hline
single, changeable object (required)
& {}\texttt{const Ptr<S> \&s}  $\;\;\;\;$\textbf{or}$\;\;\;\;$ {}\texttt{S \&s} \\
%& \begin{minipage}{20ex}
%    {\tiny .}\\
%    {}\texttt{const Ptr<S> \&s} \\
%    or \\
%    {}\texttt{S \&s} \\
%    {\tiny .}
%  \end{minipage} \\
\hline
single, changeable object (optional)
& \texttt{const Ptr<S> \&s} \\
\hline
array of non-changeable objects
& {}\texttt{const ArrayView<const S> \&s} \\
\hline
array of changeable objects
& {}\texttt{const ArrayView<S> \&s} \\
\hline
%
\end{tabular} \\[3ex]
%
\begin{tabular}{|l|l|}
%
\multicolumn{2}{c}{\textbf{Passing Value Objects, Persisting Associations}} \\
%
\hline
Argument Purpose
& Formal Argument Declaration \\
\hline
\hline
array of non-changeable objects
& {}\texttt{const ArrayRCP<const S> \&s} \\
\hline
array of changeable objects
& {}\texttt{const ArrayRCP<S> \&s} \\
\hline
\end{tabular}

%
%
%\end{minipage}
%
\end{center}
%} % end fbox
\caption{\label{fig:func_args_value_type}
Idioms for passing value-type objects to C++ functions.}
%
\end{table}


\begin{table}[p]
%
%\fbox{
\begin{center}
%
\begin{tabular}{|l|l|}
%
\multicolumn{2}{c}{\textbf{Passing IN Non-Persisting Associations to Reference (or Value) Objects as Func Args}} \\
%
\hline
Argument Purpose
& Formal Argument Declaration \\
\hline
\hline
single, non-changeable object (required)
& {}\texttt{const A \&a} \\
\hline
single, non-changeable object (optional)
& {}\texttt{const Ptr<const A> \&a} \\
\hline
single, changeable object (required)
& {}\texttt{const Ptr<A> \&a}  $\;\;\;\;$\textbf{or}$\;\;\;\;$ {}\texttt{A \&a} \\
%& \begin{minipage}{20ex}
%    {\tiny .}\\
%    {}\texttt{const Ptr<A> \&a} \\
%    or \\
%    {}\texttt{A \&a} \\
%    {\tiny .}
%  \end{minipage} \\
\hline
single, changeable object (optional)
& \texttt{const Ptr<A> \&a} \\
\hline
array of non-changeable objects
& {}\texttt{const ArrayView<const Ptr<const A> > \&aa} \\
\hline
array of changeable objects
& {}\texttt{const ArrayView<const Ptr<A> > \&aa} \\
\hline
\end{tabular} \\[3ex]
%
\begin{tabular}{|l|l|}
%
\multicolumn{2}{c}{\textbf{Passing IN Persisting Associations to Reference (or Value) Objects as Func Args}} \\
%
\hline
Argument Purpose & Formal Argument Declaration \\
\hline
\hline
single, non-changeable object
& {}\texttt{const RCP<const A> \&a} \\
\hline
single, changeable object
& {}\texttt{const RCP<A> \&a} \\
\hline
array of non-changeable objects
& {}\texttt{const ArrayView<const RCP<const A> > \&aa} \\
\hline
array of changeable objects
& {}\texttt{const ArrayView<const RCP<A> > \&aa} \\
\hline
\end{tabular} \\[3ex]
%
\begin{tabular}{|l|l|}
%
\multicolumn{2}{c}{\textbf{Passing OUT Persisting Associations for Reference (or Value) Objects as Func Args}} \\
%
\hline
Argument Purpose
& Formal Argument Declaration \\
\hline
\hline
single, non-changeable object
& {}\texttt{const Ptr<RCP<const A> > \&a} \\
\hline
single, changeable object
& {}\texttt{const Ptr<RCP<A> > \&a} \\
\hline
array of non-changeable objects
& {}\texttt{const ArrayView<RCP<const A> > \&aa} \\
\hline
array of changeable objects
& {}\texttt{const ArrayView<RCP<A> > \&aa} \\
\hline
\end{tabular} \\[3ex]
%
\begin{tabular}{|l|l|}
%
\multicolumn{2}{c}{\textbf{Passing OUT Semi-Persisting Associations for Reference (or Value) Objects as Func Args}} \\
%
\hline
Argument Purpose
& Formal Argument Declaration \\
\hline
\hline
single, non-changeable object
& {}\texttt{const Ptr<Ptr<const A> > \&a} \\
\hline
single, changeable object
& {}\texttt{const Ptr<Ptr<A> > \&a} \\
\hline
array of non-changeable objects
& {}\texttt{const ArrayView<Ptr<const A> > \&aa} \\
\hline
array of changeable objects
& {}\texttt{const ArrayView<Ptr<A> > \&aa} \\
\hline
\end{tabular}

\end{center}
\caption{\label{fig:func_args_ref_type}
Idioms for passing reference-type objects to C++ functions.}
%} % end fbox
\end{table}


Here we describe a convention for the specification of the formal
arguments for C++ functions that that maximizes compile-time and
run-time checking, yields near optimal performance, and is highly self
documenting.  A key component to this specification is that no raw C++
pointers are used.  Raw pointers are the cause of almost all memory
usage problems in C++.  Raw C++ references, on the other hand, are
safe to use as long as the object reference they are being used to
point to is valid and no persisting association exists.  All of these
classes perform runtime checks of correct memory usage when debug
checking is enabled at configure time (i.e., using
{}\texttt{--enable-teuchos-debug} with {}\texttt{configure}).
However, in optimized builds these classes perform no debug checking
at all and yield the same performance achieved using raw C++ pointers.
The classes {}\texttt{RCP} and {}\texttt{ArrayRCP} always implement
reference counting in optimized and debug builds and therefore impart
necessary extra overhead that never goes away (unless you don't use
them).  The other classes {}\texttt{Ptr} and {}\texttt{ArrayView} do
not use any reference counting in an optimized build and therefore
impart little extra overhead.

Tables {}\ref{fig:func_args_value_type} and
{}\ref{fig:func_args_ref_type} give conventions for passing single
objects and arrays of objects for value-type {}\index{Types!Value
Type} and reference-type {}\index{Types!Reference Type} objects,
respectively.  In this specification, the classes
{}\texttt{Teuchos\-::Ptr}, {}\texttt{Teuchos\-::RCP},
{}\texttt{Teuchos\-::ArrayRCP}, and {}\texttt{Teuchos\-::ArrayView}
are used as a means to pass objects of another type (shown as
{}\texttt{S} and {}\texttt{A} in Tables
{}\ref{fig:func_args_value_type} and {}\ref{fig:func_args_ref_type}).
This specification addresses the five different properties that must
be considered when passing an object to a function:

\begin{itemize}

{}\item Is it a single object or an array of objects?

{}\item Does the object or array of objects use value semantics or
reference semantics?

{}\item Is the object or array of objects changeable or non-changeable
(i.e.\ const)?

{}\item Is this establishing a persisting or non-persisting association?

{}\item Is the object or array of objects optional or required?

\end{itemize}

The first four of these properties are directly expressed in the C++
code in all cases shown in Tables {}\ref{fig:func_args_value_type} and
{}\ref{fig:func_args_ref_type}.  The specification for whether an
argument or object is required or optional must be documented in the
function's interface specification (i.e.\ in a Doxygen documentation
{}\texttt{param} field).  Here we state that, by default, an argument
will be assumed to be required unless otherwise stated.  The only
exception for this implicit assumption for non-null objects is
{}\texttt{const Ptr<const T>\&} for single, non-changeable,
non-persisting, objects where these always mean that the argument is
optional.  If such an argument is required, it is specified as
{}\texttt{const T\&}.

An array of value objects is passed as contiguous storage through an
{}\texttt{ArrayView<S>} or {}\texttt{ArrayView<const S>} object.  An
array of reference objects, however, cannot be passed in contiguous
storage for the objects themselves and instead must be passed as
contiguous storage of (smart) pointers to the objects using
{}\texttt{ArrayView<const Ptr<const A> >} for non-persisting
associations or {}\texttt{ArrayView<const RCP<const A> >} for
persisting associations.  The {}\texttt{const} can be removed from the
either {}\texttt{Ptr}/{}\texttt{RCP} or {}\texttt{A} depending on what
is allowed to change or not change during the function call.

Note that in the case of {}\texttt{RCP} and {}\texttt{ArrayRCP}
objects, that these can be treated as output objects in their own
right.  For example, passing an {}\texttt{RCP<T>} object into a
function to be set to point to a different {}\texttt{A} object would
be specified in the function prototype as {}\texttt{const Ptr<RCP<T>
>\&} or {}\texttt{RCP<T>\&} depending on preference.  Note that it
never makes sense for {}\texttt{Ptr} or {}\texttt{ArrayView} to be set
in this way since such a function call would almost always be
establishing a persisting association with the embedded objects and
therefore would require (according to the standards established here)
using an {}\texttt{RCP} or {}\texttt{ArrayRCP} object.


%
\subsubsection*{Variations in passing single changeable objects}
%

The only area of contention in this specification is how to handle
arguments for required single changeable objects.  The specification
described here allows either passing them through a smart pointer as
{}\texttt{const Ptr<T>\&} or as a raw non-const object reference as
{}\texttt{T\&}.  In Item 25 in {}\cite{C++CodingStandards05}, the
authors recommend passing a raw non-const object reference
{}\texttt{T\&} for changeable required objects, which seems very
reasonable.  However, other notable authors {}\cite[Section Section
5.5]{stroustrup97} and {}\cite[Section 13.2]{CodeComplete2nd04}
recommend passing a pointer instead, as it provides a visual clue that
the object is being modified in the function call.  Of course, our
specification does not allow raw pointers so we pass a {}\texttt{const
Ptr<T>\&} object instead.  To consider the issues, for example, from
looking at the following function call which (if any) argument(s) is
being modified?

{\small\begin{verbatim}
  someFunction(a, b, c, d);
\end{verbatim}}

To tell for sure, one would have to look at the function prototype:

{\small\begin{verbatim}
  void someFunction(const A& a, const B& b, const C& c, D& d);
\end{verbatim}}

to see that it is the {}\texttt{d} argument is being modified in the
function call.

Now consider the convention that all changeable arguments be passed in
through a pointer as {}\texttt{const Ptr<T>\&}, giving the new
prototype:

{\small\begin{verbatim}
  void someFunction(const A& a, const B& b, const C& c,
    const Ptr<D>& d );
\end{verbatim}}

Now the function call looks like:

{\small\begin{verbatim}
  someFunction(a, b, c, outArg(d));
\end{verbatim}}

where {}\texttt{outArg(...)} is a templated non-member function that
returns a {}\texttt{Ptr<T>} object given a raw reference
{}\texttt{T\&}.  Now the client function call itself is self
documenting.  Also, given that this specification states that all
{}\texttt{Ptr<T>} arguments are required to be non-null by default,
this specifies that passing an argument as {}\texttt{const Ptr<T>\&}
has all of the same meaning by default that passing {}\texttt{T\&}
does.  Of course now we have given up a compile-time check for a
non-null argument for {}\texttt{T\&} with a debug-only runtime check
that {}\texttt{const Ptr<T>\&} is non-null.


%
{}\subsubsection{Idioms for returning objects from C++ functions}
%

Idioms for how objects are returned from C++ functions are also
important in order to achieve C++ code that is efficient, safe (both
compile-time and debug-mode run-time checking), and is as
self-documentaing as possible.  Tables
{}\ref{fig:func_return_value_type} and
{}\ref{fig:func_return_reference_type} give common specifications for
returning single objects and arrays of objects for both value-type and
reference-type objects.  Five different types of properties that must
be defined and considered when returning an object (or array of
objects) from a function are:

\begin{itemize}

{}\item Is it a single object or an array of objects?

{}\item Does the object or array of objects use value semantics or
reference semantics?

{}\item Is the object or array of objects changeable or non-changeable
(i.e.\ const)?

{}\item Is this establishing a persisting or non-persisting
association?

{}\item Is the object or array of objects optional or required?

\end{itemize}


\begin{table}[p]
%
%\fbox{
\begin{center}
%
%\begin{minipage}{{}\textwidth}
%
\begin{tabular}{|l|l|}
%
\multicolumn{2}{c}{\textbf{Returning Non-Persisting Associations to Value Objects}} \\
%
\hline
Purpose
& Return Type Declaration \\
\hline
\hline
Single copied object
& {}\texttt{S} \\
\hline
Single non-changeable object (required)
& {}\texttt{const S\&} \\
\hline
Single non-changeable object (optional)
& {}\texttt{Ptr<const S>} \\
\hline
Single changeable object (required)
& {}\texttt{S\&} \\
\hline
Single changeable object (optional)
& {}\texttt{Ptr<S>} \\
\hline
Array of non-changeable objects
& {}\texttt{ArrayView<const S>} \\
\hline
Array of changeable objects
& {}\texttt{ArrayView<S>} \\
\hline
%
\end{tabular} \\[3ex]
%
\begin{tabular}{|l|l|}
%
\multicolumn{2}{c}{\textbf{Returning Persisting Associations to Value Objects}} \\
%
\hline
Purpose
& Return Type Declaration \\
\hline
\hline
Array of non-changeable objects
& {}\texttt{ArrayRCP<const S>} \\
\hline
Array of changeable objects
& {}\texttt{ArrayRCP<S>} \\
\hline
\end{tabular} \\[3ex]
%
\begin{tabular}{|l|l|}
%
\multicolumn{2}{c}{\textbf{Returning Semi-Persisting Associations to Value Objects}} \\
%
\hline
Purpose
& Return Type Declaration \\
\hline
\hline
Array of non-changeable objects
& {}\texttt{ArrayView<const S>} \\
\hline
Array of changeable objects
& {}\texttt{ArrayView<S>} \\
\hline
\end{tabular}

%
%
%\end{minipage}
%
\end{center}
%} % end fbox
\caption{\label{fig:func_return_value_type}
Idioms for returning value-type objects from C++ functions.}
%
\end{table}


\begin{table}[p]
%
%\fbox{
\begin{center}
%
%\begin{minipage}{{}\textwidth}
%
\begin{tabular}{|l|l|}
%
\multicolumn{2}{c}{\textbf{Returning Reference-Type Objects, Non-Persisting Associations}} \\
%
\hline
Purpose
& Return Type Declaration \\
\hline
\hline
Single cloned object
& {}\texttt{RCP<A>} \\
\hline
Single non-changeable object (required)
& {}\texttt{const A\&} \\
\hline
Single non-changeable object (optional)
& {}\texttt{Ptr<const A>} \\
\hline
Single changeable object (required)
& {}\texttt{A\&} \\
\hline
Single changeable object (optional)
& {}\texttt{Ptr<A>} \\
\hline
Array of non-changeable objects
& {}\texttt{ArrayView<const Ptr<const A> >} \\
\hline
Array of changeable objects
& {}\texttt{ArrayView<const Ptr<A> >} \\
\hline
%
\end{tabular} \\[3ex]
%
\begin{tabular}{|l|l|}
%
\multicolumn{2}{c}{\textbf{Returning Reference-Type Objects, Persisting Associations}} \\
%
\hline
Purpose
& Return Type Declaration \\
\hline
\hline
Single non-changeable object
& {}\texttt{RCP<const A>} \\
\hline
Single changeable object
& {}\texttt{RCP<A>} \\
\hline
Array of non-changeable objects
& {}\texttt{ArrayView<const RCP<const A> >} \\
\hline
Array of changeable objects
& {}\texttt{ArrayView<const RCP<A> >} \\
\hline
\end{tabular}

%
%
%\end{minipage}
%
\end{center}
%} % end fbox
\caption{\label{fig:func_return_reference_type}
Idioms for returning reference-type objects from C++ functions.}
%
\end{table}


These five different properites are the same five described for formal
function arguments.  Again, the first four of these properties are
mostly clearly defined in the C++ code itself.  However, again, it is
not always possible to directly state in the C++ code declarations
whether the object (or array of objects) is optional or required (but
in some cases for value-type objects it is as shown in Table
{}\ref{fig:func_return_value_type}).

The semantics of return objects is different than for formal function
arguments and the specifications described in Tables
{}\ref{fig:func_return_value_type} and
{}\ref{fig:func_return_reference_type}.  There are several differences
that you can see from looking at Tables
{}\ref{fig:func_args_value_type} and {}\ref{fig:func_args_ref_type},
and Tables {}\ref{fig:func_return_value_type} and
{}\ref{fig:func_return_reference_type}.

The first difference that we will discuss between formal functions
arguments and return values relates to using constant references for
formal arguments versus returning raw objects as return values.  While
all of the memory mangement objects of type {}\texttt{Ptr},
{}\texttt{RCP}, {}\texttt{ArrayView}, and {}\texttt{ArrayRCP} are all
passed by constant reference in Tables
{}\ref{fig:func_args_value_type} and {}\ref{fig:func_args_ref_type},
alternatively they are always returned as objects (i.e.\ by value) in
Tables {}\ref{fig:func_return_value_type} and
{}\ref{fig:func_return_reference_type}.  The reason that these memory
management objects should always be returned by value is that this is
needed to correctly set up the reference counting machinary to
properly set up persisting relationships and to enable debug runtime
checking (e.g.\ to detect dangling references).

In order to o understand the importance of returning memory management
objects by value instead of by reference, first consider Listing
{}\ref{listing:unsafe_raw_C++_reference1} that looks to be perfectly
safe code.

\begin{listing}: A seemingly safe use raw C++ references that is not \\
\label{listing:unsafe_raw_C++_reference1}
{\small\begin{verbatim}
  void seeminglySafeFoo(Blob &blob, const Flab &flab)
  {
    blob.doGoodStuff(flab);
  }
\end{verbatim}}
\end{listing}

The code in Listing {}\ref{listing:unsafe_raw_C++_reference1} does not
itself look unsafe.  However, the reason that it unsafe comes from the
code that calls {}\texttt{seeminglySafeFoo(...)} and the code that
implements {}\texttt{Blob} shown in Listing
{}\ref{listing:unsafe_raw_C++_reference2}.

\begin{listing}: Code that makes seeminglySafeFoo(...) fail\\
\label{listing:unsafe_raw_C++_reference2}
{\small\begin{verbatim}
  class Blob
  {
    RCP<Flab> flab_;
  public:
    Blob() : flab_(flab()) {}
    const RCP<Flab>& getFlab() { return flab_; }
    void doGoodStuff(const Flab &flab_in)
    {
      flab_ = flab(); // Using non-member constructor
      flab_->conflab(flab_in);
    }
  };


  void badCallingFunction()
  {
    Blob blob;
    seeminglySafeFoo(blob, *blob.getFlab());
  }
\end{verbatim}}
\end{listing}

When the code in Listings {}\ref{listing:unsafe_raw_C++_reference1}
and {}\ref{listing:unsafe_raw_C++_reference2} executes, it will most
likely cause a segfault when it runs, if we are lucky.  If we are
unlucky, the code will actually seem to be working correctly on the
machine where we test it but will explode later (parhaps years later)
when run under different circumstances.  The reason that the code in
Listings {}\ref{listing:unsafe_raw_C++_reference1} and
{}\ref{listing:unsafe_raw_C++_reference2} is faulty is because the
{}\texttt{Flab} object that is passed through the call
{}\texttt{seeminglySafeFoo(blob, *blob.getFlab())} to
{}\texttt{blob.doGoodStuff(flab)} is invalidated before it is used
because it gets destoryed and is replaced by a new object in the
expression {}\texttt{flab\_ = flab()}.  When this happens, the object
now represented as the raw C++ reference {}\texttt{flab\_in} is
deleted which causes the code in the expression
{}\texttt{flab\_->conflab(flab\_in)} to be in error, and the behavior
of the program is undefined (and again will segfault if we are lucky).

How did we get into this situation?  What if we changed the code to
replace the raw C++ references with RCP-wrapped objects?  Well,
consider the updated code in Listing
{}\ref{listing:unsafe_raw_C++_reference3}.

\begin{listing}: Still unsafe code  \\
\label{listing:unsafe_raw_C++_reference3}
{\small\begin{verbatim}
  class Blob
  {
    RCP<Flab> flab_;
  public:
    Blob() : flab_(flab()) {}
    const RCP<Flab>& getFlab() { return flab_; }
    void doGoodStuff(const RCP<Flab> &flab_in)
    {
      flab_ = flab(); // Using non-member constructor
      flab_->conflab(flab_in);
    }
  };


  void seeminglySafeFoo(Blob &blob, const RCP<Flab> &flab)
  {
    blob.doGoodStuff(flab);
  }


  void badCallingFunction()
  {
    Blob blob;
    seeminglySafeFoo(blob, blob.getFlab());
  }
\end{verbatim}}
\end{listing}

Is the code in Listing {}\ref{listing:unsafe_raw_C++_reference3}
correct?  The sad answer is no, it is not.  The {}\texttt{Flab} object
returned from {}\texttt{blob.getFlab()} will still get deleted before
it is used in the expression {}\texttt{flab\_->conflab(flab\_in)}.
What is going on here?  The core of the problem is that the function
{}\texttt{Blob::getFlab()} is incorrectly implemented.  Functions must
always return {}\texttt{RCP} objects by value and never by reference
as shown in Tables {}\ref{fig:func_return_value_type} and
{}\ref{fig:func_return_reference_type}.  By returning a raw C++
reference to the RCP object, a peristing association with the client
is never properly established and this is the root casue the whole
problem.

Now consider the updated code in Listing
{}\ref{listing:safe_raw_C++_reference3} that goes back to using raw
C++ references where appropriate but now returns the
{}\texttt{RCP<Flab>} object by value as it should.

\begin{listing}: Correctly returning RCP by value yielding safe code \\
\label{listing:safe_raw_C++_reference3}
{\small\begin{verbatim}
  class Blob
  {
    RCP<Flab> flab_;
  public:
    Blob() : flab_(flab()) {}
    RCP<Flab> getFlab() { return flab_; } // Returns by value now!
    void doGoodStuff(const RCP<Flab> &flab_in)
    {
      flab_ = flab(); // Using non-member constructor
      flab_->conflab(flab_in);
    }
  };


  void seeminglySafeFoo(Blob &blob, const Flab &flab)
  {
    blob.doGoodStuff(flab);
  }


  void goodCallingFunction()
  {
    Blob blob;
    seeminglySafeFoo(blob, *blob.getFlab());
  }
\end{verbatim}}
\end{listing}

Is the code represented in Listing
{}\ref{listing:safe_raw_C++_reference3} now safe and correct?  Yes it
is.  The reason that it is now safe and correct is that a persisting
relationship is now being correctly created by the function call
{}\texttt{blob.getFlab()} in that a new temporary {}\texttt{RCP<Flab>}
object is created.  From this new temporary {}\texttt{RCP<Flab>}
object, a raw C++ reference is then returned from
{}\texttt{*blob.getFlab()} and passed through.  In this case, since
the reference count on the existing {}\texttt{Flab} object is now two
instead of one, the expression {}\texttt{flab\_ = flab()} will not
delete the existing {}\texttt{Flab} object and the following
expression {}\texttt{flab\_->conflab(flab\_in)} will have two valid
{}\texttt{Flab} objects.  After the function
{}\texttt{seeminglySafeFoo(blob, *blob.getFlab())} exits, the first
{}\texttt{Flab} object will finally be deleted but that is just fine.

The second difference between formal function arguments and return
values is what persisting and non-persisting associations mean related
to function returns.  In the case of objects returned from C++
functions, a persisting association is one where the object returned
from a C++ function is remembered past the end of the statement where
the C++ function returning the objects is called.  For example,
consider the code in Listing {}\ref{listing:persisting-func-return-1}.

\begin{listing}:\\
\label{listing:persisting-func-return-1}
{\small\begin{verbatim}
  void foo(Blob& blob)
  {
    const Flab &flab = blob.getFlab();
    blob.doStuff();
    flab.doMoreStuff();
  }
\end{verbatim}}
\end{listing}

The code in Listing {}\ref{listing:persisting-func-return-1}
represents a persisting association because because the
{}\texttt{Flab} object returned in the expression {}\texttt{const Flab
\&flab = blob.getFlab()} is remembered past the statement where it is
called and is used later in calling {}\texttt{flab.doMoreStuff()}.
This type of code is all too common in C++ programs (including a lot
of code I have written over the last 10 years) but it is not safe
because it is not properly respecting the notion of persisting
assocations.  To see why the code in Listing
{}\ref{listing:persisting-func-return-1} is so bad, consider a
possible unfortunate implementation of the {}\texttt{Blob} class shown
in Listing {}\ref{listing:bad-Blob-non-persisting}:

\begin{listing}:\\
\label{listing:bad-Blob-non-persisting}
{\small\begin{verbatim}
  class Blob {
    RCP<Flab> flab_;
  public:
    Blob() : flab_(flab()) {}
    const Flab& getFlab() const { return *flab_; }
    void doStuff()
      {
        flab_ = flab(); // Non-member constructor
        ...
      }
  };
\end{verbatim}}
\end{listing}

What happens of course is that the behavior of the code in Listings
{}\ref{listing:persisting-func-return-1} and
{}\ref{listing:bad-Blob-non-persisting} is undefined and will most
likely result in a segfault (if we are lucky).  The reason this is bad
code is that the {}\texttt{Flab} object reference that gets returned
from {}\texttt{blob.getFlab()} is not used until after the function
{}\texttt{Blob::doStuff()} gets called which will delete the
{}\texttt{Flab} object and replace it with another one.  This results
in {}\texttt{flab.doMoreStuff()} being called on a deleted object.
Again, this will typically result in a segfault, but on some systems
in some cases the program might actually seem to run just fine,
perhaps even for years.  This of course is an error that a tool like
Valgrind or Purify would likely catch pretty easily which is why these
tools are very useful to have around.  So what rule did we break in
Listng {}\ref{listing:persisting-func-return-1}?  This now brings us
back to our defintion of a persisting association related to a return
value which is:

\begin{itemize}

{}\item\textit{Persisting associations} are associations that exist
between two or more objects that extend past a single function call
for formal function arguments, or a single statement for function
return objects.

\end{itemize}

What this means is that any object that is returned as a raw C++
reference from a function must be used in the same statement from
where the returning function is called.  Therefore, the function in
Listing {}\ref{listing:persisting-func-return-1} should be rewritten
as shown in Listing {}\ref{listing:non-persisting-func-return-1}.

\begin{listing}:\\
\label{listing:non-persisting-func-return-1}
{\small\begin{verbatim}
  void foo(Blob& blob)
  {
    blob.getFlab().doMoreStuff();
    blob.doStuff();
  }
\end{verbatim}}
\end{listing}

Here, of course, we are assuming that the order of evaluation of the
functions is not important.

Note that functions returning raw C++ references are common and are
faily safe to use as long as the returned object is used in the same
statment where the function is called.  For example, this is what is
commonly done when a non-const reference to an element from a
user-defined array class object is returned and set in the same
statements such as shown in Listing
{}\ref{listing:non-persisting-array-return-1}.

\begin{listing}:\\
\label{listing:non-persisting-array-return-1}
{\small\begin{verbatim}
  void foo(std::vector<int>& a)
  {
    a[0] = 5; // Non-persisting function return association
    ...
  }
\end{verbatim}}
\end{listing}

What is typically not safe, of course, is when you try to save a
reference to an object and then use it like in Listing
{}\ref{listing:bad-persisting-array-return-1}.

\begin{listing}:\\
\label{listing:bad-persisting-array-return-1}
{\small\begin{verbatim}
  void foo(std::vector<int>& a)
  {
    int &a_0 = a[0]; // Incorrect persisting association
    a.resize(20);
    a_0 = 5;         // Will likely segfault if we are lucky!
    ...
  }
\end{verbatim}}
\end{listing}

The problem with the code in Listing
{}\ref{listing:bad-persisting-array-return-1} is that the
{}\texttt{a.resize(20)} function might cause a new buffer to be
allocated and the existing buffer to be deleted.  This will of course
make the reference returned in {}\texttt{int \&a\_0 = a[0]} invalid
when it is later written to in {}\texttt{a\_0 = 5}.

The whole point of the example code Listings
{}\ref{listing:non-persisting-array-return-1} and
{}\ref{listing:bad-persisting-array-return-1} is to demonstrate the
working definition of persisting \& non-persisting associations as
they relate to objects returned from functions and support the idioms
shown in Tables {}\ref{fig:func_return_value_type} and
{}\ref{fig:func_return_reference_type}.


%
{}\subsection{Reference-counting machinery in-depth}
%

In order to effectively use these memory management classes and to
debug problems when they occur, one must understand the basic
reference-counting approach being used.  Basic reference counting with
smart pointers is well established in the C++ literature [???] but a
basic overview and specific details about the approach used in the
Teuchos memory management classes is appropriate to describe here.  Of
equal importance is to describe how the reference-counting
infrastructure can be used to address some boundary cases that can be
used to help solve some fundamental problems with reference counting.

The basic reference counting machinery being used by the classes is
first described.  Next, the issue of circular references and weak
pointers are described.  The last issue to discuss is the general
roles and responsibilities for the various actors in software using
reference counting.


%
{}\subsubsection{Basic reference counting machinery}
%

The foundation for the reference-counting machinery used by all of the
reference counting classes is shown in Figure ???.  The class
{}\texttt{RCPNode} is an abstract base class that contains the strong
and weak reference counts themselves and inline functions for
manipulating the counts as efficiently as possible.  The templated
concrete subclass {}\texttt{RCPNodeRef} is what actually stores the
raw C++ pointer to the reference-counted object.  This class is also
templated on an deallocation policy object that determines how the
object is reclaimed.  The class {}\texttt{RCPNodeHandle} is a simple
handle class that automates the manipulation of the reference counts
by overloading the copy constructor and assignment operator functions.
This avoids having to replicate reference counting updating and
downdating in the user-level classes that contain it.  All of the
functions on {}\texttt{RCPNodeHandle} are inlined and the only data
member is a pointer to the underlying {}\texttt{RCPNode} object.
Therefore, the class {}\texttt{RCPNodeHandle} imparts zero space and
time overhead and removes all duplication in how the reference count
node object is handled.  Note that the classes {}\texttt{RCPNode},
{}\texttt{RCPNodeImpl}, and {}\texttt{RCPNodeHandle}, are used for
both {}\texttt{RCP} and {}\texttt{ArrayRCP} classes.

ToDo: Put in UML class diagram figure showing basic reference counting
classes along with the ArrayRCP and RCP classes.

It is critical to understand that the foundation for sharing objects
using reference counting is that only one {}\texttt{RCPNode} object
can exist for any object that is shared.  Figure ??? shows an example
of several {}\texttt{RCP} objects pointing to the same concrete object
of type {}\texttt{ConcreteT} through different base class interfaces.
If the programmer follows the idioms described in Section ??? and
outlined in Appendix ???, it will always be the case that only one
reference-counting node object will exist.

ToDo: Put in a UML object diagram showing several RCPNodeHandle
objects pointing to the same RCPNodeImpl and therefore ConcreteT
object with interfaces BaseT1, BaseT2 �

However, consider what happens when one or more of the commandments
are broken and more than one node object exists pointing to the same
underlying {}\texttt{ConcreteT} object as shown if Figure ???.  The
problem is that the two {}\texttt{RCPNodeImpl} objects do not know
about each other and the first one who has its strong reference count
go to zero will result in the underlying {}\texttt{ConcreteT} object
being deleted.  In this case, the other remaining
{}\texttt{RCPNodeImpl} object, and all of the resulting
{}\texttt{ArrayRCP} and/or {}\texttt{RCP} objects pointing to it will
be left with a non-null pointer to a now deleted {}\texttt{ConcreteT}
object.  If the client tries to access the underlying object through
one of these now invalid references, it will likely result in a
segfault (if we are lucky).  Or, if we are unlucky, the program might
seem to work okay on the platform we are currently testing on but will
eventually explode at the worst possible time. Putting in the
machinery to try to track multiple {}\texttt{RCPNode} objects for the
same underlying {}\texttt{ConcreteT} object would be fairly expensive
in space and/or time overhead and therefore is not done in a release
build.  However, in a debug build, you can enable runtime checks for
just this situation as described in Section ???.  Therefore, if you
are willing to pay for a little extra overhead, these classes will
detect this type of invalid usage.

ToDo: Put in UML object diagram showing two {}\texttt{RCPNodeImpl}
objects pointing to the same {}\texttt{ConcreteT} objects which will
result in a double delete and/or accessing a deleted object after it
is destoried, whatever comes first.


%
{}\subsubsection{Circular references and weak pointers}
%

The fundamental weakness of low-overhead reference counting as
described in this paper and used in the Teuchos memory management
classes is that there is no bullet-proof way to address circular
references that result in memory leaks.  Because of possible circular
references, only system-level garbage collection methods, as
implemented in Java and Python can robustly clean up memory in every
case of circular reference.  As stated earlier, given backward
compatibility constraints, many existing C++ programs cannot be used
with any C++ implementation that might implement garbage collection,
not now or ever.

To understand the problem with circular references, consider Figure
??? showing the circular reference between two objects {}\texttt{a}
and {}\texttt{b}.  Here object {}\texttt{a} contains an {}\texttt{RCP}
pointing to object {}\texttt{b}, and object {}\texttt{b} contains an
{}\texttt{RCP} pointing to object {}\texttt{a}.  In this situation,
when {}\texttt{ClientA} and {}\texttt{ClientB} destroy their
{}\texttt{RCP} objects pointing to the underlying {}\texttt{a} and
{}\texttt{b} objects, the reference counts will not go to zero because
of the circular reference between {}\texttt{a} and {}\texttt{b}.  This
will result in a memory leak that a tool like Valgrind or Purify
should contain about.  If lots of objects with circular references are
constantly being created and destroyed creating these types of memory
leaks, obviously we have a problem and the system could run out of
memory.

ToDo: Show a UML object diagram showing a classical circular reference
between two objects A and B.

While there is no completely general and bullet proof way to address
the circular reference problem, there is a fairly simple and cheap
approach that can effectively resolve circular references in most
cases.  The different approaches described here supported by these
reference counted classes is the definition of a weak
reference-counted pointer.  As shown in Figure ???, this is
accomplished through a {}\texttt{strength\_} attribute with values
{}\texttt{STRONG} and {}\texttt{WEAK}.  By default, all
{}\texttt{[Array]RCP} objects are {}\texttt{STRONG}.  When an
{}\texttt{[Array]RCP} is {}\texttt{STRONG}, then the underlying
{}\texttt{ConcreteT} object is guaranteed to stay around.  However,
when the {}\texttt{[Array]RCP} is {}\texttt{WEAK}, the underlying
{}\texttt{ConcreteT} can get deleted when strong count goes to zero.

So how does this work?  Figure ??? shows the same setup with a
circular references as shown in Figure ??? �

ToDo: Copy the material from the slides showing how the a weak pointer
resolves the circular reference problem!

With respect the weak references, the Teuchos classes
{}\texttt{[Array]RCP} differ substantially from the Boost and
therefore C++09 standard reference counting classes.  With the classes
{}\texttt{[Array]RCP}, the attribution of strong work weak is made at
runtime.  At runtime, an external client can decide to make
{}\texttt{a}'s reference to {}\texttt{b} weak or {}\texttt{b}'s
reference to {}\texttt{a} weak depending on the given circumstance.
With the Boost and C++09 {}\texttt{shared\_ptr} class, you have to use
a separate class {}\texttt{weak\_ptr} to represent a weak pointer.
The problem with the Boost approach then is that you have to decide at
compile time if a particular reference is going to be weak or strong.
While there are some cases where you can always assume the reference
needs be weak (like in the self reference case described in Section
???), there are more complex cases where you cannot so easily.  For
example, if we were to use the {}\texttt{[shared\_ptr} and
{}\texttt{weak\_ptr} classes, we would have to decide at compile time
to make {}\texttt{a}'s reference or {}\texttt{b}'s reference weak.
The decision you make might work for one set of use cases that you
currently know about, but for more complex usage cases you have not
discovered yet you may need to switch it.  In fact, in the same
program you may have some cases where {}\texttt{a} will be deleted
before {}\texttt{b} and other cases where {}\texttt{b} will be deleted
before {}\texttt{a}.  With the classes {}\texttt{[shared\_ptr} and
{}\texttt{weak\_ptr}, this is impossible (at least not without storing
both smart pointer types in each class {}\texttt{A} and {}\texttt{B}
object and then using one or the other which is not very elegant or
efficient).  The only argument for the compile-time approach uses by
Boost and C++09 is performance and a little extra speed and memory
overhead but the results in Section ??? show that this extra runtime
overhead is minimal.  Overall, the overhead induced by the flexible
runtime approach of the {}\texttt{[Array]RCP} classes is well worth
the minimal extra overhead.  Typically, the classes
{}\texttt{[Array]RCP} are used to manage objects much larger than what
is contained in the infrastructure for the {}\texttt{[Array]RCP}
objects so the additional memory is also insignificant.

Section ??? describes how weak pointers are used in runtime debug
checking for dangling non-persisting references.  Section ??? 
describes how weak pointers are used in dealing with object self
references.  In summary, the this reference counting machinery
addressing both strong and weak references is a very powerful tool but
to use it effectively, you need to understand the basic semantics.
The good news is that 99\% of the people who use the classes
{}\texttt{[Array]RCP} will never need to know the difference between a
strong and weak reference and will by default use just strong
references.  Weak references will get used under the hood in debug
builds of the code but they are totally transparent to client code and
the programmer.


%
{}\subsubsection{Customized deallocators}
\label{sec:deallocators}
%

The most common use of {}\texttt{RCP} is to manage the lifetime of
objects allocated using operator {}\texttt{new} and deallocated using
operator {}\texttt{delete} (or {}\texttt{new []} and {}\texttt{delete
[]} for {}\texttt{ArrayRCP}).  For these use cases, the built-in
behavior in {}\texttt{RCP} does exactly the right thing for this
average case.  However, there are situations when one cannot simply
call {}\texttt{delete} to deallocate an object.  Some examples of
these situations include:

\begin{enumerate}

{}\item When reference counts for objects are managed by requiring
clients to explicitly call increment and decrement functions.  This
situation occurs when using CORBA {}\cite{ref:corba} and COM
{}\cite{ref:com} for instance.  Such an approach is also presented in
{}\cite[Item 29]{ref:meyers_1996} in the subsection ``A
Reference-Counting Base Class''.  In these protocols, deallocation
occurs automatically behind the scenes when this other reference count
goes to zero and does not occur through an explicit call to operator
{}\texttt{delete} as with the default behavior for {}\texttt{RCP}.

{}\item When objects are managed by certain types of object databases.
In some object databases, an object that is grabbed from the database
must be explicitly returned to the database in order to allow proper
object deletion to take place later.

{}\item A different reference-counted pointer class is used to
initially get access to the managed object.  For example, suppose some
piece of peer software works with {}\texttt{boost::shared\_ptr} (see
{}\cite{ref:boost}) referenced-counted objects while the resident
software works with {}\texttt{RCP} objects.  It then becomes critical
no object is deleted until all the clients using either of these smart
pointer types remove their references (i.e.\ destroy their smart
pointers) to this underlying object.

\end{enumerate}

There are also other situations where one cannot simply assume that
calling operator {}\texttt{delete} is used to release an object.  The
bottom line is that in order to be general, one must allow arbitrary
policies to be used to deallocate an object after clients are finished
using the object.

Perhaps the key differentating property between a flexible high
quality reference-counted pointer implementation and a poor
implementation is the capability to allow the user to define an
arbitrary deallocator policy that defines exactly what it means to
release a reference-counted object (or array of objects).  The
reference-counted Teuchos classes {}\texttt{RCP} and
{}\texttt{ArrayRCP}, as well as {}\texttt{boost::shared\_ptr} all
allow the client to specify a user-defined deallocation policy object
when the first reference-counted object is constructed.

The code associated with customized deallocation policies for
{}\texttt{RCP} (what are identical for {}\texttt{ArrayRCP}) are shown
in Listing {}\ref{listing:RCP-dealloc}.

\begin{listing}: Decalarations for customized deallocation policies for {}\texttt{RCP} \\
\label{listing:RCP-dealloc}
{\small\begin{verbatim}
  // Default decallocation policy for RCP
  template<class T>
  class DeallocDelete
  {
  public:
    typedef T ptr_t;
    void free( T* ptr ) { if(ptr) delete ptr; }
  };

  template<class T>
  class RCP {
  public:
    ...
    template<class Dealloc_T>
    RCP(T* p, Dealloc_T dealloc, bool has_ownership);
    ...
  };
  
  template<class T, class Dealloc_T>
  RCP<T> rcp(T* p, Dealloc_T dealloc, bool owns_mem);
  
  template<class Dealloc_T, class T>
  const Dealloc_T& get_dealloc(const RCP<T>& p);
  
  template<class Dealloc_T, class T>
  Dealloc_T& get_nonconst_dealloc(const RCP<T>& p);
  
  template<class Dealloc_T, class T>
  Ptr<const Dealloc_T> get_optional_dealloc(const RCP<T>& p);
  
  template<class Dealloc_T, class T> 
  Ptr<Dealloc_T> get_optional_nonconst_dealloc(const RCP<T>& p);

\end{verbatim}}
\end{listing}

All deallocator objects must support the typedef member
{}\texttt{ptr\_t} and function member {}\texttt{free()}.  The concept
of a template policy interface (also called a function object
\cite[Section 18.4]{ref:stroustrup_2000}) should be familiar to
semi-advanced users of the STL (part of the standard C++ library).

To demonstrate the use of a deallocator object, let us assume that we
must wrap objects of type {}\texttt{A} managed by the the following
object database

{\small\begin{verbatim}
  class ObjectADB {
    ...
    A& get(int id);
    void release(int id);
    ...
  };
\end{verbatim}}

In the above object database, objects are accessed and released using
an id.  How this id is specified and determined is not important here.
Let us suppose that we want to define an abstract factory that returns
objects of type {}\texttt{A} wrapped in {}\texttt{RCP<A>} objects
using a database object of type {}\texttt{ObjectADB} shown above.  For
this abstract factory, objects of type {}\texttt{A} will be allocated
from a list of ids given to the factory.  The outline of this abstract
factory subclass looks like:

{\small\begin{verbatim}
  class ObjectADBFactory : public AbstractFactory<A> {
  public:
    ObjectADBFactory(const RCP<ObjectADB>& db, const ArrayView<int>& ids)
     : db_(db), ids_(ids) {}
    RCP<A> create() { ... };  // Overridden from AbstractFactory
  private:
    RCP<ObjectjADB> db_;
    Array<int> ids_;
  };
\end{verbatim}}

The above abstract factory subclass {}\texttt{ObjectADBFactory}
inherits from a generic {}\texttt{AbstractFactory} base class that
defines a pure virtual method {}\texttt{create()}.  In order to
implement the {}\texttt{create()} function, a deallocator class must
be defined and used.  For this purpose we define:

{\small\begin{verbatim}
  class DeallocObjectADB
  {
  public:
    DeallocObjectADB(const RCP<ObjectjADB>& db, int id)
      : db_(db), id_(id) {}
    typedef A ptr_t;
    void free(A* ptr) { db_->release(id_); }
  private:
    RCP<ObjectjADB> db_;
    int id_;
    DeallocObjectADB(); // not defined and not to be called!
  };
\end{verbatim}}

Now we can define the implementation of the {}\texttt{create()}
function override as:

{\small\begin{verbatim}
  RCP<A> ObjectADBFactory::create()
  {
    TEST_FOR_EXCEPTION(ids_.size()==0, std::runtime_error, "No ids are left!");
    const int id = ids_.pop();
    return rcp(&db_->get(id), DeallocObjectADB(db_, id), true);
  }
\end{verbatim}}

The following program shows the use of the above factory class:

{\small\begin{verbatim}
  int main()
  {
    // Create the object database and populate it (and save the ids)
    RCP<ObjectADB> db;
    Array<int> ids;
    ...
    // Create the abstract factory object
    ObjectADBFactory  ftcy(db, ids());
    // Create some A objects and use them
    RCP<A> a_ptr1 = fcty.create();
    ...
    return 0;
  }
\end{verbatim}}

In the above program, all of the objects of type {}\texttt{A} are
created and removed seamlessly without the client code that interacts
with {}\texttt{RCP} and {}\texttt{AbstractFactory} knowing anything
about what is going on under the hood.


%
{}\subsubsection{Embedded objects}
\label{sec:embedded-objecs}
%

Support for customized deallocator policy objects described in Section
{}\ref{sec:deallocators} turns out to be a pretty flexible feature.
The ability to embedd any arbitrary object in the {}\texttt{RCPNode}
object gives you an efficient way to define a different deallocation
policy that is invoked by the destructor on the object instead of
requiring an explicit deallocation policy object.  In addition, you
can also tack on any extra data that you would like and embedd in the
underlying {}\texttt{RCPNodeImpl} object.  The only restriction is
that you have to make the choice of what you want to embedd in the
{}\texttt{RCPNode} object when the very first {}\texttt{RCP} object is
created which in turn creates the concrete {}\texttt{RCPNode} object.
If you want the flexibility to embedd other data in the underlying
{}\texttt{RCPNode} object after it has been created then you will have
to use the ``extra data'' feature which is described in Section
{}\ref{sec:extra-data}.  The advantage of embedding objects in the
{}\texttt{RCPNodeImpl} object is that it can be quite a bit more
efficient than using the ``extra data'' feature which requires more
runtime-support.

ToDo: Fill in!


%
{}\subsubsection{Extra data}
\label{sec:extra-data}
%

ToDo: Fill in.


%
{}\subsubsection{General roles and responsibilities: Factories and
general clients}
%

There are two fundamentally different sets of actors that play two
different roles in the use of the reference counted classes: a)
factory entities that first create the reference counted object, and
b) general clients that accept and use a shared reference counted
object through a passed-in reference-counted class.

Factory entities first create the reference-counted object/array
and/or construct the first {}\texttt{[Array]RCP object} containing it.
The most basic type of factory are the non-member constructor
functions described in Section ???.  When the first
{}\texttt{[Array]RCP object} is created, the factory gets to decide
exactly how object/array will be released when the last reference
count goes to zero.  The default, of course, is to just simply call
{}\texttt{delete} or {}\texttt{delete []} on the contained raw pointer
which is the default behavior.  However, the factory can also chooce
any arbitrary action imaginable to occur when the last reference-count
goes to zero.  This is set up using a template deallocator policy
object described in Section {}\ref{sec:deallocators}.

On the other side, the responsibilities of the general client that
shares and uses a reference-counted object are very simple.  These
responsibilities are:

\begin{itemize}

{}\item Accept the persisting relationship for a shared
reference-counted object/array through an {}\texttt{[Array]RCP} object
(as described in Section ???).

{}\item Share the reference-counted object/array with other clients by
creating a copy if your {}\texttt{[Array]RCP} object and giving it to
them.

{}\item When you are finished using the object, simply delete or set
to null all of your {}\texttt{[Array]RCP} objects.  If some other
client is still using the object, it will still remain.  If yours is
the last reference, then the deallocator policy object that is
embedded in the underlying {}\texttt{RCPNodeImpl} object will know
exactly how to clean up and reclaim the object/array.

\end{itemize}

That is all there is to it.  Factories create the underlying object(s)
wrapped in the first {}\texttt{[Array]RCP} object and define how the
referenced object(s) will be reclaimed when it is time.  General
clients just accept and maintain their references to shared
objects/arrays by accepting and storing {}\texttt{[Array]RCP} object
and then setting them to null when they are finished using the
object(s).


%
{}\subsection{Debug-mode runtime checking}
\label{sec:debug-mode-runtime-checking}
%

The primary reason that these Teuchos memory management classes need
to be developed in tandem and know each other's internal
implementations to some extend is to be able to implement robust and
effective debug-mode runtime testing.  The debug-mode runtime testing
that is built into these classes is surprisingly strong and will catch
nearly every type of programmer error that is possible, as long as raw
C++ pointers are never externally exposed.  The different categories
of debug-mode runtime testing are described below along with what the
typical diagnostic error messages look like.


%
{}\subsubsection{Detection of null dereferences and range checking}
\label{sec:null-dereferences-range-checking}
%

One of the most basic types of debug runtime checking performed by the
Teuchos memory management classes are attempts to dereferences a null
pointer and range checking.

ToDo: Fill this in!


%
{}\subsubsection{Detection of circular references}
%

ToDo: Fill this in!


%
{}\subsubsection{Detection of dangling references}
\label{sec:dangling-references}
%

ToDo: Fill this in!


%
{}\subsubsection{Detection of other invalid usage}
\label{sec:dangling-references}
%

ToDo:

(*) Discuss creating two owning RCP objects that point to the same concrete
object but don't share the same node.


%
{}\subsubsection{Performance of debug-mode checking versus memory
checking tools}
%

ToDo:

(*) Compare O(1) debug-mode checking to valgrind and purify

(*) Compare more expensive debug-mode with node-tracing checking to
valgrind and purify


%
{}\subsubsection{Holes in debug-mode runtime checking}
%

If you follow all of the idioms and rules outlined in this paper, then
the only issue you will have to address that is not 100\% obvious are
circular references.  However, if programmers never made any mistakes,
we would not need debug-mode runtime testing in the first place.
While the level of debug-mode runtime testing implemented in the
Teuchos memory management classes is unmatched, there are a few
different types of programmer errors that debug-mode runtime checking
will not catch.  First, as soon as you expose a raw C++ pointer and
start copying it around, all bets are off.  However, even if you never
expose a C++ pointer, you can still get into trouble.  One unfortunate
case involves the use of raw C++ references.

If C++ references are only used as formal function arguments for C++
functions, you will almost never have a problem.  However, incorrectly
returning an {}\texttt{RCP} object by reference instead of by value,
as is described in Section ???, can result in invalid C++ references.
Also, if you use references like in the below code, you can of course
have dangling raw C++ references.

{\small\begin{verbatim}
  RCP<A> a_ptr = newA();
  A &a = *a_ptr;
  ...
  a->someFunc(); // This object may not be valid any more!
\end{verbatim}}

The above code fragment violates the use of raw C++ references only
for non-persisting associations.  Above, the statement {}\texttt{A \&a
= *a\_ptr} results in the creation of a persisting relationship in that
it extends past the statement where it was created.

In summary, as soon as an object reference is exposed through a raw
C++ pointer or a raw C++ reference, the runtime checking can no longer
detect errors.  Therefore, never expose a raw C++ pointer (except for
the situations described in Section {}\ref{sec:role-of-raw-pointers})
and only expose and use raw C++ references in strictly non-persisting
associations.


%
{}\subsubsection{Exception handling and debugging}
%

ToDo: Fill this in!


%
{}\subsection{Optimized performance}
%

ToDo: Fill this in!


%
{}\subsubsection{Reference counting overhead}
%

ToDo: Fill this in!


%
{}\subsubsection{Array access overhead}
%

ToDo: Fill this in!


%
{}\subsubsection{Array iterator overhead}
%

ToDo: Fill this in!


%
{}\subsection{Related idioms}
%

ToDo: Fill this in!


%
{}\subsubsection{Objects maintaining self references}
%

ToDo: Fill this in!


%
{}\subsubsection{The generalized view idiom}
%

ToDo: Fill this in!


%
{}\subsubsection{Inverting object ownership and avoiding circular references}
\label{sec:inverting-obj-ownership}
%

A rare situration that can occur is when you have an object that
maintains an RCP to another object but you want to expose the second
object and have it remember the first object.  To demonstrate,
consider the two classes in Listing {}\ref{listing:B_owns_A_decl}.

\begin{listing}: Two classes where one maintains an RCP to the other \\
\label{listing:B_owns_A_decl}
{\small\begin{verbatim}
  class A { ... };

  RCP<A> createA(...);

  class B {
  public:
    static RCP<B> create(const RCP<A> &a)
      {return rcp(new B(a)); }
    RCP<A> getA() { return a_; }
    void unsetA() { a_ = null; }
    ...
  private:
    RCP<A> a_;
    B(const RCP<A> &a) a_(a) {}
  };

  RCP<B> createB(const RCP<A> &a)
  {return B::create(a);}
\end{verbatim}}
\end{listing}

The class {}\texttt{A} in Listing {}\ref{listing:B_owns_A_decl} may
involve some complex initialization or it may only be an abstract
interface with multiple subclasses.  In either case, it may make sense
to provide a factory function (or a set of such functions) that
creates and initializes a {}\texttt{B} object for different complex
initializations of {}\texttt{A} objects as shown in Listing
{}\ref{listing:createBFactory}.

\begin{listing}: A factory function that creates a B object wrapping a complex A object \\
\label{listing:createBFactory}
{\small\begin{verbatim}
  RCP<B> createBFactory(...)
  {
     // Complex initialization of A
     RCP<A> a;
     ...
     // Wrapped B
     return createB(a);
  }
\end{verbatim}}
\end{listing}

Up to now, this is pretty standard code.  The client would typically
hold an {}\texttt{RCP<B>} object to the {}\texttt{B} object and would
manage the lifetime of the {}\texttt{A} object implicity wrapped in
the {}\texttt{B} object.

However, now consider a rare use case where a client may only want to
deal directy with the {}\texttt{A} object but still maintain the
{}\texttt{B} object for use later.  There are a few approaches that
one could try to implement this inversion of RCP ownership but there
is a way to enable this that is 100\% bullet-proof without having to
change the existing {}\texttt{A} or {}\texttt{B} classes or any other
code at all.  The way to do this is to use the
{}\texttt{rcpWithEmbeddedObj(...)} function as shown in Listing
{}\ref{listing:A_owns_B_owns_A}.

\begin{listing}: A factory function that returns a A object
embedded with a B object \\
\label{listing:A_owns_B_owns_A}
{\small\begin{verbatim}
  RCP<A> createAFactory(...)
  {
    RCP<B> b = createBFactory(...);
    return rcpWithEmbeddedObj(b->getA().getRawPtr(),
      std::make_pair(b->getA(), b), false);
  }
\end{verbatim}}
\end{listing}

Without going into a lot of detail, what the code in Listing
{}\ref{listing:A_owns_B_owns_A} accomplishes is that it defines a new
{}\texttt{RCP<A>} object with a new {}\texttt{RCPNode} object that
uses the other existing {}\texttt{RCP<A>} and {}\texttt{RCP<B>}
objects to define ownership and ensure that the underlying
{}\texttt{A} object and {}\texttt{B} do not go away until the last
{}\texttt{RCP<A>} object copied from the object returned by the
function {}\texttt{createAFactory(...)} has gone away.  The reason
that {}\texttt{false} is passed into the
{}\texttt{rcpWithEmbeddedObj(...)} is because it is the embedded
objects {}\texttt{RCP<A>} and {}\texttt{RCP<B>} that define the
deallocation and not the default behavior which calls
{}\texttt{delete}.  The reason that both {}\texttt{RCP<A>} and
{}\texttt{RCP<B>} are passed as an embedded object (stored in an
{}\texttt{std::pair} object) is that we need to make sure the
{}\texttt{A} object does not get deleted in case some client calls the
{}\texttt{B::unsetA()} function.

Given this data-structure, another piece of code can then extract the
underlying {}\texttt{RCP<B>} object as shown in Listing
{}\ref{listing:Extract_B_from_A}.

\begin{listing}: A function that extracts the B object from the A object \\
\label{listing:Extract_B_from_A}
{\small\begin{verbatim}
  RCP<B> extractBFromA(const RCP<A> &a)
  {
    typedef std::pair<RCP<A>, RCP<B> > ABPair_t;
    ABPair_t abPair = getEmbeddedObj<A, ABPair_t>(a);
    return abPair->second;
  }
\end{verbatim}}
\end{listing}

That is all there is to it.  Now this is not the sort of thing that
you want to expose to general clients but it can be very handy to have
this type of flexibility when implementing the guts of your library
code.  Note that the function {}\texttt{createAFactory(...)} in
Listing {}\ref{listing:A_owns_B_owns_A} has to expose a raw C++
pointer but it does so only in a single statement and therefore is not
to serious a violation of the idioms defined in this paper.  This
would be consider specialized low level code and is not recommended
unless it is absolutely necessary.

However, The example shown above just shows the flexibility of these
memory management classes and what some of the possibilities if you
understand the underlying reference-counting machinary a little.


%
{}\subsection{Summary of common problems and fundamental shortcomings}
%

ToDo: Review the material on the C++ memory model for implicit
conversions and why smart pointers will *always* be at a usage
disadvantage w.r.t. raw C++ pointers.  Point back to the section on
implicit conversions.


%
{}\subsection{Comparison with other class libraries and the standard
library}
%

ToDo: Discuss the similarities and differences between the Teuchos
classes and boost classes.


%
{}\subsection{Essential and accidental complexity}
%

While the idioms described in this document (outlined in Section
{}\ref{sec:idioms}) may look complex, one has to consider that it is
not really the idioms that are complex but the essential attributes of
object relationships that are complex.  Frederick Brooks refers to
this as {}\textit{essential complexity} as opposed to
{}\textit{accidental complexity} {}\cite{MythicalManMonth95}.
{}\textit{Accidental complexity} in programming refers to complexity
resulting from the details of the programming language or environment
which are not directly related to solving the problem at hand.
Accidental complexity has largely been removed as higher level
languages have been developed {}\cite[Chapter 16]{MythicalManMonth95}.
However, raw pointers in C and C++ and manual resource management
(when that is not the main focus of the program) are definitely a
lingering category of accidental
complexity\footnote{\texttt{http://discuss.joelonsoftware.com/default.asp?joel.3.278613.51}}.
Alternatively, {}\textit{essential complexity} exists because of the
nature of the problem at hand and no programming language will remove
it (but we can use object-oriented and other design approaches to
partition and abstract this essential complexity such that we can
write and maintain large-scale programs).

What the Teuchos Memory Management classes do is that they remove much
of the accidental complexity of using raw pointers and manual resource
management and instead they more directly address the essential
complexity of writing programs in making important concepts explicit
that are implicit in most languages (including raw C++).  Dealing with
the nature of relationships between objects is essential complexity
and for every relationship between two classes (for example in a UML
class diagram {}\cite{UMLDistilledThirdEdition04}) one must answer the
essential questions:

\begin{itemize}

{}\item\textit{What is the multiplicity of the relationship?}  (i.e.\
is there just one object or are there more than one object at the
other end of the association?).  In UML class diagrams, a singular
multiplicity relationship is represented using a {}\texttt{1} and
multiplicity greater than one is represented using {}\texttt{1..*}
(See Figure ??? for examples).

{}\item\textit{Is the object optional or required?}  In a UML class
diagram, an optional object is represented using {}\texttt{0..1} while
a required object is represented as {}\texttt{1} (see Figure ??? for
examples).

{}\item\textit{Is the object changeable or non-changeable?}  In UML
class diagrams, a non-changeable object is given the attribute
{}\texttt{\{readOnly\}}.  In UML, by default, all objects at the end
of an association are assumed changeable.

{}\item\textit{Is the association persisting or non-persisting?}  In a
UML class diagram, non-persisting associations are referred to as
``dependency associations'' and can be given the keyword
{}\texttt{$\ll$parameter$\gg$} and are represented with a dotted line.
Persisting associations are referred to as ``relationships'' and are
represented as solid lines (see Figure ??? for examples)?

\end{itemize}

Note that while UML is an expressive language that allows you to
explicitly represent the above essential information, most
programming languages cannot (at least not the raw language).
Consider that in Java and Python that it is impossible to distinguish
between persisting and non-persisting associations because every
user-defined object is always managed through an indirect reference
handed by the garbage-collected language.  This causes big problems
when it comes time to try to understand a complex program written in
these languages.  For example, consider the agony that Feathers goes
through in many refactorings described in [Feathers ???, Sections ??? 
and ???] in trying to determine the nature of objects as to whether
they are actually embedded in each other (persisting) or are just
passed to each other (non-persisting).  For instance, in [Feathers,
Section ???], the author tries to determine if the passed in object is
remembered or not and is going to be changed or not.  Python has no
user-definable concept of {}\texttt{const} but the Python language
itself understands the need for {}\texttt{const} by having built-in
immutable data-types like strings and tuples.

One of the goals of the idioms defined in this paper is to change the
above essential complexities from implicit concepts to explicit
concepts directly stated in code (see ``Making Implicit Concepts
Explicit'' in {}\cite[Chapter 9]{DomainDrivenDesign}).  The essential
attributes of object relationships (i.e.\ multiplicity, persistent
vs.\ non-persistent, changeable vs.\ non-changeable) are present in
every program no mater what high-level programming language is used
{}\cite{MythicalManMonth95, CodeComplete2nd04,
WorkingEffectivelyWithLegacyCode05}.  The issue is that most
executable languages (not withstanding executable XML {}\cite[Chapter
1]{UMLDistilledThirdEdition04}) lack the expressiveness to make these
concepts explicit.  The Teuchos memory management classes and the
associated idioms described in this paper provide a means to make many
of these essential concepts explicit in C++ in a way that is not
possible in any other widely used high-level programming language,
period.  

While the Teuchos memory management classes go a long way in
removing some of the accidental complexity of programing in C++ due to
manual memory management, some of the (of the many others) types of
remaining accidental complexity include:

\begin{itemize}

{}\item\textit{Value sematics verses\ reference semantics}: The
distinction between value semantics verses reference semantics in a
C++ concept that does not directly relate to solving a problem or
representing a model in code and is therefore accidental complexity.
In Java and Python, all user-defined types use reference semantics but
in C++ we can take advantage of objects with value semantics which
gives us more efficient code and more control in C++ than what is
possible in Java or Python.  However, this extra control could be
classified as Accedential complexity (which we tolerate for the sake
of added control and improved performance).

{}\item\textit{Pointer semantics for memory management types
{}\texttt{Ptr}, and {}\texttt{RCP}}: In order to access the underying
object through the types {}\texttt{Ptr}, and {}\texttt{RCP}, you have
to use pointer semantics using {}\texttt{*a} and
{}\texttt{a->someMember()}.  The C++ language makes it impossible to
allow direct access to the underlying type like raw C++ references
allow.  Pointer semantics are not essental to the nature of problem
solving (as proven by all the langauges that don't have pointers
including Java and Python) and must therefore be categorized as
accidental complexity.  Note, however, that the types
{}\texttt{ArrayView} and {}\texttt{ArrayRCP} were not listed in this
category because you can use these array classes using just the
{}\texttt{operator[](size\_type)} function and do not need to use any
pointer semantics.  In fact, {}\texttt{ArrayView} does not even
support any pointer-like functions and the pointer-like functions on
{}\texttt{ArrayRCP} are only really there to allow it to be used as a
general purpose checked iterator in a debug build.  While pointer
semantics are not an essential concept they do actually come in handy
to define iterators into containers and present a much more compact
iterator interface that what you will find in other languages.
Pointers semantics are not all bad.

\end{itemize}

%
{}\section{Conclusions}
%

Using reference-counting in Teuchos Memory Management classes allows
you to remove unnecessary constraints in the use of objects in
removing arbitrary lifetime ordering constraints which are a type of
unnecessary coupling {}\cite{CodeComplete2nd04}.  The code you write
with these classes will be more likely to be correct on first writing,
will be less likely to contain silent (but deadly) memory usage
errors, and will be much more robust to later refactoring and
maintenance.  These classes and idioms largely address the technical
issues in resolving the fragile built-in C++ memory management model
(with the exception of circular references which has no easy solution
but can be managed as discussed).  All that remains is to teach these
classes and idioms and expand their usage in C++ codes.  The long-term
viability of C++ as a usage-able and productive language depends on
it.  Otherwise, if C++ is no safer than C, then is the greater
complexity of C++ worth what you get as extra features?  C is smaller
and easier to learn and since most programmers don't know
object-orientation all that well anyway then what are they really
getting out of C++ other than more complexity?  C++ zealots will argue
this point but the reality is that C++ is becoming less popular while
the popularity of C is remaining fairly stable over the last 10 years.
I believe that idioms like are advocated here can help to avert this
trend but it will require wide community buy-in and a change in the
way we teach C++.

To make our programs more secure, compiler venders or static analysis
tools (e.g.\ klocwork) could implement a preprocessor-like language
similar to OpenMP that would allow the programmer to declare (in
comments) that certain blocks of code should be ``pointer-free'' or
allow smaller blocks to be ``pointers allowed''.  This would
significantly improve the robustness of the code.



% ---------------------------------------------------------------------- %
% References
%

\clearpage
% If hyperref is included, then \phantomsection is already defined.
% If not, we need to define it.
\providecommand*{\phantomsection}{}
\phantomsection
\addcontentsline{toc}{section}{References}
\bibliographystyle{plain}
\bibliography{references}

% ---------------------------------------------------------------------- %
% Appendices should be stand-alone for SAND reports. If there is only
% one appendix, put \setcounter{secnumdepth}{0} after \appendix
%
\appendix


%
{}\section{Summmary of the idioms for using the Teuchos memory
management classes}
%

\begin{center}

\begin{tabular}{|l|l|l|}
%
\multicolumn{3}{c}{\textbf{Equivalencies for const production for raw and smart pointers}} \\
\hline
Description
& Raw pointer 
& Smart pointer \\
\hline
\hline
Basic declaration (non-const obj)
& {}\texttt{typedef A* ptr\_A}
& {}\texttt{RCP<A>} \\
\hline
Basic declaration (const obj)
& {}\texttt{typedef const A* ptr\_const\_A}
& {}\texttt{RCP<const A>} \\
\hline
\hline
non-const pointer, non-const object
& {}\texttt{ptr\_A}
& {}\texttt{RCP<A>} \\
\hline
const pointer, non-const object
& {}\texttt{const ptr\_A}
& {}\texttt{const RCP<A>} \\
\hline
non-const pointer, const object
& {}\texttt{ptr\_const\_A}
& {}\texttt{RCP<const A>} \\
\hline
const pointer, const object
& {}\texttt{const ptr\_const\_A}
& {}\texttt{const RCP<const A>} \\
\hline
\end{tabular}


\begin{tabular}{|l|l|}
%
\multicolumn{2}{c}{\textbf{Class Data Members for Value-Type Objects}} \\
%
\hline
Data member purpose
& Data member delcaration \\
\hline
\hline
non-shared, single, const object
& {}\texttt{const S s\_;} \\
\hline
non-shared, single, non-const object
& {}\texttt{S s\_;} \\
\hline
non-shared array of non-const objects
& {}\texttt{Array<S> as\_;} \\
\hline
shared array of non-const objects
& {}\texttt{RCP<Array<S> > as\_;} \\
\hline
non-shared statically sized array of non-const objects
& {}\texttt{Tuple<S,N> as\_;} \\
\hline
shared statically sized array of non-const objects
& {}\texttt{RCP<Tuple<S,N> > as\_;} \\
\hline
shared fixed-sized array of const objects
& {}\texttt{ArrayRCP<const S> as\_;} \\
\hline
shared fixed-sized array of non-const objects
& {}\texttt{ArrayRCP<S> as\_;} \\
\hline
\end{tabular}


\begin{tabular}{|l|l|}
%
\multicolumn{2}{c}{\textbf{Class Data Members for Reference-Type Objects}} \\
%
\hline
Data member purpose
& Data member delcaration \\
\hline
\hline
non-shared or shared, single, const object
& {}\texttt{RCP<const A> a\_;} \\
\hline
non-shared or shared, single, non-const object
& {}\texttt{RCP<A> a\_;} \\
\hline
non-shared array of shared const objects
& {}\texttt{Array<RCP<const A> > aa\_;} \\
\hline
non-shared array of shared non-const objects
& {}\texttt{Array<RCP<A> > aa\_;} \\
\hline
shared fixed-sized array of shared const objects
& {}\texttt{ArrayRCP<RCP<const A> > aa\_;} \\
\hline
``...'' (const ptr)
& {}\texttt{ArrayRCP<const RCP<const A> > aa\_;} \\
\hline
shared fixed-sized array of shared non-const objects
& {}\texttt{ArrayRCP<RCP<const A> > aa\_;} \\
\hline
``...'' (const ptr)
& {}\texttt{ArrayRCP<const RCP<const A> > aa\_;} \\
\hline
\end{tabular}


\pagebreak

\begin{tabular}{|l|l|}
%
\multicolumn{2}{c}{\textbf{Passing Value Objects, Non-Persisting Associations}} \\
%
\hline
Argument Purpose
& Formal Argument Declaration \\
\hline
\hline
single, non-changeable object (required)
& {}\texttt{S s} $\;\;\;\;$\textbf{or}$\;\;\;\;$ {}\texttt{const S s} $\;\;\;\;$\textbf{or}$\;\;\;\;$ {}\texttt{const S \&s} \\
%& \begin{minipage}{12ex}
%    {\tiny .}\\
%    {}\texttt{S s} \\
%    or \\
%    {}\texttt{const S s} \\
%    or \\
%    {}\texttt{const S \&s}
%    {\tiny .}
%  \end{minipage} \\
\hline
single, non-changeable object (optional)
& {}\texttt{const Ptr<const S> \&s} \\
\hline
single, changeable object (required)
& {}\texttt{const Ptr<S> \&s}  $\;\;\;\;$\textbf{or}$\;\;\;\;$ {}\texttt{S \&s} \\
%& \begin{minipage}{20ex}
%    {\tiny .}\\
%    {}\texttt{const Ptr<S> \&s} \\
%    or \\
%    {}\texttt{S \&s} \\
%    {\tiny .}
%  \end{minipage} \\
\hline
single, changeable object (optional)
& \texttt{const Ptr<S> \&s} \\
\hline
array of non-changeable objects
& {}\texttt{const ArrayView<const S> \&s} \\
\hline
array of changeable objects
& {}\texttt{const ArrayView<S> \&s} \\
\hline
%
\end{tabular} \\[3ex]
%
\begin{tabular}{|l|l|}
%
\multicolumn{2}{c}{\textbf{Passing Value Objects, Persisting Associations}} \\
%
\hline
Argument Purpose
& Formal Argument Declaration \\
\hline
\hline
array of non-changeable objects
& {}\texttt{const ArrayRCP<const S> \&s} \\
\hline
array of changeable objects
& {}\texttt{const ArrayRCP<S> \&s} \\
\hline
\end{tabular}


%
\begin{tabular}{|l|l|}
%
\multicolumn{2}{c}{\textbf{Passing IN Non-Persisting Associations to Reference (or Value) Objects as Func Args}} \\
%
\hline
Argument Purpose
& Formal Argument Declaration \\
\hline
\hline
single, non-changeable object (required)
& {}\texttt{const A \&a} \\
\hline
single, non-changeable object (optional)
& {}\texttt{const Ptr<const A> \&a} \\
\hline
single, changeable object (required)
& {}\texttt{const Ptr<A> \&a}  $\;\;\;\;$\textbf{or}$\;\;\;\;$ {}\texttt{A \&a} \\
%& \begin{minipage}{20ex}
%    {\tiny .}\\
%    {}\texttt{const Ptr<A> \&a} \\
%    or \\
%    {}\texttt{A \&a} \\
%    {\tiny .}
%  \end{minipage} \\
\hline
single, changeable object (optional)
& \texttt{const Ptr<A> \&a} \\
\hline
array of non-changeable objects
& {}\texttt{const ArrayView<const Ptr<const A> > \&aa} \\
\hline
array of changeable objects
& {}\texttt{const ArrayView<const Ptr<A> > \&aa} \\
\hline
\end{tabular} \\[3ex]
%
\begin{tabular}{|l|l|}
%
\multicolumn{2}{c}{\textbf{Passing IN Persisting Associations to Reference (or Value) Objects as Func Args}} \\
%
\hline
Argument Purpose & Formal Argument Declaration \\
\hline
\hline
single, non-changeable object
& {}\texttt{const RCP<const A> \&a} \\
\hline
single, changeable object
& {}\texttt{const RCP<A> \&a} \\
\hline
array of non-changeable objects
& {}\texttt{const ArrayView<const RCP<const A> > \&aa} \\
\hline
array of changeable objects
& {}\texttt{const ArrayView<const RCP<A> > \&aa} \\
\hline
\end{tabular} \\[3ex]
%
\begin{tabular}{|l|l|}
%
\multicolumn{2}{c}{\textbf{Passing OUT Persisting Associations for Reference (or Value) Objects as Func Args}} \\
%
\hline
Argument Purpose
& Formal Argument Declaration \\
\hline
\hline
single, non-changeable object
& {}\texttt{const Ptr<RCP<const A> > \&a} \\
\hline
single, changeable object
& {}\texttt{const Ptr<RCP<A> > \&a} \\
\hline
array of non-changeable objects
& {}\texttt{const ArrayView<RCP<const A> > \&aa} \\
\hline
array of changeable objects
& {}\texttt{const ArrayView<RCP<A> > \&aa} \\
\hline
\end{tabular} \\[3ex]
%
\begin{tabular}{|l|l|}
%
\multicolumn{2}{c}{\textbf{Passing OUT Semi-Persisting Associations for Reference (or Value) Objects as Func Args}} \\
%
\hline
Argument Purpose
& Formal Argument Declaration \\
\hline
\hline
single, non-changeable object
& {}\texttt{const Ptr<Ptr<const A> > \&a} \\
\hline
single, changeable object
& {}\texttt{const Ptr<Ptr<A> > \&a} \\
\hline
array of non-changeable objects
& {}\texttt{const ArrayView<Ptr<const A> > \&aa} \\
\hline
array of changeable objects
& {}\texttt{const ArrayView<Ptr<A> > \&aa} \\
\hline
\end{tabular}


\pagebreak

\begin{tabular}{|l|l|}
%
\multicolumn{2}{c}{\textbf{Returning Non-Persisting Associations to Value Objects}} \\
%
\hline
Purpose
& Return Type Declaration \\
\hline
\hline
Single copied object
& {}\texttt{S} \\
\hline
Single non-changeable object (required)
& {}\texttt{const S\&} \\
\hline
Single non-changeable object (optional)
& {}\texttt{Ptr<const S>} \\
\hline
Single changeable object (required)
& {}\texttt{S\&} \\
\hline
Single changeable object (optional)
& {}\texttt{Ptr<S>} \\
\hline
Array of non-changeable objects
& {}\texttt{ArrayView<const S>} \\
\hline
Array of changeable objects
& {}\texttt{ArrayView<S>} \\
\hline
%
\end{tabular} \\[3ex]
%
\begin{tabular}{|l|l|}
%
\multicolumn{2}{c}{\textbf{Returning Persisting Associations to Value Objects}} \\
%
\hline
Purpose
& Return Type Declaration \\
\hline
\hline
Array of non-changeable objects
& {}\texttt{ArrayRCP<const S>} \\
\hline
Array of changeable objects
& {}\texttt{ArrayRCP<S>} \\
\hline
\end{tabular} \\[3ex]
%
\begin{tabular}{|l|l|}
%
\multicolumn{2}{c}{\textbf{Returning Semi-Persisting Associations to Value Objects}} \\
%
\hline
Purpose
& Return Type Declaration \\
\hline
\hline
Array of non-changeable objects
& {}\texttt{ArrayView<const S>} \\
\hline
Array of changeable objects
& {}\texttt{ArrayView<S>} \\
\hline
\end{tabular}


\begin{tabular}{|l|l|}
%
\multicolumn{2}{c}{\textbf{Returning Reference-Type Objects, Non-Persisting Associations}} \\
%
\hline
Purpose
& Return Type Declaration \\
\hline
\hline
Single cloned object
& {}\texttt{RCP<A>} \\
\hline
Single non-changeable object (required)
& {}\texttt{const A\&} \\
\hline
Single non-changeable object (optional)
& {}\texttt{Ptr<const A>} \\
\hline
Single changeable object (required)
& {}\texttt{A\&} \\
\hline
Single changeable object (optional)
& {}\texttt{Ptr<A>} \\
\hline
Array of non-changeable objects
& {}\texttt{ArrayView<const Ptr<const A> >} \\
\hline
Array of changeable objects
& {}\texttt{ArrayView<const Ptr<A> >} \\
\hline
%
\end{tabular} \\[3ex]
%
\begin{tabular}{|l|l|}
%
\multicolumn{2}{c}{\textbf{Returning Reference-Type Objects, Persisting Associations}} \\
%
\hline
Purpose
& Return Type Declaration \\
\hline
\hline
Single non-changeable object
& {}\texttt{RCP<const A>} \\
\hline
Single changeable object
& {}\texttt{RCP<A>} \\
\hline
Array of non-changeable objects
& {}\texttt{ArrayView<const RCP<const A> >} \\
\hline
Array of changeable objects
& {}\texttt{ArrayView<const RCP<A> >} \\
\hline
\end{tabular}


\pagebreak

{}\textbf{Conversions for data-types for single objects}
\includegraphics*[angle=270,scale=0.65]{TeuchosPtrConversions} \\[5ex]



{}\textbf{Conversions for data-types for contiguous arrays}
\includegraphics*[angle=270,scale=0.65]{TeuchosArrayConversions}


\end{center}


%
\section{Teuchos memory management commandments (Guidelines)}
%

\begin{commandment}
Thou shall not expose raw pointers in any high-level C++ code.
\end{commandment}
{}\textit{Exception:} Only expose raw pointers when interfacing with
non-compliant code.

\begin{commandment}
Thou shall only use raw C++ references for non-persisting associations
(see Section {}\ref{sec:persisting-nonpersisting-associations}).
\end{commandment}

\begin{commandment}
Thou shall not call raw {}\texttt{new} or {}\texttt{delete} in any
high-level C++ code to dynamically allocate and destroy single
objects.  Instead, create memory using a user-defined non-member
constructor function (see Section
{}\ref{sec:nonmember-constructor-idiom}).
\end{commandment}

\begin{commandment}
Thou shall not call raw operator {}\texttt{new []} or {}\texttt{delete
[]} in any high-level C++ code to dynamically allocate and destroy
contiguous arrays of data.  Instead, use functions such as
{}\texttt{Teuchos::Array<T>(n)} and {}\texttt{Teuchos::arcp<T>(n)}
(see ???).
\end{commandment}

\begin{commandment}
Thou shall not directly create and use compile-time fixed sized arrays
with {}\texttt{T[N]}.  Instead, create compile-time fixed-sized arrays
using {}\texttt{Teuchos::Tuple<T,N>} and convert to
{}\texttt{Teuchos::ArrayView<T>} for more general usage.
\end{commandment}

\begin{commandment}
Thou shall use {}\texttt{Teuchos::Array} as a general purpose
contiguous container instead of {}\texttt{std::vector} (see Section
{}\ref{sec:Array}).
\end{commandment}

\begin{commandment}
Thou shall only convert between different memory management objects
using the provided implicit and excplicit conversion functions.  Thou
shall never expose a raw C++ pointer to perform a conversion.
\end{commandment}
{}\textit{Exception}: Some very advanced and rare use cases might have
you exposing a raw C++ pointer (see Section
{}\ref{sec:inverting-obj-ownership} for the only example described in
this paper).

\begin{commandment}
Thou shall only pass in the types {}\texttt{Teuchos::Ptr},
{}\texttt{Teuchos::RCP}, {}\texttt{Teuchos::ArrayView}, and
{}\texttt{Teuchos::ArrayRCP} by constant reference (e.g.\
{}\texttt{const RCP<T> \&a} and never by non-const reference (e.g.\
never do {}\texttt{RCP<T> \&a}.
\end{commandment}
{}\textit{Exception}: The only time you should ever pass in a
non-const reference to one of these types (e.g.\ {}\texttt{RCP<T> \&a})
is when the function will modify what data the object points to.
However, if this is the case, it is typically better and more clear to
pass in the object through a {}\texttt{Teuchos::Ptr} object (e.g.\
{}\texttt{const Ptr<RCP<T> > \&a}).


%
{}\section{Argument for using an signed integer for
{}\texttt{size\_type} in the Teuchos array classes}
\label{sec:unsigned_size_type}
%

ToDo: Add discussion of why the Teuchos array classes all use a signed
integer for size\_type and why that is better.  This is important
because of the break from std::vector.




\begin{SANDdistribution}[NM]
% \SANDdistCRADA	% If this report is about CRADA work
% \SANDdistPatent	% If this report has a Patent Caution or Patent Interest
% \SANDdistLDRD	% If this report is about LDRD work
% External Address Format: {num copies}{Address}
%\SANDdistExternal{}{}
%\bigskip
%% The following MUST BE between the external and internal distributions!
%\SANDdistClassified % If this report is classified
% Internal Address Format: {num copies}{Mail stop}{Name}{Org}
%\SANDdistInternal{}{}{}{}
% Mail Channel Address Format: {num copies}{Mail Channel}{Name}{Org}
%\SANDdistInternalM{}{}{}{}
%\SANDdistInternal{2}{MS 9018}{Central Technical Files}{8944}
%\SANDdistInternal{2}{MS 0899}{Technical Library}{4536}
\end{SANDdistribution}

\end{document}
