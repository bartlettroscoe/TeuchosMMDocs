%
\section{Passing objects to and from functions: details}
\label{rcp:apdx:passing_args}
%

The purpose of this section is to explain the reasoning behind the
idiom recommendations given in Table {}\ref{rcp:tbl:fnc-decl} for how
to pass objects to and from C++ functions and how
{}\texttt{Ref\-Count\-Ptr<>} fits in with these idioms.  First it
should be stated that many different code projects do not follow these
idioms.  Usually this is because the developers where unaware of these
idioms and why they are useful to follow.  Even if a developer is
familiar with these idioms for raw data types, the may not be as
familiar with how {}\texttt{Ref\-Count\-Ptr<>} fits in.

As stated in Section {}\ref{rcp:sec:passing-args}, it is recommended
to use raw object references for passing transitive arguments
(i.e.~those that are only accessed during a function call) to a
function.  Here, the term reference is used in a more abstract sense
where a reference can be implemented as a C++ references such as
{}\texttt{const A \&a} or as a C++ pointer such as {}\texttt{A *a}.
On the other hand, it is recommended to pass
{}\texttt{Ref\-Count\-Ptr<>} wrapped object references
(i.e.~{}\texttt{const Ref\-Count\-Ptr<A> \&a}) for any non-transitive
arguments (i.e.~those where a ``memory'' of the object in the server
implementing the function persists after the function returns).

Passing {}\texttt{Ref\-Count\-Ptr<>} wrapped object references to and
from functions will differ slightly from how raw object references are
passed.  As a rule of thumb when thinking about how to use
{}\texttt{Ref\-Count\-Ptr<>} the pass references to and from
functions, one should think of a {}\texttt{Ref\-Count\-Ptr<>} object
as though it was really just a raw pointer.  The main difference is
that while raw pointers to single objects are usually passed by value
(i.e.~{}\texttt{A *a}) to a function (which is very efficient) a
{}\texttt{Ref\-Count\-Ptr<>} wrapped single object on the other hand
should usually be passed by {}\texttt{const} reference
(i.e.~{}\texttt{const Ref\-Count\-Ptr<A> \&a}).  It is important to
pass single {}\texttt{Ref\-Count\-Ptr<>} object by {}\texttt{const}
reference and not by non-{}\texttt{const} reference
(i.e.~{}\texttt{Ref\-Count\-Ptr<A> \&a}) since the compiler will only
perform automatic conversions for {}\texttt{const} references.
However, passing {}\texttt{Ref\-Count\-Ptr<>} objects by value is
harmless but passing {}\texttt{Ref\-Count\-Ptr<>} objects by
{}\texttt{const} reference eliminates unnecessary (while quite trivial)
constructor and destructor function calls.  In any case, passing
{}\texttt{Ref\-Count\-Ptr<>} objects by {}\texttt{const} reference as
apposed to by value simplifies stepping through code in a debugger and
should be preferred if only for this reason.  On the other hand, when
passing arrays of objects wrapped within {}\texttt{Ref\-Count\-Ptr<>}
objects, it is recommended to pass an array of
{}\texttt{Ref\-Count\-Ptr<>} objects (i.e.~{}\texttt{const
Ref\-Count\-Ptr<A> a[]} and not an array of
{}\texttt{Ref\-Count\-Ptr<>} references.  This is discussed below.

Before getting into great detail as to the why and how of the
recommendations in Table {}\ref{rcp:tbl:fnc-decl} and exactly how
{}\texttt{Ref\-Count\-Ptr<>} fits in, consider the following example
function prototype.

{\scriptsize\begin{verbatim}
void foo1(
  const int              size_x
  ,const double          x[]
  ,const int             size_a
  ,const A*              a[]
  ,const int             size_y
  ,double                y[]
  ,double                *z
  );
\end{verbatim}}

By comparing the argument declarations (which use good argument names)
in the above function to the recommendations in Table
{}\ref{rcp:tbl:fnc-decl}, and without looking at any extra
documentation (if it even exists), we already know a lot about the
arguments being passed into and out of this function.  First, it is
clear that {}\texttt{x} is an array of non-mutable {}\texttt{double}
objects with an presumed length of {}\texttt{size\_x}.  Similarly, we
know that {}\texttt{y} is an array of mutable {}\texttt{double}
objects that may be set or modified.  We also know that {}\texttt{z}
is a single mutable {}\texttt{double} object that may be changed
inside of the function.  What the above C++ declarations, however, do
not tell us is whether the arguments {}\texttt{x}, {}\texttt{a},
{}\texttt{y} and {}\texttt{z} are optional (i.e.~a {}\texttt{NULL}
pointer can be passed in for them) or are required (i.e.~a
non-{}\texttt{NULL} pointer can be passed in for them).  We also do
not know if the mutable objects passed in {}\texttt{y} and
{}\texttt{z} are only output arguments or are input/output arguments.
This type of information must be specified in developer-supplied
documentation.  Note that while a required single non-mutable object
such as {}\texttt{z} could be passed using a non-{}\texttt{const}
reference, using a non-{}\texttt{const} reference for such an argument
is usually not advised and this issue is addressed a little later.

Note that the above function {}\texttt{foo(...)} involved all
transitive (i.e.~only accessed with the function) arguments which is
often the case.  Now lets consider the use of non-transitive argument
types as supported using {}\texttt{Ref\-Count\-Ptr<>}.

{\scriptsize\begin{verbatim}
class SomeClass {
public:
  ...
  void foo2(
    const int                                 size_x
    ,double                                   x[]
    ,const RefCountPtr<std::vector<double> >  &y
    )
  {
    y_ = y;
    ...
  }
  ..
private:
  RefCountPtr<std::vector<double> >  y_;
};
\end{verbatim}}

{}\noindent{}The above example function
{}\texttt{SomeClass\-::foo2(...)} is an example where a non-transitive
argument, {}\texttt{y}, is passed into a function and is then
``remembered''.  In this case, the {}\texttt{Ref\-Count\-Ptr<>}
argument {}\texttt{y} is delivering a
{}\texttt{std\-::vector<\-double>} object to be used as private data
for the class object.

%
\subsection{Passing non-mutable objects}
%

Non-mutable transitive objects (i.e.~objects that are passed into a
function, are not modified and no memory of these objects is
maintained) can be passed by value or by reference (i.e.~through a C++
pointer or a C++ reference) and as either single objects or as arrays.
When non-mutable objects are passed to a function using a C++ pointer
or a C++ reference, the declaration should always use the
{}\texttt{const} modifier.  Failure to use the {}\texttt{const}
modifier in these cases goes against the design of the C++ language.
The use of the {}\texttt{const} modifier both serves as documentation
and helps the implementor of the function avoid accidentally changing
the object.  However, when a small concrete object of type
{}\texttt{S} (where {}\texttt{S} is a {}\texttt{double} or
{}\texttt{int} for instance) is being passed it may also be passed by
value (with or without the {}\texttt{const} modifier as shown in Table
{}\ref{rcp:tbl:fnc-decl}).  When a concrete object is passed by value,
the {}\texttt{const} modifier does not in any way affect the client
code that calls the function and the client's copy of the argument is
guaranteed not to be changed regardless of whether the
{}\texttt{const} modifier is used or not.  On the other hand, the
{}\texttt{const} modifier on a C++ pointer or C++ reference does not
really guarantee that the object will not be changed.  The
{}\texttt{const} modifier used with pass-by-value only restricts the
modification of the copied argument in the function implementation.
Some see the use of {}\texttt{const} pass-by-value as a means to help
function implementors from making mistakes while others see this as a
unnecessary restriction on the freedom in the implementation of the
function.

When a non-mutable input argument is required, it should be passed by
value (i.e.~{}\texttt{S s}) or by {}\texttt{const} C++ reference
(i.e.~{}\texttt{const S \&s}).  However, when a non-mutable input
argument is optional, it must be passed using a {}\texttt{const} C++
pointer (i.e.~{}\texttt{const S *s}) since this pointer is be allowed
to be {}\texttt{NULL} for when the argument is not specified.

Arrays of non-mutable objects should either be passed as an array of
objects (i.e.~in the case where the object's type is a small concrete
data type such as {}\texttt{double}) such as {}\texttt{S s[]} or
{}\texttt{const S s[]}, or as an array of {}\texttt{const} C++
pointers (i.e.~in the case where the type is a large object or the
type has reference semantics) such as {}\texttt{const S* s[]}.  Note
that the syntax {}\texttt{const S s[]} (or {}\texttt{const A* a[]})
for passing an array of objects is to be preferred over the C++
pointer syntax {}\texttt{const S *s} (or {}\texttt{const A** a}).
This is because that even though the raw C++ pointer syntax
{}\texttt{const S *s} is perfectly legal and proper C++ usage, this
form does not allow one to tell if this pointer is supposed to point
to a single object or to an array of objects and therefore the syntax
{}\texttt{const S s[]} is to be preferred.  Also note that the syntax
{}\texttt{const A* a[]} allows individual components of
{}\texttt{a[i]} to be optional (i.e.~{}\texttt{a[i]==NULL} allowed) or
required (i.e.~{}\texttt{a[i]!=NULL} required) in addition to the
entire array being optional (i.e.~{}\texttt{a==NULL} allowed) or
required (i.e.~{}\texttt{a!=NULL} required).

Note that while it is possible to pass objects as an array of object
references such as {}\texttt{const S\& s[]}, it is generally very
difficult to initialize the references right when the array is first
created (as is required by the standard).  Therefore it is not
recommended to passed objects as an array of references but instead to
use an array of pointers as shown in Table {}\ref{rcp:tbl:fnc-decl}.

Note that unlike in the case of passing single non-mutable objects,
that there is no a convenient way to help differentiate whether an
array of objects being passed to a function is optional or not.  When
passing arrays of objects, it must be documented whether the objects
are required or optional.

Non-mutable non-transitive objects (i.e.~objects that are passed into
a function, are not modified but some memory of these objects is
maintained after the function call) should be passed through
{}\texttt{Ref\-Count\-Ptr<>} wrapped objects.  Single object
references of this type should be passed as {}\texttt{const
Ref\-Count\-Ptr<const A> \&a}.  Note the presences of the
{}\texttt{const} templated type {}\texttt{<const A>} which states that
clients may not modify the underlying wrapped object (see Section
{}\ref{rcp:sec:init-rcp-objects} for an explanation of how to express
combinations of {}\texttt{const} or non-{}\texttt{const} pointers to
{}\texttt{const} or non-{}\texttt{const} objects using
{}\texttt{Ref\-Count\-Ptr<>}).  As stated earlier, the
{}\texttt{Ref\-Count\-Ptr<>} object itself should be passed as a
{}\texttt{const} reference to avoid unnecessary constructor and
destructor calls.  Note that unlike passing single raw object
references to a function, there is no easy way at compile time to
differentiate between a required and an optional argument when passing
a {}\texttt{Ref\-Count\-Ptr<>} wrapped object.  For the argument
{}\texttt{const Ref\-Count\-Ptr<const A> \&a}, if it is required the
precondition {}\texttt{a.get()!=NULL} should be documented while if
the argument is optional then it should be documented that
{}\texttt{a.get()==NULL} is allowed.

Passing an array of non-mutable non-transitive objects wrapped in
{}\texttt{Ref\-Count\-Ptr<>} objects is a little different.  While it
is recommended to passing single object references using
{}\texttt{const} {}\texttt{Ref\-Count\-Ptr<>} C++ references, an array
of {}\texttt{Ref\-Count\-Ptr<>} object should be passed as an array of
{}\texttt{Ref\-Count\-Ptr<>} objects as {}\texttt{const
Ref\-Count\-Ptr<const A> a[]} rather than trying to pass an array of
C++ references to {}\texttt{Ref\-Count\-Ptr<>} objects such as
{}\texttt{const Ref\-Count\-Ptr<const A>\& a[]}.  As stated earlier,
it is extremely difficult to work with arrays of references.  Note
that just as is the case when passing an array of raw object
references, there is no good way to specify at compile time whether
the entire array {}\texttt{Ref\-Count\-Ptr<>} objects is required or
is optional and this instead must be explicitly documented.  Note also
that just in in the case for passing arrays of raw object references,
the declaration {}\texttt{const Ref\-Count\-Ptr<const A> a[]} allows
individual components of {}\texttt{a[i]} to be optional
(i.e.~{}\texttt{a[i].get()==NULL} allowed) or required
(i.e.~{}\texttt{a[i].get()!=NULL} required) in addition to the entire
array being optional (i.e.~{}\texttt{a==NULL} allowed) or required
(i.e.~{}\texttt{a!=NULL} required).

%
\subsection{Passing mutable objects}
%

Many of the issues involved with passing mutable objects (i.e.~client
maintained objects that are modified within a function) are the same as
for passing non-mutable objects such as described above but there are
some differences.  First off, objects that are to have their state
changed within a function must be passed by reference (i.e.~C++
pointer or C++ reference) and can not be passed by value.  In the case
of mutable arguments, non-{}\texttt{const} instead of {}\texttt{const}
C++ pointers are used to pass raw object references.  Likewise,
{}\texttt{Ref\-Count\-Ptr<A>} objects are used instead of
{}\texttt{Ref\-Count\-Ptr<const A>} objects.  Other than that, all of
the issues are exactly the same as in the case of passing non-mutable
objects.

One apparent area of controversy is the issue of whether
non-\texttt{const} reference or non-\texttt{const} pointer arguments
should be used for passing mutable objects to a function.  If the
argument is optional, then the choice is clear and the argument should
be declared as a non-\texttt{const} pointer.  However, if the argument
is required then some would say that a non-\texttt{const} reference
should always be used.  Interested readers should note, however, that
references were added to C++ primarily to support operator overloading
(see {}\cite[Section 3.7]{ref:design_evol_cpp}) and also for return
values.  If one is writing an overloaded operator function that
modifies one of its arguments (such as the stream insertion and
extraction operators {}\texttt{std::istream\& operator<<(std::istream
\&i, ... )} and {}\texttt{std::ostream\& operator>>(std::ostream \&o,
... )} for instance) then a non-\texttt{const} reference argument must
be used (and note the non-\texttt{const} reference return type as
well).  When the function is not an overloaded operator function then
the choice is not so clear.  Strustroup (the original inventor of C++
{}\cite{ref:design_evol_cpp}) makes a case for using
non-\texttt{const} pointers for mutable arguments in most non-operator
functions in {}\cite[Section 5.5]{ref:stroustrup_2000}.  In short, the
use of non-{}\texttt{const} C++ references over non-{}\texttt{const}
C++ pointer arguments for required mutable objects gives a false sense
of security and in the end does not really guarantee better quality
software.  While that standard states that every C++ reference must be
bound to a valid C++ object this is not guaranteed in any way.  For example,
consider the following program that will segfault when run.

{\scriptsize\begin{verbatim}

#include <iostream>

void f( int& i )
{
  i = 5;
  std::cout << "\ni = "<<i<<std::endl;
}

void g( int* i )
{
  f(*i);
}

int main()
{
  int a  = 2, *b = NULL;
  g(&a);
  g(b);
  return 0;
}

\end{verbatim}}

The above program can be put into a file ({}\texttt{bad\_ref.cpp} for
instance) and be compiled and linked with no errors by any standard
compliant C++ compiler (for instance using g++ as {}\texttt{g++ -o
bad\_ref.exe bad\_ref.cpp}).  However, when this program is run (as
{}\texttt{./bad\_ref.exe} for instance) the output will look something
like:

{\scriptsize\begin{verbatim}

#./bad_ref.exe

i = 5
Segmentation fault (core dumped)

\end{verbatim}}

The reason for the segmentation fault should be quite obvious.  In the
case of the above flat program, a very smart compiler might be able to
produce a warning but in the general case, where the main program and
the functions {}\texttt{f()} and {}\texttt{g()} are compiled in
separate source files, there is no way a C++ compiler can possibly
catch this error at compile time.  This example just proves that the
language does not guarantee that C++ references are bound to valid
objects.

Since non-{}\texttt{const} C++ references don't really give any more
guarantees than non-{}\texttt{const} C++ pointers, the chose of
whether to use non-{}\texttt{const} C++ references over
non-{}\texttt{const} C++ pointers should be made on other grounds.
The primary justification for using non-{}\texttt{const} C++ pointer
arguments for mutable objects is to force a different specification
for non-mutable and mutable objects being passed into a C++ function
that both give visual clues of what objects are going to be changed
and provides some compile-time checking if arguments change from being
non-mutable to mutable.  For example, consider the following program

{\scriptsize\begin{verbatim}

#include <iostream>

void f( const std::vector<int> &i, const std::vector<int> &j, int *k );

int main()
{
  std::vector<int>  i(5,0), j(5,1);
  int k;
  ...
  f( i, j, &k );
  ...
  std::foreach( j.begin(), j.end(), ... );
  ...
  return 0;
}

\end{verbatim}}

In the above program, even if you never even saw the prototype
for the function {}\texttt{f()}, you can assume that the argument
{}\texttt{k} is being modified by the function because of the visual
clue {}\texttt{\&k} in the statement

{\scriptsize\begin{verbatim}
  f( i, j, &k );
\end{verbatim}}

{}\noindent{}while rightly assuming that the arguments {}\texttt{i}
and {}\texttt{j} will not be modified.  Later on in this program,
the statement

{\scriptsize\begin{verbatim}
  std::foreach( j.begin(), j.end(), ... );
\end{verbatim}}

{}\noindent{}uses the array {}\texttt{j}, which is assumed to still be
in its initialized state.  Now suppose that some smarty pants developer
decides that he can implement the function {}\texttt{f()} better if
the array in {}\texttt{j} is used as workspace.  If non-\texttt{const}
references are the norm, then this developer may decide to change the
prototype of {}\texttt{f()} to

{\scriptsize\begin{verbatim}
void f( const std::vector<int> &i, std::vector<int> &j, int *k );
\end{verbatim}}

{}\noindent{}which now allows the function to use {}\texttt{j} as
workspace.  Removing a {}\texttt{const} modifier from a function like
this may cause some calls to this function to not compile (which is a
very good thing) but calls like in the above program will still
compile just fine.  If this change is made to the function
{}\texttt{f()} and the argument {}\texttt{j} is silently changed
without any clue to the code in {}\texttt{main()}, then the behavior
of the rest of the code that follows the call to {}\texttt{f()} may be
silently changed (such as was the case in this function).  This could
be a very hard bug to track down.

On the other hand, if non-\texttt{const} C++ pointer arguments are
always used for mutable objects, then the prototype for {}\texttt{f()}
must be changed to 

{\scriptsize\begin{verbatim}
void f( const std::vector<int> &i, std::vector<int> *j, int *k );
\end{verbatim}}

{}\noindent{}if the argument {}\texttt{j} is to be used a workspace.
In this case, the function call

{\scriptsize\begin{verbatim}
  f( i, j, &k );
\end{verbatim}}

{}\noindent{}would not compile and the developer that maintains this
code would have a great opportunity to change the calling code to cope
with the fact that {}\texttt{j} is no longer a non-mutable argument.
In this case, either a copy of {}\texttt{j} could be created to pass
into {}\texttt{f()} or the logic after the call to {}\texttt{f()}
could no longer assume that {}\texttt{j} is unchanged.

In summary, all things being equal, tracking down runtime errors
associated with using raw non-\texttt{const} C++ pointer arguments is
easier than tracking down runtime errors that result from using
non-\texttt{const} C++ reference arguments and using
non-\texttt{const} C++ pointer arguments give a visual clue as to what
arguments are being changed and what arguments are being left
unmodified.  This is a subjective opinion but is one that is held by
many esteemed C++ experts (including Strustroup himself).

However, while C++ references where {}\underline{not} purposefully
added to the language to support non-\texttt{const} pass-by-reference
in non-operator functions, there are circumstances where using
non-\texttt{const} reference arguments is to be preferred.  An example
might be one where the fact that an argument is going to be changed as
a result of a function call is obvious from the name of the function.
This is the case, for instance, in the non-member, non-operator function

{\scriptsize\begin{verbatim}
template<class T2, class T1> RefCountPtr<T2>  rcp_const_cast(const RefCountPtr<T1>& p1);
\end{verbatim}}

{}\noindent{}where it is obvious that when called like:

{\scriptsize\begin{verbatim}
void f( const RefCountPtr<const A> &a )
{
  g( rcp_const_cast<A>(a) );
}
\end{verbatim}}

{}\noindent{}that the underlying reference-counted object buried in
{}\texttt{a} is likely to be modified as a result of this function
call.

The approach advocated here is to prefer non-\texttt{const} C++
pointer arguments in non-operator functions unless there is a
compelling reason to use a non-\texttt{const} C++ reference instead
such as when there is unlikely to be any confusion whether the
argument is going to be changed.  Making this determination requires
experience, taste and in the end is a subjective decision.

While it was stated earlier that {}\texttt{Ref\-Count\-Ptr<>} should
almost always be passed as {}\texttt{const Ref\-Count\-Ptr<>\&}, there
is one use case where a pointer to a {}\texttt{Ref\-Count\-Ptr<>}
object should be passed to a function instead and that is when the
calling client code is to acquire a {}\texttt{Ref\-Count\-Ptr<>} to an
object held by the server code.  When only a single
{}\texttt{Ref\-Count\-Ptr<>} is to be acquired for a single object,
then it usually best to just return a {}\texttt{Ref\-Count\-Ptr<>}
object from function as its return type such as in the case

{\scriptsize\begin{verbatim}
RefCountPtr<Base> createBase(bool someFlag);
\end{verbatim}}

{}\noindent{}described in Section {}\ref{rcp:sec:extra-data}.
However, when more than one {}\texttt{Ref\-Count\-Ptr<>} object is to
be initialized and given to a client, then one should consider passing
pointers to the {}\texttt{Ref\-Count\-Ptr<>} objects to a function
(see the function {}\texttt{uninitialize()} in Appendix
{}\ref{rcp:apdx:sep-construct-init}).
